\documentclass[acmsmall]{acmart}\settopmatter{printfolios=true}

\usepackage{microtype}
\usepackage{textcomp}
\usepackage[scaled=0.8]{DejaVuSansMono}
\usepackage{setspace}
\usepackage{stackengine}
\usepackage{wrapfig}
\renewcommand\stacktype{L}
%\linespread{0.95}

\usepackage{balance}
\usepackage{moresize}
\usepackage{csquotes}
\usepackage{upgreek}

\usepackage{amsmath}
\usepackage{amssymb}
\usepackage{mathtools}
\usepackage{stmaryrd}

\usepackage{mathpartir}
\usepackage{galois}

\usepackage{mdframed}

\newcommand\lamif{$\lambda$IF}

\definecolor{IdentifierColor}{rgb}{0.15,0.15,0.50}
\newcommand{\ID}[1]{{\color{IdentifierColor}#1}}

\definecolor{PuncColor}{rgb}{0.52,0.24,0.14}
\newcommand{\PN}[1]{{\color{PuncColor}#1}}

\definecolor{KeywordColor}{rgb}{0.52,0.24,0.14}
\newcommand{\KY}[1]{{\color{KeywordColor}#1}}

\definecolor{ValueColor}{rgb}{0.13,0.55,0.13}
\newcommand{\VL}[1]{{\color{ValueColor}#1}}

\definecolor{ResultColor}{rgb}{0.10,0.10,0.79}
\newcommand{\RE}[1]{{\color{ResultColor}#1}}

\definecolor{ErrorColor}{rgb}{0.75,0.10,0.10}
\newcommand{\EO}[1]{{\color{ErrorColor}#1}}


\usepackage{listings}
\lstset%
  {language=Lisp
  ,upquote=true
  ,mathescape=true
  ,basicstyle=\ttfamily\color{PuncColor}
  ,alsoletter=+-*/?'\#0123456789^:
  ,identifierstyle=\color{IdentifierColor}
  ,keywords=
    {define,let,match,if
    ,do,define-monad,for/monad+
    ,\#lang
    ,\#:when
    ,\#:unless
    }
  ,keywordstyle=\color{KeywordColor}
  ,emph=
    {return,bind
    ,zero?
    ,ask-env,local-env
    ,ext,find,alloc
    ,get-store,put-store,update-store
    ,tell
    ,get-dead,put-dead,update-dead
    ,fail,mplus
    ,get-,put-,update-
    ,ask-,local-
    ,get-path-cond,refine
    ,ask-cache-in,local-cache-in
    ,get-cache-out,put-cache-out,update-cache-out
    }
  ,emphstyle=\color{IdentifierColor}\emph
  ,literate=
    {‹λ›}{{{\KY{«\uplambda»}}}}1
    {‹δ›}{{{\ID{«\updelta»}}}}1
    {‹σ›}{{{\ID{«\upsigma»}}}}1
    {‹ς›}{{{\ID{«\varsigma»}}}}1
    {‹ρ›}{{{\ID{«\uprho»}}}}1
    {‹α›}{{{\ID{«\upalpha»}}}}1
    {‹ψ›}{{{\ID{«\uppsi»}}}}1
    {‹φ›}{{{\ID{«\upphi»}}}}1
    {‹θ›}{{{\ID{«\uptheta»}}}}1
    {‹Σ›}{{{\ssmall\ID{«Σ»}}}}1
    {‹∅›}{{{\ID{«∅»}}}}1
    {‹←›}{{{\scriptsize«←»}}}1
    {‹≔›}{{{\scriptsize«≔»}}}1
    {‹₀›}{{{\ID{«\,⸤⦑0⦒⸥»}}}}1
    {‹₁›}{{{\ID{«\,⸤⦑1⦒⸥»}}}}1
    {‹₂›}{{{\ID{«\,⸤⦑2⦒⸥»}}}}1
    {‹′›}{{{\ID{«\,′»}}}}1
    {‹¢›}{{{\ID{\$}}}}1
    {‹∈›}{{{\ID{«∈»}}}}1
    {‹×›}{{{\ssmall\ID{«×»}}}}1
    {‹⊥›}{{{\ssmall\ID{«⊥»}}}}1
    {‹¬›}{{{\ssmall\ID{«¬»}}}}1
    {‹ᴵ›}{{{\ID{\,«⸢⦑I⦒⸣»}}}}1
    {‹ᴼ›}{{{\ID{\,«⸢⦑O⦒⸣»}}}}1
    {‹⁺›}{{{\ID{\,«⸢⦑+⦒⸣»}}}}1
    {‹⸢›in‹⸣›}{{{\ID{«⸢⦑in⦒⸣»}}}}1
    {‹⸢›out‹⸣›}{{{\ID{«⸢⦑out⦒⸣»}}}}2
    {`}{{{\`{}}}}1
    {get-‹¢›}{{{\ID{\emph{get-\$}}}}}5
    {put-‹¢›}{{{\ID{\emph{put-\$}}}}}5
    {update-‹¢›}{{{\ID{\emph{update-\$}}}}}8
    {=}{{{\ID{=}}}}1
  ,classoffset=1
  ,keywords=
    {'add1,'sub1,'+,'-,'*,'/,'failure,'N
    ,\#t,\#f,0,1,2,3,4,5,6,7,8,9
    }
  ,keywordstyle=\color{ValueColor}
  }
\lstdefinestyle{result}
  {basicstyle=\ttfamily\color{ResultColor}
  ,keywordstyle=\color{ResultColor}
  ,identifierstyle=\color{ResultColor}
  ,literate=
    {‹λ›}{{{«\uplambda»}}}1
    {‹δ›}{{{«\updelta»}}}1
    {‹σ›}{{{«\upsigma»}}}1
    {‹ς›}{{{«\varsigma»}}}1
    {‹ρ›}{{{«\uprho»}}}1
    {‹φ›}{{{«\upphi»}}}1
    {‹θ›}{{{«\uptheta»}}}1
    {‹Σ›}{{{\ssmall«Σ»}}}1
    {‹∅›}{{{«∅»}}}1
    {‹←›}{{{«←»}}}1
    {‹≔›}{{{«≔»}}}1
    {‹₀›}{{{«\,⸤⦑0⦒⸥»}}}1
    {‹₁›}{{{«\,⸤⦑1⦒⸥»}}}1
    {‹₂›}{{{«\,⸤⦑2⦒⸥»}}}1
    {‹′›}{{{«\,′»}}}1
    {‹¢›}{{{\$}}}1
    {‹∈›}{{{«∈»}}}1
    {‹×›}{{{\ssmall«×»}}}1
    {‹⊥›}{{{\ssmall«⊥»}}}1
    {‹¬›}{{{\ssmall«¬»}}}1
    {timeout}{{{\EO{\emph{\textbf{timeout}}}}}}7
  }

\newcommand{\rfloat}[1]{\begin{flushright}\fbox{#1}\end{flushright}\vspace{-2.1em}} % -2.1em}}
\newcommand{\resultskip}{\vspace{-0.8em}} % -1.2em}}
\newcommand{\figskip}{\vspace{0pt}} % -1.3em}}
%\newcommand{\captionskip}[1]{\vspace{-0.75em}\caption{#1}\vspace{-1em}}
\newcommand\captionskip[1]{\caption{#1}}
\newcommand{\up}[1]{\overbrace{\vphantom{X^X}#1}}
\newcommand{\down}[1]{\underbrace{\vphantom{X^X}#1}}
\newcommand{\mathgobble}{-0.25em}
\newcommand{\monadgobble}{-0.5em}

\startPage{1}


%% Bibliography style
\bibliographystyle{ACM-Reference-Format}
%% Citation style
%% Note: author/year citations are required for papers published as an
%% issue of PACMPL.
\citestyle{acmauthoryear}  %% For author/year citations
%\citestyle{acmnumeric}     %% For numeric citations
%\setcitestyle{nosort}      %% With 'acmnumeric', to disable automatic
                            %% sorting of references within a single citation;
                            %% e.g., \cite{Smith99,Carpenter05,Baker12}
                            %% rendered as [14,5,2] rather than [2,5,14].
%\setcitesyle{nocompress}   %% With 'acmnumeric', to disable automatic
                            %% compression of sequential references within a
                            %% single citation;
                            %% e.g., \cite{Baker12,Baker14,Baker16}
                            %% rendered as [2,3,4] rather than [2-4].

\begin{document}

\title{Abstracting Definitional Interpreters (Appendix)} % for Higher-Order Programming Languages}
\subtitle{Functional Pearl}

\author{David Darais}
\orcid{nnnn-nnnn-nnnn-nnnn}             %% \orcid is optional
\affiliation{
  %\position{Position1}
  %\department{Department1}              %% \department is recommended
  \institution{University of Maryland}            %% \institution is required
  %\streetaddress{Street1 Address1}
  \city{College Park}
  \state{Maryland}
  %\postcode{Post-Code1}
  %\country{Country1}
}
\email{darais@cs.umd.edu}          %% \email is recommended

\author{Nicholas Labich}
\orcid{nnnn-nnnn-nnnn-nnnn}             %% \orcid is optional
\affiliation{
  %\position{Position1}
  %\department{Department1}              %% \department is recommended
  \institution{University of Maryland}            %% \institution is required
  %\streetaddress{Street1 Address1}
  \city{College Park}
  \state{Maryland}
  %\postcode{Post-Code1}
  %\country{Country1}
}
\email{labichn@cs.umd.edu}          %% \email is recommended

\author{Phúc C. Nguyễn}
\orcid{nnnn-nnnn-nnnn-nnnn}             %% \orcid is optional
\affiliation{
  %\position{Position1}
  %\department{Department1}              %% \department is recommended
  \institution{University of Maryland}            %% \institution is required
  %\streetaddress{Street1 Address1}
  \city{College Park}
  \state{Maryland}
  %\postcode{Post-Code1}
  %\country{Country1}
}
\email{pcn@cs.umd.edu}          %% \email is recommended


\author{David {Van Horn}}
\orcid{nnnn-nnnn-nnnn-nnnn}             %% \orcid is optional
\affiliation{
  %\position{Position1}
  %\department{Department1}              %% \department is recommended
  \institution{University of Maryland}            %% \institution is required
  %\streetaddress{Street1 Address1}
  \city{College Park}
  \state{Maryland}
  %\postcode{Post-Code1}
  %\country{Country1}
}
\email{dvanhorn@cs.umd.edu}          %% \email is recommended

%% \authorinfo{Phúc C. Nguyễn}
%%            {University of Maryland}
%%            {pcn@cs.umd.edu}


\setcopyright{rightsretained}
\acmJournal{PACMPL}
\acmYear{2017}
\acmVolume{1}
\acmNumber{ICFP}
\acmArticle{12}
\acmMonth{9}
\acmDOI{10.1145/3110256}
\acmPrice{}

\keywords{interpreters, abstract interpreters}

\begin{CCSXML}
<ccs2012>
 <concept>
 <concept_id>10011007.10010940.10010992.10010998.10011000</concept_id>
 <concept_desc>Software and its engineering~Automated static analysis</concept_desc>
 <concept_significance>500</concept_significance>
 </concept>
 <concept>
 <concept_id>10011007.10011006.10011008.10011009.10011012</concept_id>
 <concept_desc>Software and its engineering~Functional languages</concept_desc>
 <concept_significance>500</concept_significance>
 </concept>
</ccs2012>
\end{CCSXML}


\ccsdesc[500]{Software and its engineering~Automated static analysis}
\ccsdesc[500]{Software and its engineering~Functional languages}


\begin{abstract}
This is the appendix to accompany \citet{local:adi}.
\end{abstract}

\maketitle

%\category{CR-number}{subcategory}{third-level}
%\terms term1, term2
%\keywords keyword1, keyword2

\appendix
\section{Formalism}\label{s:formalism}

In this section we formalize our approach to designing definitional abstract
interpreters. We begin with a ``ground truth'' big-step semantics and concludes
with the fixpoint iteration strategy described in Section~\ref{s:cache}, which
we prove sound and computable w.r.t. a synthesized abstract semantics. The
design is systematic, and applies to arbitrary developments which use big-step
operational semantics. We demonstrate the systematic process as applied to a
subset of the language described in Figure~\ref{f:syntax}, which we call
\lamif:
\begin{alignat*}{1}
e ∈ exp ⩴ n ∣ x ∣ λx.e ∣ e(e) ∣ ⟬if0⟭(e)❴e❵❴e❵ ∣ b(e,e) 
\end{alignat*}
\vspace{-2.0em}
\begin{alignat*}{6}
                n ∈&&\mathrel{} nums  \mathrel{\hphantom{≔}} &\mathrel{}             &\quad\quad ℓ ∈ &&\mathrel{}  addr ≔ &\mathrel{} var × time 
\\[\mathgobble] x ∈&&\mathrel{} vars  \mathrel{\hphantom{≔}} &\mathrel{}             &\quad\quad σ ∈ &&\mathrel{} store ≔ &\mathrel{} addr → val⸤⊥⸥ 
\\[\mathgobble] b ∈&&\mathrel{} binop                    ≔   &\mathrel{} ❴plus,…❵    &\quad\quad v ∈ &&\mathrel{}   val ⩴ &\mathrel{} n ∣ ⟨λx.e,ρ⟩ 
\\[\mathgobble] τ ∈&&\mathrel{}  time                    ≔   &\mathrel{} ℕ           &\quad\quad ρ ∈ &&\mathrel{}   env ≔ &\mathrel{} var → addr⸤⊥⸥ 
\end{alignat*}

\paragraph{Concrete Semantics}

\begin{figure*} %{{{
\begin{flushright}[\emph{Concrete Evaluation}]\quad\fbox{«ρ,τ⊢e,σ⇓v,σ′»}\end{flushright}
\vspace{-0.75em}
\begin{mathpar}
  \inferrule*[left=(Lit)]{ }{ρ , τ ⊢ n , σ ⇓ n , σ}

  \inferrule*[left=(Var)]{ }{ρ , τ ⊢ x , σ ⇓ σ(ρ(x)) , σ}

  \inferrule*[left=(Lam)]{ }{ρ , τ ⊢ λx.e , σ ⇓ ⟨λx.e,ρ⟩ , σ}\vspace{-0.75em}

  \inferrule*[left=(Bin)]{
  ρ , τ ⊢ e₁ , σ  ⇓ v₁ , σ₁ \\
  ρ , τ ⊢ e₂ , σ₁ ⇓ v₂ , σ₂}
  {ρ , τ ⊢ b(e₁,e₂) , σ ⇓ ⟦b⟧(v₁,v₂) , σ₂}\vspace{-0.75em}

  \inferrule*[left=(App),right={\begin{minipage}{2em}\ssmall
    \begin{alignat*}{1}
    \begin{alignedat}{2} 
⟨λx.e′,ρ′⟩ = &\mathrel{} v₁ \\[-0.5em]
        ℓ  = &\mathrel{} ⟨x,τ′⟩ \\[-0.5em]
        τ′ \mathrel{\hphantom{=}} &\mathrel{} ⦑\emph{fresh}⦒
      \end{alignedat}
    \end{alignat*}
  \end{minipage}}]{
  ρ       , τ  ⊢ e₁    , σ        ⇓ v₁ , σ₁ \\
  ρ       , τ  ⊢ e₂    , σ₁       ⇓ v₂ , σ₂ \\
  ρ′[x↦ℓ] , τ′ ⊢ e′    , σ₂[ℓ↦v₂] ⇓ v′ , σ₃}
  {ρ      , τ ⊢ e₁(e₂) , σ        ⇓ v′ , σ₃}\vspace{-0.75em}

  \inferrule*[left=(IfT),right={\ssmall «n=0»}]{
  ρ , τ ⊢ e₁ , σ ⇓ n , σ₁ \\
  ρ , τ ⊢ e₂ , σ₁ ⇓ v , σ₂}
  {ρ , τ ⊢ ⟬if0⟭(e₁)❴e₂❵❴e₃❵ , σ ⇓ v , σ₂}

  \inferrule*[left=(IfF),right={\ssmall «n≠0»}]{
    ρ , τ ⊢ e₁ , σ  ⇓ n , σ₁ \\
    ρ , τ ⊢ e₃ , σ₁ ⇓ v , σ₂}
  {ρ , τ ⊢ ⟬if0⟭(e₁)❴e₂❵❴e₃❵ , σ ⇓ v , σ₂}
\end{mathpar}
\begin{flushright}[\emph{Concrete Reachability}]\quad\fbox{«ρ,τ⊢e,σ⇑ς»}\end{flushright}
\vspace{-0.25em}
\begin{mathpar}
  \inferrule*[left=(Refl)]{ }{ρ,τ⊢e,σ⇑⟨e,ρ,σ,τ⟩}

  \inferrule*[left=(RBin1)]
  {ρ,τ⊢e₁,σ⇑ς}
  {ρ,τ⊢b(e₁,e₂),σ⇑ς}

  \inferrule*[left=(RBin2)]
  {  ρ,τ⊢e₁,σ⇓v₁,σ₁
  \\ ρ,τ⊢e₂,σ₁⇑ς}
  {ρ,τ⊢b(e₁,e₂),σ⇑ς}\vspace{-0.75em}

  \inferrule*[left=(RApp1)]
   {ρ,τ⊢e₁,σ⇑ς}
   {ρ,τ⊢e₁(e₂),σ⇑ς}

   \inferrule*[left=(RApp2),right={\ssmall «⟨λx.e′,ρ′⟩=v₁»}]
   {  ρ,τ⊢e₁,σ⇓v₁,σ₁
   \\ ρ,τ⊢e₂,σ₁⇑ς}
   {ρ,τ⊢e₁(e₂),σ⇑ς}\vspace{-0.75em}

   \inferrule*[left=(RApp3),right={\begin{minipage}{2em}\ssmall
       \begin{alignat*}{1}
       \begin{alignedat}{2} 
    ⟨λx.e′,ρ′⟩ = &\mathrel{} v₁ \\[-0.5em]
           ℓ  = &\mathrel{} ⟨x,τ′⟩ \\[-0.5em]
           τ′ \mathrel{\hphantom{=}} &\mathrel{} ⦑\emph{fresh}⦒
         \end{alignedat}
       \end{alignat*}
     \end{minipage}}]
  {  ρ,τ⊢e₁,σ ⇓v₁,σ₁
  \\ ρ,τ⊢e₂,σ₁⇓v₂,σ₂
  \\ ρ′[x↦ℓ],τ′⊢e′,σ₂[ℓ↦v₂]⇑ς}
  {ρ,τ⊢e₁(e₂),σ⇑ς}\vspace{-0.75em}

  \inferrule*[left=(RIf1)]
  {ρ,τ⊢e₁,σ⇑ς}
  {ρ,τ⊢⟬if0⟭(e₁)❴e₂❵❴e₃❵,σ⇑ς}

  \inferrule*[left=(RIfT),right={\ssmall «n=0»}]
  {  ρ,τ⊢e₁,σ⇓n,σ₁
  \\ ρ,τ⊢e₂,σ₁⇑ς}
  {ρ,τ⊢⟬if0⟭(e₁)❴e₂❵❴e₃❵,σ⇑ς}\vspace{-0.75em}

  \inferrule*[left=(RIfF),right={\ssmall «n≠0»}]
  {  ρ,τ⊢e₁,σ⇓n,σ₁
  \\ ρ,τ⊢e₃,σ₁⇑ς}
  {ρ,τ⊢⟬if0⟭(e₁)❴e₂❵❴e₃❵,σ⇑ς}
\end{mathpar}
\caption{\lamif{} Big-step Concrete Evaluation and Reachability Semantics}
\label{f:lamif-concrete}
\end{figure*} %}}}

We begin with the concrete semantics of \lamif as a big-step evaluation
relation «ρ,τ⊢e,σ⇓v,σ′», shown in Figure~\ref{f:lamif-concrete}. The definition
is mostly standard: «ρ» and «σ» are the environment and store, «e» is the
initial expression, and «v» is the resulting value. The argument «τ» represents
``time,'' which when abstracted supports modeling execution contexts like
call-site sensitivity. Concretely time is modelled as a natural number, and all
that is required is that ``fresh'' numbers are available for allocating values
in the store.

\begin{figure*} %{-{
\begin{flushright}[\emph{Collecting Evaluation}]\quad\fbox{«ρ,τ⊢e,∿{σ}⇓∿{v},∿{σ}′»}\end{flushright}
\vspace{-0.75em}
\begin{mathpar}
  \inferrule*[left=(OLit)]{ }{ρ,τ⊢n,∿{σ}⇓❴n❵,∿{σ}}

  \inferrule*[left=(OVar)]{ }{ρ,τ⊢x,∿{σ}⇓∿{σ}(ρ(x)),∿{σ}}

  \inferrule*[left=(OLam)]{ }{ρ,τ⊢λx.e,∿{σ}⇓❴⟨λx.e,ρ⟩❵,∿{σ}}\vspace{-0.75em}

    \inferrule*[left=(OBin)]
  {  ρ,τ⊢e₁,∿{σ}⇓∿{v}₁,∿{σ}₁
  \\ ρ,τ⊢e₂,∿{σ}₁⇓∿{v}₂,∿{σ}₂}
  {  ρ,τ⊢b(e₁,e₂),∿{σ}⇓∿{⟦b⟧}(∿{v}₁,∿{v}₂),∿{σ}₂}\vspace{-0.75em}

  \inferrule*[left=(OApp),right={\begin{minipage}{2em}\ssmall
    \begin{alignat*}{1}
    \begin{alignedat}{2} 
              ⟨λx.e′,ρ′⟩ ∈ &\mathrel{} ∿{v}₁ \\[-0.5em]
                      ℓ  = &\mathrel{} ⟨x,τ′⟩ \\[-0.5em]
                      τ′ \mathrel{\hphantom{=}} &\mathrel{} ⦑\emph{fresh}⦒
      \end{alignedat}
    \end{alignat*}
  \end{minipage}}]
  {  ρ      ,τ ⊢e₁    ,∿{σ} ⇓∿{v}₁,∿{σ}₁
  \\ ρ      ,τ ⊢e₂    ,∿{σ}₁⇓∿{v}₂,∿{σ}₂
  \\ ρ′[x↦ℓ],τ′⊢e′    ,∿{σ}₂[ℓ↦∿{v}₂]⇓∿{v}′,∿{σ}₃}
  {ρ      ,τ ⊢e₁(e₂),∿{σ} ⇓∿{v}′,∿{σ}₃}\vspace{-0.75em}

  \inferrule*[left=(OIfT),right={\ssmall «0∈∿{v}₁»}]
  {  ρ,τ⊢e₁,∿{σ}⇓∿{v}₁,∿{σ}₁
  \\ ρ,τ⊢e₂,∿{σ}₁⇓∿{v},∿{σ}₂}
  {  ρ,τ⊢⟬if0⟭(e₁)❴e₂❵❴e₃❵,∿{σ}⇓∿{v},∿{σ}₂}

  \inferrule*[left=(OIfF),right={\begin{minipage}{2em}\ssmall
      \begin{alignat*}{2}
        n ∈ &\mathrel{} ∿{v}₁ \\[-0.5em]
        n ≠ &\mathrel{} 0
      \end{alignat*}
    \end{minipage}}]
    {  ρ,τ⊢e₁,∿{σ}⇓∿{v}₁,∿{σ}₁
    \\ ρ,τ⊢e₃,∿{σ}₁⇓∿{v},∿{σ}₂}
    {  ρ,τ⊢⟬if0⟭(e₁)❴e₂❵❴e₃❵,∿{σ}⇓∿{v},∿{σ}₂}
\end{mathpar}
\begin{flushright}[\emph{Collecting Reachability}]\quad\fbox{«ρ,τ⊢e,∿{σ}⇑∿{ς}»}\end{flushright}
\vspace{-0.25em}
\begin{mathpar}
  \inferrule*[left=(ORefl)]{ }{ρ,τ⊢e,∿{σ}⇑⟨e,ρ,∿{σ},τ⟩}

  \inferrule*[left=(ORBin1)]
  {ρ,τ⊢e₁,∿{σ}⇑ς}
  {ρ,τ⊢b(e₁,e₂),∿{σ}⇑∿{ς}}

  \inferrule*[left=(ORBin2)]
  {  ρ,τ⊢e₁,∿{σ}⇓∿{v}₁,∿{σ}₁
  \\ ρ,τ⊢e₂,∿{σ}₁⇑∿{ς}}
  {ρ,τ⊢b(e₁,e₂),∿{σ}⇑∿{ς}}\vspace{-0.75em}

  \inferrule*[left=(ORApp1)]
   {ρ,τ⊢e₁,∿{σ}⇑∿{ς}}
   {ρ,τ⊢e₁(e₂),∿{σ}⇑∿{ς}}

   \inferrule*[left=(ORApp2),right={\ssmall «⟨λx.e′,ρ′⟩∈∿{v}₁»}]
   {  ρ,τ⊢e₁,∿{σ}⇓∿{v}₁,∿{σ}₁
   \\ ρ,τ⊢e₂,∿{σ}₁⇑∿{ς}}
   {ρ,τ⊢e₁(e₂),∿{σ}⇑∿{ς}}\vspace{-0.75em}

   \inferrule*[left=(ORApp3),right={\begin{minipage}{2em}\ssmall
       \begin{alignat*}{1}
       \begin{alignedat}{2} 
  ⟨λx.e′,ρ′⟩ ∈ &\mathrel{} ∿{v}₁ \\[-0.5em]
           ℓ  = &\mathrel{} ⟨x,τ′⟩ \\[-0.5em]
           τ′ \mathrel{\hphantom{=}} &\mathrel{} ⦑\emph{fresh}⦒
         \end{alignedat}
       \end{alignat*}
     \end{minipage}}]
  {  ρ,τ⊢e₁,∿{σ} ⇓∿{v}₁,∿{σ}₁
  \\ ρ,τ⊢e₂,∿{σ}₁⇓∿{v}₂,∿{σ}₂
  \\ ρ′[x↦ℓ],τ′⊢e′,∿{σ}₂[ℓ↦∿{v}₂]⇑∿{ς}}
  {ρ,τ⊢e₁(e₂),∿{σ}⇑∿{ς}}\vspace{-0.75em}

  \inferrule*[left=(ORIf1)]
  {ρ,τ⊢e₁,∿{σ}⇑∿{ς}}
  {ρ,τ⊢⟬if0⟭(e₁)❴e₂❵❴e₃❵,∿{σ}⇑∿{ς}}

  \inferrule*[left=(ORIfT),right={\ssmall «0∈∿{v}₁»}]
  {  ρ,τ⊢e₁,∿{σ}⇓∿{v}₁,∿{σ}₁
  \\ ρ,τ⊢e₂,∿{σ}₁⇑∿{ς}}
  {ρ,τ⊢⟬if0⟭(e₁)❴e₂❵❴e₃❵,∿{σ}⇑∿{ς}}\vspace{-0.75em}

  \inferrule*[left=(ORIfF),right={\begin{minipage}{2em}\ssmall 
    \begin{alignat*}{2}
      n ∈ &\mathrel{} ∿{v}₁ \\[-0.5em]
      n ≠ &\mathrel{} 0
    \end{alignat*}
    \end{minipage}}]
    {  ρ,τ⊢e₁,∿{σ}⇓∿{v}₁,∿{σ}₁
    \\ ρ,τ⊢e₃,∿{σ}₁⇑∿{ς}}
    {ρ,τ⊢⟬if0⟭(e₁)❴e₂❵❴e₃❵,∿{σ}⇑∿{ς}}
\end{mathpar}
\caption{Big-step Collecting Evaluation and Reachability Semantics}
\label{f:lamif-collecting}
\end{figure*} %}-}

\paragraph{Reachability}

The primary limitation of using big-step semantics as a starting point for
abstraction is that intermediate computations are not represented in the model
for evaluation. For example, consider the program that applies the identity
function to an expression that loops, which we notate «Ω»:
\[ (λx.x)(Ω) \]
A big-step evaluation relation can only describe results of terminating
computations, and because this program never terminates, such a relation says
nothing about the behavior of the program. A good static analyzer will explore
the behavior of «Ω» to (possibly) discover that it loops, or more importantly,
to provide analysis results (like data-flow or side-effects) for intermediate
computation states.

The need to analyze intermediate states is the primary reason that big-step
semantics are overlooked as a starting point for abstract interpretation. To
remedy the situation, while remaining in a big-step setting, we introduce a
big-step \emph{reachability} relation, notated «ρ,τ⊢e,σ⇑ς» and also shown in
Figure~\ref{f:lamif-concrete}. Configurations «ς» are tuples «⟨e,ρ,σ,τ⟩»,
and are reachable when evaluation passes through the configuration at any point
on its way to a final value, or during an infinite loop.

The complete big-step semantics of an expression («e») under environment («ρ»),
store («σ») and time («τ»), which we notate «⟦e⟧⸢bs⸣(ρ,σ,τ)», is then the set
of all \emph{reachable evaluations}:
\begin{alignat*}{2}
                ⟦e⟧⸢bs⸣(ρ,σ,τ) ≔ ❴⟨v,σ″⟩ ∣ &\mathrel{} ρ,σ,τ⇑⟨e′,ρ′,σ′,τ′⟩ 
\\[\mathgobble]                          ∧ &\mathrel{} ρ′,τ′⊢e′,σ′⇓v,σ″❵
\end{alignat*}
We construct a formal bridge between the big-step and small-step worlds through
the complete big-step semantics («⟦e⟧⸢bs⸣») and a complete small-step semantics
«↝⸢*⸣», which is traditionally used as the starting point of abstraction for
program analysis:
\begin{alignat*}{1}
                & ⟦e⟧⸢ss⸣(ρ,σ,τ) ≔ 
\\[\mathgobble] & ␣\begin{alignedat}{2} ❴⟨v,σ″⟩ ∣ &\mathrel{} ∀κ. ⟨e,ρ,σ,τ,κ⟩ ↝⸢*⸣ ⟨e′,ρ′,σ′,τ′,κ′\!⧺\!κ⟩ 
                   \\[\mathgobble]              ∧ &\mathrel{} ⟨e′,ρ′,σ′,τ′,κ′⧺κ⟩ ↝⸢*⸣ ⟨v,ρ″,σ″,τ″,κ′\!⧺\!κ⟩❵
                   \end{alignedat}
\end{alignat*}
We connect the complete big-step and small-step semantics through the following
theorem:
\begin{theorem}[Complete Big-step/Small-step Equivalence]
  \[ ⟦e⟧⸢bs⸣(ρ,σ,τ) = ⟦e⟧⸢ss⸣(ρ,σ,τ) \]
\end{theorem}
The proof is by induction on the big-step derivation for «⊆», and on the
transitive small-step derivation for «⊇».

\paragraph{Collecting Semantics}

Before abstracting the semantics—in pursuit of a sound static analysis
algorithm—we pass through a big-step collecting evaluation and reachability
semantics, notated «ρ,τ⊢e,∿{σ}⇓∿{v},∿{σ}» and «ρ,τ⊢e,∿{σ}⇑∿{ς}» and shown in
Figure~\ref{f:lamif-collecting}, where «∿{v}», «∿{σ}» and «∿{ς}» range over
collecting state spaces:
\begin{alignat*}{3}
                ∿{v} ∈ &&\mathrel{} ∿{val}      &\mathrel{} ≔ ℘(val) 
\\[\mathgobble] ∿{σ} ∈ &&\mathrel{} ∿{store}    &\mathrel{} ≔ addr ↦ ∿{val} 
\\[\mathgobble] ∿{ς} ∈ &&\mathrel{} ∿{config}   &\mathrel{} ≔ exp × env × ∿{store} × time
\end{alignat*}
and the denotation for binary operators («⟦b⟧») is lifted to a collecting
denotation operator «∿{⟦b⟧}»:
\[ ∿{⟦b⟧}(∿{v}₁,∿{v}₂) ≔ ❴⟦b⟧(v₁,v₂) ∣ v₁ ∈ ∿{v}₁ ∧ v₂ ∈ ∿{v}₂❵ \]

The big-step collecting and reachability relations are structurally similar to
the concrete semantics. The primary differences are the use of set containment
(«∈») in place of equality («=») when branching on application and conditional
expressions.

The big-step collecting reachability semantics is a sound approximation of the
big-step concrete reachability semantics:
\begin{theorem}[Collecting Reachability Semantics Soundness]
\begin{alignat*}{2}
                & ⦑If⦒    &&\quad ρ,τ⊢e,σ⇑⟨e′,ρ′,σ′,τ′⟩ \quad ⦑and⦒ \quad ρ′,τ′⊢e′,σ′⇓v,σ″
\\[\mathgobble] & ⦑where⦒ &&\quad η(σ) ⊑ ∿{σ}
\\[\mathgobble] & ⦑then⦒  &&\quad ρ,τ⊢e,∿{σ}⇑⟨e′,ρ′,∿{σ}′,τ′⟩ \quad ⦑and⦒ \quad ρ′,τ′⊢e,∿{σ}′⇓∿{v},∿{σ}″ 
\\[\mathgobble] & ⦑where⦒ &&\quad η(σ′) ⊑ ∿{σ}′ \quad ⦑and⦒ \quad v ∈ ∿{v} \quad ⦑and⦒ \quad η(σ″) ⊑ ∿{σ}″
\end{alignat*}
\end{theorem}
The proof is by induction on the concrete big-step derivation. The extraction
function «η» is defined separately for stores («σ») and configurations («ς»):
\begin{alignat*}{1}
   & η(σ)(ℓ) ≔ ❴σ(ℓ)❵ \quad\quad η(⟨e,ρ,σ,τ⟩) ≔ ⟨e,ρ,η(σ),τ⟩
\end{alignat*}
and the partial ordering on stores and configurations is pointwise:
\begin{alignat*}{1}
  & ∿{σ}₁ ⊑ ∿{σ}₂ \quad \mathrel{⦑\emph{iff}⦒} \quad ∀ℓ.\mathrel{} ∿{σ}₁(ℓ) ⊆ ∿{σ}₂(ℓ)
  \\[\mathgobble] & ⟨e₁,ρ₁,∿{σ}₁,τ₁⟩ ⊑ ⟨e₂,ρ₂,∿{σ}₂,τ₂⟩ \quad \mathrel{⦑\emph{iff}⦒} 
  \\[\mathgobble] & \hspace{2em} e₁ = e₂ ∧ ρ₁ = ρ₂ ∧ ∿{σ}₁ ⊑ ∿{σ}₂ ∧ τ₁ = τ₂
\end{alignat*}

\paragraph{Finite Abstraction}

\begin{figure*} %{-{
\begin{flushright}[\emph{Abstract Evaluation}]\quad\fbox{«♯{ρ},♯{τ}⊢e,♯{σ}⇓♯{v},♯{σ}′»}\end{flushright}
\vspace{-0.75em}
\begin{mathpar}
  \inferrule*[left=(ALit)]{ }{♯{ρ},♯{τ}⊢n,♯{σ}⇓♯{η}(n),♯{σ}}

  \inferrule*[left=(AVar)]{ }{♯{ρ},♯{τ}⊢x,♯{σ}⇓♯{σ}(♯{ρ}(x)),♯{σ}}

  \inferrule*[left=(ALam)]{ }{♯{ρ},♯{τ}⊢λx.e,♯{σ}⇓♯{η}(⟨λx.e,♯{ρ}⟩),♯{σ}}\vspace{-0.75em}

    \inferrule*[left=(ABin)]
    {  ♯{ρ},♯{τ}⊢e₁,♯{σ}⇓♯{v}₁,♯{σ}₁
    \\ ♯{ρ},♯{τ}⊢e₂,♯{σ}₁⇓♯{v}₂,♯{σ}₂}
    {  ♯{ρ},♯{τ}⊢b(e₁,e₂),♯{σ}⇓♯{⟦b⟧}(♯{v}₁,♯{v}₂),♯{σ}₂}\vspace{-0.75em}

  \inferrule*[left=(AApp),right={\begin{minipage}{2em}\ssmall
    \begin{alignat*}{1}
    \begin{alignedat}{2} 
           ⟨λx.e′,♯{ρ}′⟩ ∈ &\mathrel{} ⌊γ⌋⸤clo⸥(♯{v}₁) \\[-0.5em]
                   ♯{ς}  = &\mathrel{} ⟨e₁(e₂),♯{ρ},♯{σ},♯{τ}⟩ \\[-0.5em]
                   ♯{ℓ}  = &\mathrel{} ⟨x,♯{τ}′⟩ \\[-0.5em]
                   ♯{τ}′ = &\mathrel{} ♯{next}(♯{τ},♯{ς})
      \end{alignedat}
    \end{alignat*}
  \end{minipage}}]
  {  ♯{ρ}      ,♯{τ} ⊢e₁    ,♯{σ} ⇓♯{v}₁,♯{σ}₁
  \\ ♯{ρ}      ,♯{τ} ⊢e₂    ,♯{σ}₁⇓♯{v}₂,♯{σ}₂
  \\ ♯{ρ}′[x↦♯{ℓ}],♯{τ}′⊢e′    ,♯{σ}₂⊔[♯{ℓ}↦♯{v}₂]⇓♯{v}′,♯{σ}₃}
  {♯{ρ}      ,♯{τ} ⊢e₁(e₂),♯{σ} ⇓♯{v}′,♯{σ}₃}\vspace{-0.75em}

  \inferrule*[left=(AIfT),right={\ssmall «0∈⌊γ⌋⸤0⸥(♯{v}₁)»}]
  {  ♯{ρ},♯{τ}⊢e₁,♯{σ}⇓♯{v}₁,♯{σ}₁
  \\ ♯{ρ},♯{τ}⊢e₂,♯{σ}₁⇓♯{v},♯{σ}₂}
  {  ♯{ρ},♯{τ}⊢⟬if0⟭(e₁)❴e₂❵❴e₃❵,♯{σ}⇓♯{v},♯{σ}₂}
  \;
  \inferrule*[left=(AIfF),right={\ssmall «¬0∈⌊γ⌋⸤¬0⸥(♯{v}₁)»}]
  {  ♯{ρ},♯{τ}⊢e₁,♯{σ}⇓♯{v}₁,♯{σ}₁
  \\ ♯{ρ},♯{τ}⊢e₃,♯{σ}₁⇓♯{v},♯{σ}₂}
  {  ♯{ρ},♯{τ}⊢⟬if0⟭(e₁)❴e₂❵❴e₃❵,♯{σ}⇓♯{v},♯{σ}₂}
\end{mathpar}
\begin{flushright}[\emph{Abstract Reachability}]\quad\fbox{«♯{ρ},♯{τ}⊢e,♯{σ}⇑♯{ς}»}\end{flushright}
\vspace{-0.25em}
\begin{mathpar}
  \inferrule*[left=(ARefl)]{ }{♯{ρ},♯{τ}⊢e,♯{σ}⇑⟨e,♯{ρ},♯{σ},♯{τ}⟩}

  \inferrule*[left=(ARBin1)]
  {♯{ρ},♯{τ}⊢e₁,♯{σ}⇑ς}
  {♯{ρ},♯{τ}⊢b(e₁,e₂),♯{σ}⇑♯{ς}}

  \inferrule*[left=(ARBin2)]
  {  ♯{ρ},♯{τ}⊢e₁,♯{σ}⇓♯{v}₁,♯{σ}₁
  \\ ♯{ρ},♯{τ}⊢e₂,♯{σ}₁⇑♯{ς}}
  {♯{ρ},♯{τ}⊢b(e₁,e₂),♯{σ}⇑♯{ς}}\vspace{-0.75em}

  \inferrule*[left=(ARApp1)]
  {♯{ρ},♯{τ}⊢e₁,♯{σ}⇑♯{ς}}
  {♯{ρ},♯{τ}⊢e₁(e₂),♯{σ}⇑♯{ς}}

  \inferrule*[left=(ARApp2),right={\ssmall «⟨λx.e′,♯{ρ}′⟩∈⌊γ⌋⸤clo⸥(♯{v}₁)»}]
  {  ♯{ρ},♯{τ}⊢e₁,♯{σ}⇓♯{v}₁,♯{σ}₁
  \\ ♯{ρ},♯{τ}⊢e₂,♯{σ}₁⇑♯{ς}}
  {♯{ρ},♯{τ}⊢e₁(e₂),♯{σ}⇑♯{ς}}\vspace{-0.75em}

   \inferrule*[left=(ARApp3),right={\begin{minipage}{2em}\ssmall
       \begin{alignat*}{1}
       \begin{alignedat}{2} 
         ⟨λx.e′,♯{ρ}′⟩ ∈ &\mathrel{} ⌊γ⌋⸤clo⸥(♯{v}₁) \\[-0.5em]
            ♯{ς} = &\mathrel{} ⟨e₁(e₂),♯{ρ},♯{σ},♯{τ}⟩ \\[-0.5em]
           ♯{ℓ}  = &\mathrel{} ⟨x,♯{τ}′⟩ \\[-0.5em]
           ♯{τ}′ = &\mathrel{} ♯{next}(♯{τ},♯{ς})
         \end{alignedat}
       \end{alignat*}
     \end{minipage}}]
     {  ♯{ρ},♯{τ}⊢e₁,♯{σ} ⇓♯{v}₁,♯{σ}₁
       \\ ♯{ρ},♯{τ}⊢e₂,♯{σ}₁⇓♯{v}₂,♯{σ}₂
     \\ ♯{ρ}′[x↦♯{ℓ}],♯{τ}′⊢e′,♯{σ}₂⊔[♯{ℓ}↦♯{v}₂]⇑♯{ς}}
     {♯{ρ},♯{τ}⊢e₁(e₂),♯{σ}⇑♯{ς}}\vspace{-0.75em}

  \inferrule*[left=(ARIf1)]
  {♯{ρ},♯{τ}⊢e₁,♯{σ}⇑♯{ς}}
  {♯{ρ},♯{τ}⊢⟬if0⟭(e₁)❴e₂❵❴e₃❵,♯{σ}⇑♯{ς}}

  \inferrule*[left=(ARIfT),right={\ssmall «0∈⌊γ⌋⸤0⸥(♯{v}₁)»}]
  {  ♯{ρ},♯{τ}⊢e₁,♯{σ}⇓♯{v}₁,♯{σ}₁
  \\ ♯{ρ},♯{τ}⊢e₂,♯{σ}₁⇑♯{ς}}
  {♯{ρ},♯{τ}⊢⟬if0⟭(e₁)❴e₂❵❴e₃❵,♯{σ}⇑♯{ς}}\vspace{-0.75em}

  \inferrule*[left=(ARIfF),right={\ssmall «¬0∈⌊γ⌋⸤¬0⸥(♯{v}₁)»}]
    {  ♯{ρ},♯{τ}⊢e₁,♯{σ}⇓♯{v}₁,♯{σ}₁
    \\ ♯{ρ},♯{τ}⊢e₃,♯{σ}₁⇑♯{ς}}
    {♯{ρ},♯{τ}⊢⟬if0⟭(e₁)❴e₂❵❴e₃❵,♯{σ}⇑♯{ς}}
\end{mathpar}
\caption{Big-step Abstract Evaluation and Reachability Semantics}
\label{f:lamif-abstract}
\end{figure*} %}-}

The next step towards a computable static analysis is an abstract semantics
with a finite state space that approximates the big-step collecting semantics,
notated «♯{ρ},♯{τ}⊢e,♯{σ}⇓♯{v},♯{σ}» and «♯{ρ},♯{τ}⊢e,♯{σ}⇑♯{ς}» and shown in
Figure~\ref{f:lamif-abstract}, where «♯{ρ}», «♯{τ}», «♯{v}», «♯{σ}» and «♯{ς}»
are finite abstractions of their collecting counterparts:
\begin{alignat*}{3}
                ♯{ρ} ∈ &&\mathrel{} ♯{env}    &\mathrel{} ≔ var ↦ ♯{addr}⸤⊥⸥ 
\\[\mathgobble] ♯{ℓ} ∈ &&\mathrel{} ♯{addr}   &\mathrel{} ≔ var × ♯{time} 
\\[\mathgobble] ♯{τ} ∈ &&\mathrel{} ♯{time}   &\mathrel{} ≔ … 
\\[\mathgobble] ♯{v} ∈ &&\mathrel{} ♯{val}    &\mathrel{} ≔ … 
\\[\mathgobble] ♯{σ} ∈ &&\mathrel{} ♯{store}  &\mathrel{} ≔ ♯{addr} ↦ ♯{val} 
\\[\mathgobble] ♯{ς} ∈ &&\mathrel{} ♯{config} &\mathrel{} ≔ exp × ♯{env} × ♯{store} × ♯{time}
\end{alignat*}
The primary structural difference from the collecting semantics is the use of
join when updating the store («♯{σ}⊔[♯{ℓ}↦♯{v}]») rather than strict
replacement («∿{σ}[ℓ↦∿{v}]»). This is to preserve soundness in the presence of
address reuse, which occurs from the finite size of the address space.

The abstract denotation («♯{⟦b⟧}») is any over-approximation of the collecting
denotation («∿{⟦b⟧}») w.r.t. a Galois connection «∿{val}⇄{α}{γ}♯{val}»:
\[ ♯{⟦b⟧}(♯{v}₁,♯{v}₂) ⊒ α(∿{⟦b⟧}(γ(♯{v}₁),γ(♯{v}₂))) \]
Concretization functions «⌊γ⌋⸤clo⸥», «⌊γ⌋⸤0⸥» and «⌊γ⌋⸤¬0⸥» are computable
finite subsets of the full concretization function «γ» s.t.:
\begin{alignat*}{2}
                ⌊γ⌋⸤clo⸥(♯{v}) ≔ &\mathrel{} ❴⟨λx.e,♯{ρ}⟩ ∣ ⟨λx.e,♯{ρ}⟩ ∈ γ(♯{v})❵ 
\\[\mathgobble] ⌊γ⌋⸤0⸥(♯{v})   ≔ &\mathrel{} ❴0 ∣ 0 ∈ γ(♯{v})❵ 
\\[\mathgobble] ⌊γ⌋⸤¬0⸥(♯{v})  ≔ &\mathrel{} ❴¬0 ∣ n ∈ γ(♯{v}) ∧ n≠0❵
\end{alignat*}
Abstract sets «♯{time}» and «♯{val}» are left as parameters to the analysis
along with their operations «♯{next}», «♯{⟦b⟧}», «⌊γ⌋⸤clo⸥», «⌊γ⌋⸤0⸥»,
«⌊γ⌋⸤¬0⸥» and «⊔⸢♯{val}⸣».

The abstract semantics is a sound approximation of the collecting semantics,
which we establish through the theorem:
\begin{theorem}[Abstract Reachability Semantics Soundness]
\begin{alignat*}{2}
               & ⦑If⦒    &&\quad ρ,τ⊢e,∿{σ}⇑⟨e′,ρ′,∿{σ}′,τ′⟩ \quad ⦑and⦒ \quad ρ′,τ′⊢e′,∿{σ}′⇓∿{v},∿{σ}″
\\[\mathgobble]& ⦑where⦒ &&\quad η(ρ) ⊑ ♯{ρ} \quad ⦑and⦒ \quad η(τ) ⊑ ♯{τ} \quad ⦑and⦒ \quad η(∿{σ}) ⊑ ♯{σ}
\\[\mathgobble]& ⦑then⦒  &&\quad ♯{ρ},♯{τ}⊢e,♯{σ}⇑⟨e′,♯{ρ}′,♯{σ}′,♯{τ}′⟩ \quad ⦑and⦒ \quad ♯{ρ}′,♯{τ}′⊢e,♯{σ}′⇓♯{v},♯{σ}″ 
\\[\mathgobble]& ⦑where⦒ &&\quad η(ρ′) ⊑ ♯{ρ}′⦑,⦒\ η(τ′) ⊑ ♯{τ}′⦑,⦒\ η(∿{σ}′) ⊑ ♯{σ}′⦑,⦒\ v ∈ ∿{v}⦑,⦒\ η(σ″) ⊑ ∿{σ}″
\end{alignat*}
\end{theorem}
The proof is by induction on the big-step derivation. The extraction function
«η» is defined separately for environments («ρ»), time («τ»), collecting stores
(«∿{σ}»), values («∿{v}») and configurations («∿{ς}»). «η(τ)» and «η(∿{v})» are
given with parameters «♯{time}» and «♯{val}». «η(ρ)», «η(∿{σ})» and «η(∿{ς})»
are defined pointwise:
\begin{alignat*}{1}
  & η(ρ)(x) ≔ η(ρ(x)) \quad\quad η(∿{σ})(♯{ℓ}) ≔ ⨆⸤ℓ ∈ γ(♯{ℓ})⸥η(∿{σ}(ℓ)) 
\\[\mathgobble] & η(⟨e,ρ,τ,∿{σ}⟩) ≔ ⟨e,η(ρ),η(τ),η(∿{σ})⟩
\end{alignat*}

\paragraph{Computing the Analysis}

An analysis for the program «e₀» w.r.t. the abstract semantics is some cache
«\$ ∈ ♯{config} ↦ ℘(♯{val} × ♯{store})» that maps all configurations reachable
from the initial configuration «⟨e₀,♯{ρ}₀,♯{σ}₀,♯{τ}₀⟩» to their final values
and stores «♯{v},♯{σ}», which we notate «\$ ⊨ e₀»:
\begin{alignat*}{2}
  \$ ⊨ e₀ \quad\quad \mathrel{⦑\emph{iff}⦒} \quad\quad & 
    \begin{alignedat}{2}
                   & ⦑\emph{If}⦒     &&\quad ♯{ρ}₀,♯{τ}₀⊢e₀,♯{σ₀}⇑⟨e,♯{ρ},♯{σ},♯{τ}⟩ 
    \\[\mathgobble]& ⦑\emph{and}⦒    &&\quad ♯{ρ},♯{τ}⊢e,♯{σ}⇓♯{v},♯{σ}′  
    \\[\mathgobble]& ⦑\emph{then}⦒   &&\quad ⟨♯{v},♯{σ}′⟩ ∈ \$(⟨e,♯{ρ},♯{σ},♯{τ}⟩)
      \end{alignedat}
\end{alignat*}
The \emph{best} cache «\$⁺» is then computed as the least fixed point of the
functional «ℱ»:
\begin{alignat*}{1}
  & ℱ ∈ (♯{config} ↦ ℘(♯{val}×♯{store})) → (♯{config} ↦ ℘(♯{val}×♯{store})) 
\\[\mathgobble] & ℱ ≔ λ\$.  
\\[\mathgobble] &  \hspace{1em} ⨆⸤⟨e,♯{ρ},♯{σ},♯{τ}⟩∈\$⸥ \begin{cases}
         ❴ ⟨e,♯{ρ},♯{σ},♯{τ}⟩ ↦ ❴⟨♯{v},♯{σ}′⟩❵ ∣ ♯{ρ},♯{τ}⊢e,♯{σ}⇓⸢\$⸣♯{v},♯{σ}′ ❵ 
      \\[\mathgobble] ❴ ♯{ς} ↦ ❴❵ ∣ ♯{ρ},♯{τ}⊢e,♯{σ}⇑⸢\$⸣♯{ς}❵
   \end{cases}
\end{alignat*}
which also includes the initial configuration:
\[ \$⁺ ≔ ⦑\emph{lfp}⦒ (λ\$.\mathrel{} ℱ(\$) ⊔ ❴⟨e₀,η(ρ₀),η(σ₀),η(τ₀)⟩ ↦ ❴❵❵) \]
The relations «♯{ρ},♯{τ}⊢e,♯{σ}⇓⸢\$⸣♯{v},♯{σ}′» and «♯{ρ},♯{τ}⊢e,♯{σ}⇑⸢\$⸣♯{ς}»
are modified versions of the original abstract semantics, but with recursive
judgements replaced by «⟨♯{v},♯{σ}′⟩ ∈ \$(e,♯{ρ},♯{σ},♯{τ})» and «♯{ς} ∈
\$(e,♯{ρ},♯{σ},♯{τ})» respectively. Therefore «ℱ» is not recursive; the
recursion in the relations is lifted to the outer fixpoint of the analysis.
Because the state space «♯{config} ↦ ℘(♯{val}×♯{store})» is finite and «ℱ» is
monotonic, «\$⁺» can be computed algorithmically in finite time by Kleene
fixed-point iteration. See Nielson et al~\cite{dvanhorn:Neilson:1999} for more
background and examples of static analyzers computed in this style, and from
which the current development was largely inspired.
\begin{theorem}[Algorithm Correctness]
  «\$⁺» is a valid analysis for «e₀», that is: «\$⁺ ⊨ e₀».
\end{theorem}
The proof is by induction on the assumed derivations
«♯{ρ}₀,♯{τ}₀⊢e₀,♯{σ}₀⇑⟨♯{e},♯{ρ},♯{σ},♯{τ}⟩» and «♯{ρ},♯{τ}⊢e,♯{σ}⇓♯{v},♯{σ}′»,
and utilizes the fact that «\$⁺» is a fixed point, that is: «ℱ(\$⁺) = \$⁺». Our
final theorem relates the analysis cache «\$⁺» back to the concrete semantics
of the initial program as a sound approximation:
\begin{theorem}[Algorithm Soundness]
\begin{alignat*}{2}
                &⦑If ⦒  &&\quad ρ₀,τ₀⊢e₀,σ₀⇑⟨e,ρ,σ,τ⟩ \quad ⦑and⦒ \quad ρ,τ⊢e,σ⇓v,σ′  
\\[\mathgobble] &⦑then⦒ &&\quad ⟨♯{v},♯{σ}′⟩ ∈ \$⁺(⟨e,♯{ρ},♯{σ},♯{τ}⟩)  
\\[\mathgobble] &⦑where⦒&&\quad η(ρ) ⊑ ♯{ρ}⦑,⦒\ η(τ) ⊑ ♯{τ}⦑,⦒\ η(σ) ⊑ ♯{σ}⦑,⦒\ η(v) ⊑ ♯{v}⦑,⦒\ η(σ′) ⊑ ♯{σ}′
\end{alignat*}
\end{theorem}
The proof follows by composing Theorems~1-4.

\paragraph{Computing with Definitional Interpreters}

The algorithm described in Section~\ref{s:cache} is a more efficient strategy
for computing «\$⁺» using an extensible open-recursive definitional
interpreter. This technique is general, and bridges the gap between the
big-step abstract semantics formalized in this section and the definitional
interpreters we wish to execute to obtain analyses.

An extensible open-recursive definitional interpreter for \lamif (the small
language formalized in this section) has domain:
\begin{alignat*}{1}
  & ℰ ∈ Σ → Σ \quad ⦑\emph{where}⦒ \quad Σ ≔ ♯{config} → ℘(♯{val}×♯{store})
\end{alignat*}
and is defined such that its denotational-fixpoint («Y(ℰ)») recovers concrete
interpretation when instantiated with the concrete state-space. For example,
the recursive case for binary operator expressions is defined:
\begin{alignat*}{1}
  & ℰ(ℰ′)(⟨b(e₁,e₂),♯{ρ},♯{σ},♯{τ}) ≔  
\\[\mathgobble] & \hspace{1em} ❴\mathrel{} ♯{⟦b⟧}(♯{v}₁,♯{v₂}) 
\\[\mathgobble] & \hspace{1em} ∣\mathrel{} ⟨♯{v}₁,♯{σ}₁⟩ ∈ ℰ′(⟨e₁,♯{ρ},♯{σ},♯{τ}⟩) ∧ ⟨♯{v}₂,♯{σ}₂⟩ ∈ ℰ′(⟨e₂,♯{ρ},♯{σ}₁,♯{τ}⟩) ❵
\end{alignat*}
The iteration strategy to analyze the program «e₀» is then to run «e₀» using
«ℰ», but intercepting recursive calls to:
\begin{enumerate}
  \item Cache results for all intermediate configurations «♯{ς}»; and
  \item Cache seen states to prevent infinite loops.
\end{enumerate}
(1) is required to fulfill the specification that «\$⁺» include results for all
reachable configurations from «e₀», and (2) is required to reach a fixed point
of the analysis. To track this extra information we add functional state to the
interpreter (which was done through a monad transformer in
Section~\ref{s:cache}) of type:
\[ ♯{cache} ≔ ♯{config} ↦ ℘(♯{val}×♯{store}) \]
such that the open-recursive evaluator has type:
\begin{alignat*}{1}
  & ℰ ∈ Σ → Σ \quad ⦑\emph{where}⦒ 
\\[\mathgobble] & \hspace{1em} Σ ≔ ♯{config}×♯{cache} → ℘(♯{val}×♯{store})×♯{cache}
\end{alignat*}
The iteration to compute «\$⁺» given «ℰ» is then defined:
\begin{alignat*}{1}
  & \hspace{0em} \$⁺ ≔ ⦑\emph{lfp}⦒(λ\$ᵒ. 
\\[\mathgobble] & \hspace{1em} \mathrel{⟬let⟭} ℰ⋆ ≔ Y(λℰ′.\mathrel{} ℰ(λ⟨♯{ς},\$ⁱ⟩. 
\\[\mathgobble] & \hspace{3em}      \mathrel{⟬if⟭} ♯{ς} ∈ \$ⁱ \mathrel{⟬then⟭} ⟨\$ⁱ(♯{ς}),\$ⁱ⟩ \mathrel{⟬else⟭} 
\\[\mathgobble] & \hspace{3em}      \mathrel{⟬let⟭} ⟨♯{VS},\$⸢i\prime⸣⟩ ≔ ℰ′(♯{ς},\$ⁱ[♯{ς}↦\$ᵒ(♯{ς})]) 
\\[\mathgobble] & \hspace{3em}      \mathrel{⟬in⟭} ⟨♯{VS},\$⸢i\prime⸣[♯{ς}↦♯{VS}]⟩)) 
\\[\mathgobble] & \hspace{1em} \mathrel{⟬in⟭} π₂(ℰ⋆(⟨e₀,♯{ρ}₀,♯{σ}₀,♯{τ}₀⟩,❴❵)))
\end{alignat*}
The fixed interpreter «ℰ⋆» calls the unfixed interpreter «ℰ», but intercepts
recursive calls to perform (1) and (2) described above. When loops are
detected, the results from the previous complete result «\$ᵒ» is used, and the
outer fixpoint computes the least fixed point of this «\$ᵒ».

The end result is that, rather than compute analysis results and reachable
states naively with Kleene fixpoint iteration, we are able to reuse the
standard definitional interpreter—written in open-recursive form—to
simultaneously explore reachable states, cache intermediate configurations, and
iterate towards a least fixpoint solution for the analysis. This method is more
efficient, and reuses an extensible definitional interpreter which can recover
a wide range of analyses, including concrete interpretation.

\paragraph{Widening}

Two forms of widening can be employed to the semantics and iteration algorithm
to achieve acceptable performance for the abstract interpreter.

The first form of widening is to widen the store in the result set
«℘(♯{val}×♯{store})» to «℘(♯{val})×♯{store}» in the evaluator «ℰ»:
\begin{alignat*}{1}
  & ℰ ∈ Σ → Σ \quad ⦑\emph{where}⦒ 
\\[\mathgobble] & \hspace{1em} Σ ≔ ♯{config} × ♯{cache} → ℘(♯{val})×♯{store}×♯{cache}
\end{alignat*}
We perform this widening systematically and with no added effort through the
use of Galois Transformers~\cite{local:darais-oopsla2015} in
Section~\ref{s:widening}. The iteration strategy for this widened state space
is the same as before, which computes a fixed point of the outer cache «\$ᵒ».

The next form of widening is to pull the store out of the configuration space
\emph{entirely}, that is:
\begin{alignat*}{2}
  ♯{ς} ∈ &\mathrel{} ♯{config} ≔ exp × ♯{env} × ♯{time} 
\\[\mathgobble] \$ ∈ &\mathrel{} ♯{cache} ≔ ♯{config} ↦ ℘(♯{val})
\end{alignat*}
and:
\begin{alignat*}{1}
  & ℰ ∈ Σ → Σ \quad ⦑\emph{where}⦒ 
\\[\mathgobble] & \hspace{1em} Σ ≔ ♯{config}×♯{store}×♯{cache} → ℘(♯{val})×♯{store}×♯{cache}
\end{alignat*}
The fixed point iteration then finds a mutual least fixed-point of both the
outer cache «\$ᵒ» \emph{and} the store «♯{σ}»:
\begin{alignat*}{1}
  & \hspace{0em} ⟨\$⁺,♯{σ}⁺⟩ ≔ ⦑\emph{lfp}⦒(λ⟨\$ᵒ,♯{σ}⟩. 
\\[\mathgobble] & \hspace{1em} \mathrel{⟬let⟭} ℰ⋆ ≔ Y(λℰ′. \mathrel{} ℰ(λ⟨♯{ς},♯{σ}ⁱ,\$ⁱ⟩. 
\\[\mathgobble] & \hspace{3em}      \mathrel{⟬if⟭} ♯{ς} ∈ \$ⁱ \mathrel{⟬then⟭} ⟨\$ⁱ(♯{ς}),σⁱ,\$ⁱ⟩ \mathrel{⟬else⟭} 
\\[\mathgobble] & \hspace{3em}      \mathrel{⟬let⟭} ⟨♯{V},♯{σ}⸢i\prime⸣,\$⸢i\prime⸣⟩ ≔ ℰ′(♯{ς},♯{σ}ⁱ,\$ⁱ[♯{ς}↦\$ᵒ(♯{ς})]) 
\\[\mathgobble] & \hspace{3em}      \mathrel{⟬in⟭} ⟨♯{V},♯{σ}⸢i\prime⸣,\$⸢i\prime⸣[♯{ς}↦♯{V}]⟩)) 
\\[\mathgobble] & \hspace{1em} \mathrel{⟬in⟭} π⸤2×3⸥(ℰ⋆(⟨e₀,♯{ρ}₀,♯{τ}₀⟩,♯{σ},❴❵)))
\end{alignat*}
This second version of widening, which computes a fixpoint also over the store,
recovers a so-called \emph{flow-insensitive} analysis. In this model, all
program states are re-analyzed in the store resulting from execution. Also, the
cache («\$») does not index over store states «♯{σ}» in its domain, greatly
reducing its size, and leading to a much more efficient (although less precise)
static analyzer.

\paragraph{Recovering Classical 0CFA}

From the fully widened static analyzer, which computes a mutual fixpoint
between a cache and store, we can easily recover a classical 0CFA analysis. We
do this by instantiating «♯{time}» to the singleton abstraction «❴•❵», as was
shown in Section~\ref{s:interp}. In this setting, the lexical environment «ρ»
is uniquely determined by the program expression «e», and can therefore be
eliminated, resulting in the analysis state space:
\begin{alignat*}{2}
  ♯{ς} ∈ &\mathrel{} ♯{config} ≔ exp 
\\[\mathgobble] \$ ∈ &\mathrel{} ♯{cache} ≔ exp ↦ ℘(♯{val}) 
\\[\mathgobble] ♯{σ} ∈ &\mathrel{} ♯{store} ≔ var ↦ ℘(♯{val})
\end{alignat*}
The specification for the analysis and the fully store-widened least
fixed-point iteration for computing it recovers the constraint-based
description of 0CFA given by Nielson \emph{et al}
in~\cite{dvanhorn:Neilson:1999}, where 0CFA is defined as the smallest cache
(«\$») and store («σ») which satisfy a co-inductively defined judgment: «\$,σ ⊨
e».

\paragraph{Recovering Pushdown Analysis}

We borrow from the recent result in pushdown analysis by Gilray \emph{et
al}~\cite{local:p4f} which shows that full pushdown precision can be achieved
in a small-step store-widened abstract semantics by allocating continuations
using a particular address space: program expressions paired with abstract
environments («⟨e,♯{ρ}⟩»). In other words, «⟨e,♯{ρ}⟩» is sufficient to achieve
full pushdown precision because the tuple uniquely identifies the evaluation
context up to the final result of evaluation.

Our fully widened semantics recovers pushdown precision because the cache maps
tuples «⟨e,♯{ρ},♯{τ}⟩», which contains «⟨e,♯{ρ}⟩». We then see that abstract
time «♯{τ}» is redundant and eliminate it from the cache, resulting in a
smaller domain for the same analysis:
\begin{alignat*}{2}
  ♯{ς} ∈ &\mathrel{} ♯{config} ≔ exp × ♯{env} × ♯{time} 
\\[\mathgobble] \$ ∈ &\mathrel{} ♯{cache} ≔ exp × ♯{env} ↦ ℘(♯{val}) 
\\[\mathgobble] ♯{σ} ∈ &\mathrel{} ♯{store} ≔ var × ♯{addr} ↦ ℘(♯{val})
\end{alignat*}
An advantage of our setting is that we recover pushdown analysis also for
varying degrees of store-widening, which is not the case in Gilray \emph{et
al}~\cite{local:p4f}, although pushdown precision for non-widened semantics has
been achieved by Johnson and Van Horn~\cite{dvanhorn:Johnson2014Abstracting}.
Furthermore, the implementation of our analyzer inherits this precision through
precise call-return matching in the defining metalanguage, requiring no added
instrumentation to the state-space of the analyzer.

Going back to Nielson et al~\cite{dvanhorn:Neilson:1999}, it would be
interesting to redevelop their constraint-based analysis descriptions of kCFA
in a form that recovers pushdown precision. Such an exercise would amount to
translating our big-step abstract semantics instantiated to kCFA to a
constraint system. The resulting system would differ from classical kCFA by the
addition of environments «♯{ρ}» (which Nielson et al call context environments)
to the domain of the cache. In this way our formal framework is able to bridge
the gap between results in pushdown analysis described via small-step machines
\emph{a la} Van Horn and Might~\cite{dvanhorn:VanHorn2010Abstracting}, and
constraint-based systems \emph{a la} Nielson et al for which pushdown analysis
has yet to be described effectively.


\bibliography{davdar,dvanhorn,local}

\end{document}
