\begin{figure} %{-{
\rfloat{⸨alloc^@⸩}
\begin{lstlisting}
¦ (define (alloc x) (return x))
\end{lstlisting}
\vspace{0.75em}
\figskip\rfloat{⸨store-nd@⸩}
\begin{lstlisting}
¦ (define (find a)
¦   (do σ ← get-store
¦       (for/monad+ ([v (σ a)])
¦         (return v))))
¦ (define (ext a v)
¦   (update-store (λ (σ) (σ a (if (∈ a σ) (set-add (σ a) v) (set v))))))
\end{lstlisting}
\vspace{-0.75em}
\caption{Abstracting Allocation: 0CFA}
\label{f:0cfa-abs}
\vspace{-1em}
\end{figure} %}-}

\section{Caching and Finding Fixpoints}\label{s:cache}

At this point, the interpreter obtained by linking together ⸨monad^@⸩, ⸨δ^@⸩,
⸨alloc^@⸩ and ⸨store-nd@⸩ components will only ever visit a finite number of
configurations for a given program. A configuration (⸨ς⸩) consists of an
expression (⸨e⸩), environment (⸨ρ⸩) and store (⸨σ⸩). This configuration is
finite because: expressions are finite in the given program; environments are
maps from variables (again, finite in the program) to addresses; the addresses
are finite thanks to ⸨alloc^⸩; the store maps addresses to sets of values; base
values are abstracted to a finite set by ⸨δ^⸩; and closures consist of an
expression and environment, which are both finite.

Although the interpreter will only ever see a finite set of inputs, it
\emph{doesn't know it}.  A simple loop will cause the interpreter to diverge:
ℑ⁅
¦ > (rec f (λ (x) (f x)) (f 0))
ℑ,
¦ timeout
ℑ⁆
To solve this problem, we introduce a \emph{cache} (⸨¢⸢in⸣⸩) as input to the
algorithm, which maps from configurations (⸨ς⸩) to sets of value-and-store
pairs (⸨v×σ⸩). When a configuration is reached for the second time, rather than
re-evaluating the expression and entering an infinite loop, the result is
looked up from ⸨¢⸢in⸣⸩, which acts as an oracle. It is important that the cache
is used co-inductively: it is only safe to use ⸨¢⸢in⸣⸩ as an oracle so long as
some progress has been made first. 

The results of evaluation are then stored in an output cache (⸨¢⸢out⸣⸩), which
after the end of evaluation is “more defined” than the input cache (⸨¢⸢in⸣⸩),
again following a co-inductive argument. The least fixed-point of ⸨¢⁺⸩ of an
evaluator which transforms an oracle ⸨¢⸢in⸣⸩ and outputs a more defined oracle
⸨¢⸢out⸣⸩ is then a sound approximation of the program, because it
over-approximate all finite-number of unrollings of the unfixed evaluator. We
formalize this co-inductive process in Section~\ref{s:formalism} and prove it
sound; in this section we instead focus on the intuition and implementation for
the algorithm.

The co-inductive caching algorithm is shown in Figure~\ref{f:caching}, along
with the monad transformer stack ⸨monad-cache@⸩ which has two new components:
⸨ReaderT⸩ for the input cache ⸨¢⸢in⸣⸩, and ⸨StateT+⸩ for the output cache
⸨¢⸢out⸣⸩. We use a ⸨StateT+⸩ instead of ⸨WriterT⸩ monad transformer in the
output cache so it can double as tracking the set of seen states. The ⸨+⸩ in
⸨StateT+⸩ signifies that caches for multiple non-deterministic branches will be
merged automatically, producing a set of results and a single cache, rather
than a set of results paired with individual caches.

In the algorithm, when a configuration ⸨ς⸩ is first encountered, we place an
entry in the output cache mapping ⸨ς⸩ to «𝔥⸨(¢⸢in⸣⸩\ 𝔥⸨ς)⸩», which is the
“oracle” result. Also, whenever we finish computing the result ⸨v×σ⸩ of
evaluating a configuration ⸨ς⸩, we place an entry in the output cache mapping
⸨ς⸩ to ⸨v×σ′⸩. Finally, whenever we reach a configuration ⸨ς⸩ for which a
mapping in the output cache exists, we use it immediately, ⸨return⸩ing each
result using the ⸨for/monad+⸩ iterator. Therefore, every “cache hit” on
⸨¢⸢out⸣⸩ is in one of two possible states: 1) we have already seen the
configuration, and the result is the oracle result, as desired; or 2) we have
already computed the “improved” result (w.r.t. the oracle), and need not
recompute it.

To compute the least fixed-point ⸨¢⁺⸩ for the evaluator ⸨ev-cache⸩ we perform a
standard Kleene fixpoint iteration starting from the empty map, the bottom
element for the cache, as shown in Figure~\ref{f:fixing}.

The algorithm runs the caching evaluator ⸨eval⸩ on the given program ⸨e⸩ from
the initial environment and store. This is done inside of ⸨mlfp⸩, a monadic
least fixed-point finder. After finding the least fixed-point, the final values
and store for the initial configuration ⸨ς⸩ are extracted and returned.

Termination of the least fixed-point is justified by the monotonicity of the
evaluator (it always returns an “improved” oracle), and the finite domain of
the cache, which maps abstract configurations to pairs of values and stores,
all of which are finite.

\begin{figure} %{-{
\rfloat{⸨monad-cache@⸩}
\begin{flalign*}
& 𝔥⸨(define-monad (⸩\!\up{𝔥⸨ReaderT⸩}⸢env⸣\ 𝔥⸨(⸩\!\up{𝔥⸨FailT⸩}⸢errors⸣\ 𝔥⸨(⸩\!\up{𝔥⸨StateT⸩}⸢store⸣\ 𝔥⸨(⸩\!\up{𝔥⸨NondetT⸩}⸢mplus⸣\ 𝔥⸨(⸩\!\up{𝔥⸨ReaderT⸩}⸢in-\$⸣\ 𝔥⸨(⸩\!\up{𝔥⸨StateT+⸩}⸢out-\$⸣\ 𝔥⸨ID)))))))⸩ &
\end{flalign*}
\figskip\rfloat{⸨ev-cache@⸩}
\begin{lstlisting}
¦ (define (((ev-cache ev₀) ev) e)
¦   (do ρ   ← ask-env  σ ← get-store
¦       ς   ≔ (list e ρ σ)
¦       ¢⸢out⸣ ← get-cache-out
¦       (if (∈ ς ¢⸢out⸣)
¦           (for/monad+ ([v×σ (¢⸢out⸣ ς)])
¦             (do (put-store (cdr v×σ))
¦                 (return (car v×σ))))
¦           (do ¢⸢in⸣    ← ask-cache-in
¦               v×σ₀  ≔ (if (∈ ς ¢⸢in⸣) (¢⸢in⸣ ς) ∅)
¦               (put-cache-out (¢⸢out⸣ ς v×σ₀))
¦               v     ← ((ev₀ ev) e)
¦               σ′    ← get-store
¦               v×σ′  ≔ (cons v σ′)
¦               (update-cache-out (λ (¢⸢out⸣) (¢⸢out⸣ ς (set-add (¢⸢out⸣ ς) v×σ′))))
¦               (return v)))))
\end{lstlisting}
\vspace{-0.75em}
\caption{Co-inductive Caching Algorithm}
\label{f:caching}
\vspace{-1em}
\end{figure} %}-}

With these peices in place we construct a complete interpreter:
\begin{alignat*}{1}
& 𝔥⸨(define (eval e) (mrun ((fix-cache (fix (ev-cache ev))) e)))⸩
\end{alignat*}
When linked with ⸨δ^⸩ and ⸨alloc^⸩, this abstract interpreter is sound and
computable, as demonstrated on the following examples:
ℑ⁅
¦ > (rec f (λ (x) (f x))
¦     (f 0))
ℑ,
¦ '()
ℑ;
¦ > (rec f (λ (n) (if0 n 1 (* n (f (- n 1)))))
¦     (f 5))
ℑ,
¦ '(N)
ℑ;
¦ > (rec f (λ (x) (if0 x 0 (if0 (f (- x 1)) 2 3)))
¦      (f (+ 1 0)))
ℑ,
¦ '(0 2 3)
ℑ⁆

