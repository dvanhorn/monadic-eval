\section{A Definitional Interpreter}

We begin by first constructing a definitional interpreter for a small but
representative higher-order, functional language.  As our defining language, we
use an applicative subset of Racket, a dialect of Scheme.\footnote{This choice
is largely immaterial: any functional language would do.}

The abstract syntax of the language is defined in Figure~\ref{f:syntax}; it
includes variables, numbers, unary and binary operations on numbers,
conditionals, {\tt letrec} expressions, functions, and applications.

\begin{figure}
\begin{alignat*}{4}
   e ∈ &&\mathrel{}   exp ⩴ &\mathrel{} 𝔥⸨(vbl⸩\ x𝔥⸨)⸩             &\hspace{3em} [⦑\emph{variable}⦒]
\\     &&\mathrel{}       ∣ &\mathrel{} 𝔥⸨(num⸩\ n𝔥⸨)⸩             &\hspace{3em} [⦑\emph{number}⦒]
\\     &&\mathrel{}       ∣ &\mathrel{} 𝔥⸨(lam⸩\ x\ 𝔥⸨)⸩       &\hspace{3em} [⦑\emph{lambda}⦒]
\\     &&\mathrel{}       ∣ &\mathrel{} 𝔥⸨(ifz⸩\ e\ e\ e𝔥⸨)⸩    &\hspace{3em} [⦑\emph{conditional}⦒]
\\     &&\mathrel{}       ∣ &\mathrel{} 𝔥⸨(op1⸩\ u\ e𝔥⸨)⸩       &\hspace{3em} [⦑\emph{unary op}⦒]
\\     &&\mathrel{}       ∣ &\mathrel{} 𝔥⸨(op2⸩\ b\ e\ e𝔥⸨)⸩ &\hspace{3em} [⦑\emph{binary op}⦒]
\\     &&\mathrel{}       ∣ &\mathrel{} 𝔥⸨(app⸩\ e\ e𝔥⸨)⸩       &\hspace{3em} [⦑\emph{application}⦒]
\\     &&\mathrel{}       ∣ &\mathrel{} 𝔥⸨(lrc⸩\ x\ e\ e𝔥⸨)⸩ &\hspace{3em} [⦑\emph{letrec}⦒]
\\ x ∈ &&\mathrel{}   var ≔ &\mathrel{} ❴𝔥⸨x⸩, 𝔥⸨y⸩, …❵
\\ u ∈ &&\mathrel{}  unop ≔ &\mathrel{} ❴𝔥⸨add1⸩, …❵
\\ b ∈ &&\mathrel{} binop ≔ &\mathrel{} ❴𝔥⸨+⸩, 𝔥⸨-⸩, …❵
\end{alignat*}
\caption{Syntax}
\label{f:syntax}
\end{figure}

\begin{figure}
\rfloat{⸨ev@⸩}
\begin{lstlisting}
¦ (define ((ev ev) e)
¦   (match e
¦     [(num n) (return n)]
¦     [(vbl x)
¦      (do ρ ← ask-env
¦          (find (ρ x)))]    
¦     [(ifz e₀ e₁ e₂) 
¦      (do v  ← (ev e₀)
¦          z? ← (zero? v)
¦          (ev (if z? e₁ e₂)))]
¦     [(op1 o e₀)
¦      (do v ← (ev e₀)
¦          (δ o v))]   
¦     [(op2 o e₀ e₁)
¦      (do v₀ ← (ev e₀)
¦          v₁ ← (ev e₁)
¦          (δ o v₀ v₁))]
¦     [(lrc f e₀ e₁) 
¦      (do ρ  ← ask-env
¦          a  ← (alloc f)
¦          ρ′ ≔ (ρ f a)
¦          (ext a (cons e₀ ρ′))
¦          (local-env ρ′
¦            (ev e₁)))]
¦     [(lam x e₀)
¦      (do ρ ← ask-env
¦          (return (cons (lam x e₀) ρ)))]
¦     [(app e₀ e₁)
¦      (do (cons (lam x e₂) ρ) ← (ev e₀)
¦          v₁ ← (ev e₁)
¦          a  ← (alloc x)         
¦          (ext a v₁)
¦          (local-env (ρ x a) 
¦            (ev e₂)))]))
\end{lstlisting}
\caption{Definitional Interpreter}
\label{f:interpreter}
\end{figure}

The interpreter for the language is defined in Figure~\ref{f:interpreter}. At
first glance, it has many conventional aspects:
\begin{itemize}
\item it is compositionally defined by structural recursion on the syntax of
expressions,
\item it represents functions with a closure data structure that pairs together
the code with the environment in which a function definition was evaluated,
\item it is structured monadically and uses monad operations to interact with
the environment and store, and
\item it relies on a helper function ⸨δ⸩ to interpret primitive operations.
\end{itemize}

There are a few superficial aspects that deserve a quick note:
environments ⸨ρ⸩ are finite maps and ⸨(ρ«\ »x)⸩ denotes
«ρ(x)» while ⸨(ρ«\ »x a)⸩ denotes «ρ[x↦a]».  The
⸨do⸩-notation is just shorthand for ⸨bind⸩, as usual:
\begin{alignat*}{2}
   𝔥⸨(do x ←«\ »e . r)⸩ ≡ &\mathrel{} 𝔥⸨(bind e (λ«\ »(x) (do . r)))⸩
\\ 𝔥⸨(do x ≔«\ »e . r)⸩ ≡ &\mathrel{} 𝔥⸨(let ((x e)) (do . r))⸩
\\           𝔥⸨(do e . r)⸩ ≡ &\mathrel{} 𝔥⸨(bind e (λ«\ »(_) (do . r)))⸩
\\               𝔥⸨(do b)⸩ ≡ &\mathrel{} 𝔥⸨b⸩
\end{alignat*}

Finally, there are two unconvential aspects worth noting.  First, the
interpreter is written in an \emph{open recursive style}; the evaluator does
not call itself recursively, instead it takes as an argument a function
⸨ev⸩ which it calls in order to recur.  (We have employed a bit of
cuteness by naming the first parameter ⸨ev⸩, thereby shadowing the outer
⸨ev⸩ and making subsequent calls look like recursive calls.)  This is a
standard encoding for recursive functions in a setting without recursive
binding.  It is up to an external function, such as the Y-combinator, to close
the recursive loop.  As we will see, this open recursive form will be crucial
for interposition to collect information about the intensional properties of
evaluation.

Second, the code is clearly \emph{incomplete}.  There are a number
of free variables, noted in italics.  These free variables fall into
a few roles:
\begin{itemize}
\item providing the underlying monad of the interpreter:
⸨return⸩ and ⸨bind⸩,

\item providing an interpretation of primitives: ⸨δ⸩ and
⸨zero?⸩,

\item providing environment operations: ⸨ask-env⸩ for
retreiving the environment and ⸨local-env⸩ for installing an
environment,

\item providing store operations: ⸨ext⸩ for updating
the store, and ⸨find⸩ for dereferencing locations, and

\item a remaining operation for ⸨alloc⸩ating store locations,
used to bind variables.
\end{itemize}

Going forward, we make frequent use of sets of definitions involving free
variables, so we call such a collection a \emph{component}. We assume
components can be named (in this case, we've named the component ⸨ev@⸩,
indicated by the box in the upper-right corner) and linked together to
eliminate free variables.\footnote{We use Racket
\emph{units}~\cite{local:flatt-pldi98} to model components in our
implementation.}

\begin{figure}
\rfloat{⸨monad@⸩}
\begin{lstlisting}
¦ (define-monad 
¦   (ReaderT (FailT (StateT ID))))
\end{lstlisting}
\figskip\rfloat{⸨δ@⸩}
\begin{lstlisting}
¦ (define (δ . ovs)
¦   (match ovs
¦     [(list 'add1 n)  (return (add1 n))]
¦     [(list 'sub1 n)  (return (sub1 n))]
¦     [(list '- n)     (return (- n))]
¦     [(list '+ n₀ n₁) (return (+ n₀ n₁))]
¦     [(list '- n₀ n₁) (return (- n₀ n₁))]
¦     [(list '* n₀ n₁) (return (* n₀ n₁))]
¦     [(list 'quotient n₀ n₁)
¦      (if (= 0 n₁)
¦          fail
¦          (return (quotient n₀ n₁)))]))
¦ (define (zero? v)
¦   (return (= 0 v)))
\end{lstlisting}
\figskip\rfloat{⸨store@⸩}
\begin{lstlisting}
¦ (define (find a)
¦   (do σ ← get-store
¦       (return (σ a))))
¦ (define (ext a v) 
¦   (update-store (λ (σ) (σ a v))))
\end{lstlisting}
\figskip\rfloat{⸨alloc@⸩}
\begin{lstlisting}
¦ (define (alloc x)
¦   (do σ ← get-store
¦       (return (size σ))))
\end{lstlisting}
\caption{Components for Definitional Interpreters}
\label{f:concrete-components}
\end{figure}

Let us now examine a set of components for completing the definitional
interpreter.  Figure~\ref{f:concrete-components} gives the definition
for a series of components that complete the interpreter.  The first
and most magical component is ⸨monad@⸩, which uses our
⸨define-monad⸩ macro to generate a set of bindings based on a
monad transformer stack.  For this interpreter, we use a failure monad
to model divide-by-zero errors, a state monad to model the store, and
a reader monad to model the environment.  The ⸨define-monad⸩
form generates bindings for ⸨return⸩, ⸨bind⸩,
⸨ask-env⸩, ⸨local-env⸩, ⸨get-store⸩ and
⸨update-store⸩. 

We also add a ⸨mrun⸩ operation for running computations, which
kicks off the computation by providing the empty environment and
store:
\begin{lstlisting}
¦ (define (mrun m)
¦   (run-StateT ∅ (run-ReaderT ∅ m)))
\end{lstlisting}
While the ⸨define-monad⸩ form is hiding some details, this
component could have equivalently been written out explicitly. For
example, ⸨return⸩ and ⸨bind⸩ can be defined as:
\begin{lstlisting}
¦ (define (((return a) r) s) (cons a s))
¦ (define (((bind ma f) r) s)
¦   (match ((ma r) s)
¦     [(cons a s′) (((f a) r) s′)]
¦     ['failure 'failure]))
\end{lstlisting}
And the remaining operations are straightforward, too.  So the use of
monad transformers can be seen as a mere convenience, but as we will
see moving to more and more involved monad stacks, it's a useful one.

The ⸨δ@⸩ component defines the interpretation of primitives,
which is given in terms of the underlying monad.  Finally the
⸨alloc@⸩ component provides a definition of ⸨alloc⸩,
which fetches the store and uses its size to return a fresh address.

The ⸨store@⸩ component defines the derived operations on stores
of ⸨find⸩ and ⸨ext⸩ for finding and extending the store
in terms of the monadic operations.

The only remaing pieces of the puzzle are a fixed-point combinator,
which is straightforward to define:
\begin{lstlisting}
¦ (define ((fix f) x) ((f (fix f)) x))
\end{lstlisting}
And the main entry-point for the interpreter:
\begin{lstlisting}
¦ (define (eval e) (mrun ((fix ev) e)))
\end{lstlisting}

By taking advantage of Racket's languages-as-libraries
features~\cite{dvanhorn:TobinHochstadt2011Languages}, we can easily construct
REPLs for interacting with this interpreter.  Here are a few examples, which
make use of a concrete syntax for more succinctly writing expressions. The
identity function evaluates to an answer consisting of a closure over the empty
environment together with the empty store:
ℑ⁅
¦ > (λ (x) x)
ℑ,
¦ '(((λ (x) x) . ()) . ())
ℑ⁆
Here's an example showing a non-empty environment and store:
ℑ⁅
¦ > ((λ (x) (λ (y) x)) 4)
ℑ,
¦ '(((λ (y) x) . ((x . 0))) . ((0 . 4)))
ℑ⁆
Primitive operations work as expected:
ℑ⁅
¦ > (* (+ 3 4) 9)
ℑ,
¦ '(63 . ())
ℑ⁆
And divide-by-zero errors result in failures:
ℑ⁅
¦ > (quotient 5 (- 3 3))
ℑ,
¦ '(failure . ())
ℑ⁆
Because our monad stack places ⸨FailT⸩ above ⸨StateT⸩,
the answer includes the (empty) store at the point of the error.
Had we changed ⸨monad@⸩ to use:
\begin{lstlisting}
(define-monad 
  (ReaderT (StateT (FailT ID))))
\end{lstlisting}
failures would not include the store:
ℑ⁅
¦ > (quotient 5 (- 3 3))
ℑ,
¦ '(failure . ())
ℑ⁆

At this point, we've defined a fairly run of the mill definitional
interpreter.  Despite these pedestrian beginnings, we essentially have
the complete skeleton for everything to come.  In particular, we will
reuse ⸨ev@⸩ in all of the remaining interpreters.  Now let's do
something a bit more enchanting.

\subsection{Collecting Variations}\label{s:collecting}

The formal development of abstract interpretation often starts from a
so-called ``non-standard collecting semantics.''  A common form of
collecting semantics is a trace semantics, which collects streams of
states the interpreter reaches.  Figure~\ref{f:trace} shows the monad
stack for a tracing interpreter and a kind of ``mix-in'' for the
evaluator.  The monad stack adds ⸨WriterT⸩ using
⸨List⸩, which provides a new operation named ⸨tell⸩ for
writing items to the stream of reached states.  The ⸨ev-trace⸩
function is a wrapper around an underlying ⸨ev⸩ function which
interposes itself between each recursive call by telling the current
state of the evaluator, that is the current expression, environment,
and store.  The top-level evaluation function is then:
\begin{lstlisting}
¦ (define (eval e) 
¦   (mrun ((fix (ev-tell ev)) e)))
\end{lstlisting}

\begin{figure}
\rfloat{⸨trace-monad@⸩}
\begin{lstlisting}
¦ (define-monad
¦   (ReaderT (FailT (StateT (WriterT List ID)))))
\end{lstlisting}
\figskip\rfloat{⸨ev-tell@⸩}
\begin{lstlisting}
¦ (define (((ev-tell ev₀) ev) e)
¦   (do ρ ← ask-env
¦       σ ← get-store
¦       (tell (list e ρ σ))
¦       ((ev₀ ev) e)))
\end{lstlisting}
\caption{Trace Collecting Semantics}
\label{f:trace}
\end{figure}

Now when an expression is evaluated, we get the resulting answer and a list of
all the states seen by the evaluator, in the order in which they were seen.
For example:
ℑ⁅
¦ > (* (+ 3 4) 9)
ℑ,
¦ '((63 . ())
¦   ((* (+ 3 4) 9)()())
¦   ((+ 3 4) () ())
¦   (3 () ())
¦   (4 () ())
¦   (9 () ()))
ℑ;
¦ > ((λ (x) (λ (y) x)) 4)
ℑ,
¦ '((((λ (y) x) . ((x . 0))) . ((0 . 4)))
¦   (((λ (x) (λ (y) x)) 4) () ())
¦   ((λ (x) (λ (y) x)) () ())
¦   (4 () ())
¦   ((λ (y) x) ((x . 0)) ((0 . 4))))
ℑ⁆

Were we to swap ⸨List⸩ with ⸨Set⸩ in the monad stack, we would obtain a
\emph{reachable} state semantics, another common form of collecting semantics,
that loses the order and repetition of states.

As another collecting semantics variant, we can also consider collecting the
\emph{dead code} in a program.  Here we use a monad stack that has an addition
state component (with operations named ⸨put-dead⸩ and ⸨get-dead⸩), which will
the set of dead expressions.  Initially, this will contain all of the
subexpressions of the program.  As the interpreter recurs through expressions,
it will remove them from the dead set.

Figure~\ref{f:dead} defines the monad stack for the dead code collecting
semantics and the ⸨ev-dead@⸩ component which interposes itself on an ⸨ev⸩
function to remove the given subexpression before recurring.  Since computing
the dead code requires an outer wrapper that sets the initial set of dead code
to be all of the subexpressions in the program, it requires a ⸨eval-dead@⸩
component which consumes a \emph{closed evaluator}, i.e. something of the form
⸨(fix ev)⸩.

Putting these pieces together, the dead code collecting semantics can be
defined as:
\begin{lstlisting}
¦ (define (eval e)
¦   (mrun ((eval-dead (fix (ev-dead ev))) e)))
\end{lstlisting}

Running a program with the dead code interpreter produces an answer
and the set of expressions that were not evaluated during the running
of a program:
ℑ⁅
¦ > (if0 0 1 2)
ℑ,
¦ (cons '(1 . ()) (set 2)
ℑ;
¦ > (* (+ 3 4) 9)
ℑ,
¦ (cons '(63 . ()) (set))
ℑ;
¦ > (λ (x) x)
ℑ,
¦ (cons '(((λ (x) x) . ()) . ()) (set 'x))
ℑ;
¦ > (if0 (quotient 1 0) 2 3)]
ℑ,
¦ (cons '(failure . ()) (set 3 2))
ℑ⁆

\begin{figure}
\rfloat{⸨dead-monad@⸩}
\begin{lstlisting}
¦ (define-monad
¦   (ReaderT (StateT (StateT (FailT ID)))))
\end{lstlisting}
\figskip\rfloat{⸨ev-dead@⸩}
\begin{lstlisting}
¦ (define (((ev-dead ev₀) ev) e)
¦   (do θ  ← get-dead       
¦       (put-dead (set-remove θ e))
¦       ((ev₀ ev) e)))
\end{lstlisting}
\figskip\rfloat{⸨eval-dead@⸩}
\begin{lstlisting}
¦ (define ((eval-dead eval) e₀)
¦   (do (put-dead (subexps e₀))
¦       (eval e₀)))
\end{lstlisting}
\caption{Dead Code Collecting Semantics}
\label{f:dead}
\end{figure}

So our setup makes it easy not only to express the run of the mill
interpreter, but also different forms of collecting semantics.
Let us now start to look at abstractions.

\subsection{Abstracting Base Values}\label{s:base}

One of the things an abstract interpreter must do in order to become
decidable is to have some form of abstraction for the base types of
the language.  A very simple approach is to use a finite-element
abstract domain.  We can do this for our sole base type of numbers by
introducing a new kind of number, written ⸨'N⸩, which is an
abstract value that stands for all numbers.  Abstract values will be
introduced by alternative interpretation of the primitive operations,
given in Figure~\ref{f:abs-delta}, which simply produces ⸨'N⸩ in
all cases.  Some care must be taken in the interpretation of
⸨'quotient⸩ since if the denominator is an abstract value, the
result must include a failure since ⸨0⸩ is in the set of values
abstracted by ⸨'N⸩.  This means that dividing a number by an
abstract value must produce \emph{two answers}: ⸨'N⸩ and
⸨'failure⸩.  This is done by adding non-determinism to the
monad stack,

\begin{lstlisting}
¦ (ReaderT (FailT (StateT (NondetT ID))))
\end{lstlisting}
which provides a ⸨mplus⸩ operation for combining multiple
answers.  Non-determinism is also used in the implementation of
⸨zero?⸩, which returns both true and false on ⸨'N⸩.

\begin{figure}
\rfloat{⸨δ^@⸩}
\begin{lstlisting}
¦ (define (δ . ovs)
¦   (match ovs
¦     [(list 'add1 n)  (return 'N)]
¦     [(list 'sub1 n)  (return 'N)]
¦     [(list '+ n₀ n₁) (return 'N)]
¦     [(list '- n₀ n₁) (return 'N)]
¦     [(list '* n₀ n₁) (return 'N)]
¦     [(list 'quotient n₀ (? number? n₁))
¦      (if (= 0 n₁) fail (return 'N))]
¦     [(list 'quotient n₀ n₁)
¦      (mplus (return 'N) fail)]))
¦ (define (zero? v)
¦   (match v
¦     ['N (mplus (return #t) (return #f))]
¦     [_  (return (= 0 v))]))
\end{lstlisting}
\caption{Abstracting Primitive Operations}
\label{f:abs-delta}
\end{figure}

By linking together the abstract variant of ⸨δ⸩ and the monad
stack with non-determinism, we can obtain an evaluator that produces a
set of results:
ℑ⁅
¦ > (* (+ 3 4) 9)
ℑ,
¦ '((N . ()))
ℑ;
¦ > (quotient 5 (add1 2))
ℑ,
¦ '((failure . ()) (N . ()))
ℑ;
¦ > (if0 (add1 0) 3 4)
ℑ,
¦ '((3 . ()) (4 . ()))
ℑ⁆

If we were to link together the abstract variant of ⸨δ⸩ with
the \emph{tracing} monad stack with non-determinism added in,
\begin{lstlisting}
¦ (ReaderT (FailT (StateT 
¦   (WriterT List (NondetT ID)))))
\end{lstlisting}
we would get an evaluator that produces sets of traces:
ℑ⁅
¦ > (if0 (add1 0) 3 4)
ℑ,
¦ (set
¦  '((3 . ())
¦    ((if0 (add1 0) 3 4) () ())
¦    ((add1 0) () ())
¦    (0 () ())
¦    (3 () ()))
¦  '((4 . ())
¦    ((if0 (add1 0) 3 4) () ())
¦    ((add1 0) () ())
¦    (0 () ())
¦    (4 () ())))
ℑ⁆


It should be clear that the interepreter will only ever see a finite
set of numbers (including ⸨'N⸩), but it's definitly not true
that the interpreter halts on all inputs.  Firstly, it's still
possible to generate an infinite number of closures.  Secondly,
there's no way for the interpreter to detect when it sees a loop.  To
make a terminating abstract interpreter requires tackling both.  Let's
look next at abstracting closures.
