\section{Fixing the Cache}\label{s:fixing-cache}

The basic problem with the caching solution of Section~\ref{s:cache} is that
it cuts short the exploration of the program's behavior.  In the
soundness counter-example, the inner call to ⸨f⸩ is present in
the cache so neither branch of the conditional is taken; it is at this
point of bottoming out that we determine ⸨f⸩ may return
⸨0⸩.  Of course, now we know that the conditional should have
take the true branch since ⸨0⸩ could be returned, but it's too
late: the program has terminated.

To restore soundness, what we need to do is somehow \emph{iterate} the
interpreter so that we can re-explore the behavior knowing that
⸨f⸩ may produce ⸨0⸩.  A first thought may be to do a
complete evaluation of the program, take the resulting cache, and then
feed that in as the initial cache for a re-evaluation of the program.
But there's an obvious problem... doing so would result in a cache hit
and the saved results would be returned immediately without exploring
any new behavior.

The real solution is that we want to use the prior evaluation's cache
as a kind of co-inductive hypothesis: it's only when we detect a loop
that we want to produce all of the results stored in the prior cache.
This suggests a two-cache system in which the prior cache is only used
when initializing the local cache.  In other words, we want to use the
prior cache entry in the place of ⸨∅⸩.  When iterating the
interpreter, we always start from an empty local cache and fall back
on the prior cache results when initializing the cache entry before
making a recursive call.  Since the prior cache is never written to,
we can model the prior cache as a reader monad and add it to the
stack:
\begin{lstlisting}
¦ (ReaderT (FailT (StateT (NondetT 
¦   (ReaderT ; the prior cache
¦     (StateT+ ID))))))
\end{lstlisting}

The revised ⸨ev-cache@⸩ component is given in Figure~\ref{f:ev-cache}, which
uses the ⸨ask-⊥⸩ operation to retreive the prior cache.  If the
prior cache is empty, this code degenerates into exactly what was given in
Figure~\ref{f:ev-cache0}.

\begin{figure}
\rfloat{⸨ev-cache@⸩}
\begin{lstlisting}
¦ (define (((ev-cache ev₀) ev) e)
¦   (do ρ ← ask-env
¦       σ ← get-store
¦       ς ≔ (list e ρ σ)
¦       Σ ← get-¢
¦       (if (∈ ς Σ)
¦           (for/monad+ ([v×σ (Σ ς)])
¦             (do (put-store (cdr v×σ))
¦                 (return (car v×σ))))
¦             (do Σ⊥ ← ask-⊥
¦               ; initialize to prior, if exists
¦               (put-¢ (Σ ς (if (∈ ς Σ⊥) (Σ⊥ ς) ∅)))
¦               v  ← ((ev₀ ev) e)
¦               (update-¢
¦                 (λ (Σ) 
¦                   (Σ ς (set-add (Σ ς)
¦                                 (cons v σ)))))
¦               (return v)))))
\end{lstlisting}
\caption{Caching, with Fall-back to Prior}
\label{f:ev-cache}
\end{figure}

We are left with two remaining problems; we need to figure out: 1) how
to pipe the cache from one run of the interpreter into the next and 2)
when to stop.  The answer to both is given in Figure~\ref{f:cache-fix}.

The ⸨fix-cache⸩ function takes a closed evaluator, just like ⸨eval-dead⸩ from
Section~\ref{s:collecting}, i.e. something of the form ⸨(fix ev)⸩.  It
iteratively runs the evaluator.  Each run of the evaluator resets the ``local''
cache to empty and uses the cache of the previous run as it's fallback cache
(initially it's empty).  The computation stops when a least fixed-point in the
cache has been reached, that is, when running the evaluator with a prior gives
no changes in the resulting cache.  At that point, the result is returned.

\begin{figure}
\rfloat{⸨fix-cache@⸩}
\begin{lstlisting}
¦ (define ((fix-cache eval) e)  
¦   (do ρ ← ask-env
¦       σ ← get-store
¦       ς ≔ (list e ρ σ)
¦       (mlfp (λ (Σ) (do (put-¢ ∅-map)
¦                        (put-store σ)
¦                        (local-⊥ Σ (eval e))
¦                        get-¢)))
¦       Σ ← get-¢
¦       (for/monad+ ([v×σ (Σ ς)])
¦         (do (put-store (cdr v×σ))
¦             (return (car v×σ))))))
¦ (define (mlfp f)
¦   (let loop ([x ∅-map])
¦     (do x′ ← (f x)
¦         (if (equal? x′ x)
¦             (return (void))
¦             (loop x′)))))
\end{lstlisting}
\caption{Finding Fixpoints in the Cache}
\label{f:cache-fix}
\end{figure}

With these peices in place, we can construct an interpreter as:
\begin{lstlisting}
¦ (define (eval e)
¦   (mrun ((fix-cache (fix (ev-cache ev))) e)))
\end{lstlisting}
When linked with ⸨δ^⸩ and ⸨alloc^⸩, this interpreter is
a computable---and we conjecture, sound---abstraction of the original
definitional interpreter.  Note that the iterated evaluator always
terminates: the cache resulting from each run of the evaluator
contains \emph{at least} as much information as the prior cache, each
run of the evaluator terminates, so the iterated evaluator terminates
by the same principle as before: the cache monotonically grows and is
finite in size.

We have thus achieved our goal and can confirm it gives
the expected answers on the previous examples:
ℑ⁅
¦ > (rec f (λ (x) (f x)) (f 0))
ℑ,
¦ '()
ℑ;
¦ > (rec fact (λ (n)
¦              (if0 n 1 (* n (fact (sub1 n)))))
¦     (fact 5))
ℑ,
¦ '(N)
ℑ;
¦ > (rec f (λ (x) 
¦            (if0 x 0 (if0 (f (sub1 x)) 2 3)))
¦      (f (add1 0)))
ℑ,
¦ '(0 2 3)
ℑ⁆
Let us now take stock of what we've got.
