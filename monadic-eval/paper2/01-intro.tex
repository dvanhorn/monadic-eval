\section{Introduction}

In his landmark paper, \emph{Definitional Interpreters for Higher-order
Programming Languages}~\cite{dvanhorn:reynolds-acm72}, Reynolds observed that
when a programming language is defined by way of an interpreter, it is possible
to inherit semantic characteristics of the defining metalanguage. For example,
it is possible to define a definitional interpreter which is call-by-value when
the metalanguage is call-by-value, and call-by-name when the metalanguage is
call-by-name.

We expand on Reynolds's observation in the setting of abstract interpreters and
discover the following:
\begin{itemize}
\item Definitional interpreters, written in monadic style, can simultaneously
  define a language's semantics as well as safe approximations of those
  semantics, \emph{i.e.} abstract interpreters.
\item These definitional \emph{abstract} interpreters can inherit
  characteristics of the defining language.  In particular, precise
  call-and-return matching can be inherited, yielding a pushdown
  analysis~\cite{dvanhorn:Earl2010Pushdown,local:vardoulakis-diss12}.
\end{itemize}
Furthermore, we contribute:
\begin{itemize}
\item A systematic methodology for designing abstract interpreters via
  definitional interpreters; and
\item A soundness framework establishing the correctness of abstract
  interpreters defined in definitional style.
\end{itemize}

\paragraph{A Compositional Approach to Program Analysis}
There are many approaches to designing program analyzers for programming
languages. For \emph{higher-order} programming languages, two popular
approaches are to use abstract machines~\cite{dvanhorn:VanHorn2010Abstracting}
and constraint systems~\cite{dvanhorn:Neilson:1999}. Other foundations exist
for designing program analyzers, but no approach to-date utilizes big-step
operational semantics or definitional interpreters for program analysis of
higher-order languages. This is unfortunate because big-step semantics and
definitional interpreters are more compositional and high-level than state
machines or constraint systems. Big-step and denotational approaches exist for
first-order programming languages; the gap is in extending these ideas to the
higher-order setting.

We bridge this gap, providing the first framework for program analysis of
higher-order languages which builds upon big-step semantics and their
corresponding definitional interpreters, which are compositional by nature. 

Our key insights are to design definitional interpreters in monadic,
open-recursive style (§~\ref{s:interp}), and to design a
novel fixpoint algorithm tailored specifically to the setting of higher-order
definitional interpreters (§~\ref{s:cache} and~\ref{s:fixing-cache}). The
extensible nature of the interpreter allows us to recover a wide-range of
analyses through its instantiation, including widening techniques
(§~\ref{s:widening}), precision preserving abstractions
(§~\ref{s:alt-abstraction}), and symbolic execution for program verification
(§~\ref{s:symbolic}). Our implementation is freely available through a suite of
embedded languages in Racket (§~\ref{s:try-it}). Finally, we prove the approach
sound w.r.t. a derived big-step collecting and abstract semantics
(§~\ref{s:formalism}), where the key insight in the formalism is to model not
only standard big-step \emph{evaluation} relations, but also big-step
\emph{reachability} relations.

\paragraph{Pushdown Precision in Program Analysis}
A common problem with traditional approaches to control flow analysis is the
inability to properly match a function call with its return in the abstract
semantics. This leads to infeasible program (abstract) executions in which a
call is made from one point in the program, but control returns to another.
The PDCFA analysis of Earl \emph{et al}~\cite{dvanhorn:Earl2010Pushdown} and
CFA2 analysis of Vardoulakis and Shivers~\cite{dvanhorn:Vardoulakis2011CFA2}
were the first approaches to overcome this limitation in the higher-order
setting. In essence, these analyses replaces the traditional finite automata
abstractions of programs with pushdown automata, an approach pioneered by Reps
\emph{et al}~\cite{dvanhorn:Reps1995Precise} in the first-order setting.

We realize a new technique for defining abstract interpreters with pushdown
precision, meaning the analysis precisely matches function calls to returns. In
the setting of definitional interpreters, this property is inherited from the
defining metalanguage and requires no instrumentation to the analysis \emph{at
all} (§~\ref{s:reynolds}).

A technical difference between small-step and big-step approaches to semantics
are that small-step methods must model the execution context of evaluation,
whether through evaluation contexts~\cite{local:felleisen-TCS1992} or stack
frames~\cite{dvanhorn:Felleisen1987Calculus}. In big-step methods, there is no
model for the context; it is implicit in the definition of evaluation rules, or
in the case of definitional interpreters, implicit in the call-and-return
semantics of the defining programming language. Because a defining programming
language is precise in its call/return behavior, so is a definitional abstract
interpreter embedded within it.
