\section{Introduction}

In his landmark paper, \emph{Definitional interpreters for higher-order
languages}~\cite{dvanhorn:reynolds-acm72}, Reynolds first observed that when a
language is defined by way of an interpreter, it is possible for the defined
language to inherit semantic characteristics of the defining language of the
interpreter. For example, it is easy to write a compositional evaluator that
defines a call-by-value language if the defining language is call-by-value, but
defines a call-by-name language if the defining language is call-by-name.

In this paper, we make the following two observations:
\begin{enumerate}
\item Definitional interpreters, written in monadic style, can simultaneously
define a language's semantics as well as safe approximations of those
semantics.

\item These definitional \emph{abstract} interpreters can inherit
characteristics of the defining language.  In particular, we show that the
abstract interpreter inherits the call and return matching property of the
defining language and therefore realizes an abstract intpretation in the
pushown style of
analyses~\cite{local:vardoulakis-diss12,dvanhorn:Earl2010Pushdown}.
\end{enumerate}
A common problem of past approaches to the control flow analysis of functional
languages is the inability to properly match a function call with its return in
the abstract semantics, leading to infeasible program (abstract) executions in
which a call is made from one point in the program, but control returns to
another.  The CFA2 analysis of Vardoulakis and
Shivers~\cite{dvanhorn:Vardoulakis2011CFA2} was the first approach that
overcame this shortcoming.  In essence, this kind of analysis can be viewed as
replacing the traditional finite automata abstractions of programs with
pushdown automata~\cite{dvanhorn:Earl2010Pushdown}.

In this paper we investigate the use of definitional interpreters as the basis
for abstract interpretation of higher-order languages.  We show that a
definitional interpreter---a compositional evaluation function---written in a
monadic and componential style can express a wide variety of concrete and
abstract interpretations.  
