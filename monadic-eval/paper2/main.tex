\documentclass[acmlarge,anonymous]{acmart}\settopmatter{printfolios=true}

\usepackage{microtype}
\usepackage{textcomp}
\usepackage[scaled=0.8]{DejaVuSansMono}
\usepackage{setspace}
\usepackage{stackengine}
\renewcommand\stacktype{L}
%\linespread{0.95}

\usepackage{balance}
\usepackage{moresize}
\usepackage{csquotes}
\usepackage{upgreek}

\usepackage{amsmath}
\usepackage{amssymb}
\usepackage{mathtools}
\usepackage{stmaryrd}

\usepackage{mathpartir}
\usepackage{galois}

%\newtheorem{lemma}{Lemma}
%\newtheorem{theorem}{Theorem}
%\newtheorem{proposition}{Proposition}

%\usepackage[usenames,dvipsnames]{color}
%\definecolor{PaleBlue}{rgb}{0.90,0.90,1.0}
%\definecolor{LightGray}{rgb}{0.90,0.90,0.90}
%\definecolor{CommentColor}{rgb}{0.76,0.45,0.12}
%\definecolor{OutputColor}{rgb}{0.59,0.00,0.59}

\definecolor{IdentifierColor}{rgb}{0.15,0.15,0.50}
\newcommand{\ID}[1]{{\color{IdentifierColor}#1}}

\definecolor{PuncColor}{rgb}{0.52,0.24,0.14}
\newcommand{\PN}[1]{{\color{PuncColor}#1}}

\definecolor{KeywordColor}{rgb}{0.52,0.24,0.14}
\newcommand{\KY}[1]{{\color{KeywordColor}#1}}

\definecolor{ValueColor}{rgb}{0.13,0.55,0.13}
\newcommand{\VL}[1]{{\color{ValueColor}#1}}

\definecolor{ResultColor}{rgb}{0.10,0.10,0.79}
\newcommand{\RE}[1]{{\color{ResultColor}#1}}

\definecolor{ErrorColor}{rgb}{0.75,0.10,0.10}
\newcommand{\EO}[1]{{\color{ErrorColor}#1}}


\usepackage{listings}
\lstset%
  {language=Lisp
  ,upquote=true
  ,mathescape=true
  ,basicstyle=\ttfamily\color{PuncColor}
  ,alsoletter=+-*/?'\#0123456789^:
  ,identifierstyle=\color{IdentifierColor}
  ,keywords=
    {define,let,match,if
    ,do,define-monad,for/monad+
    ,\#lang
    ,\#:when
    ,\#:unless
    }
  ,keywordstyle=\color{KeywordColor}
  ,emph=
    {return,bind
    ,zero?
    ,ask-env,local-env
    ,ext,find,alloc
    ,get-store,put-store,update-store
    ,tell
    ,get-dead,put-dead,update-dead
    ,fail,mplus
    ,get-,put-,update-
    ,ask-,local-
    ,get-path-cond,refine
    ,ask-cache-in,local-cache-in
    ,get-cache-out,put-cache-out,update-cache-out
    }
  ,emphstyle=\color{IdentifierColor}\emph
  ,literate=
    {‹λ›}{{{\KY{«\uplambda»}}}}1
    {‹δ›}{{{\ID{«\updelta»}}}}1
    {‹σ›}{{{\ID{«\upsigma»}}}}1
    {‹ς›}{{{\ID{«\varsigma»}}}}1
    {‹ρ›}{{{\ID{«\uprho»}}}}1
    {‹φ›}{{{\ID{«\upphi»}}}}1
    {‹θ›}{{{\ID{«\uptheta»}}}}1
    {‹Σ›}{{{\ssmall\ID{«Σ»}}}}1
    {‹∅›}{{{\ID{«∅»}}}}1
    {‹←›}{{{\scriptsize«←»}}}1
    {‹≔›}{{{\scriptsize«≔»}}}1
    {‹₀›}{{{\ID{«\,⸤⦑0⦒⸥»}}}}1
    {‹₁›}{{{\ID{«\,⸤⦑1⦒⸥»}}}}1
    {‹₂›}{{{\ID{«\,⸤⦑2⦒⸥»}}}}1
    {‹′›}{{{\ID{«\,′»}}}}1
    {‹¢›}{{{\ID{\$}}}}1
    {‹∈›}{{{\ID{«∈»}}}}1
    {‹×›}{{{\ssmall\ID{«×»}}}}1
    {‹⊥›}{{{\ssmall\ID{«⊥»}}}}1
    {‹¬›}{{{\ssmall\ID{«¬»}}}}1
    {‹ᴵ›}{{{\ID{\,«⸢⦑I⦒⸣»}}}}1
    {‹ᴼ›}{{{\ID{\,«⸢⦑O⦒⸣»}}}}1
    {‹⁺›}{{{\ID{\,«⸢⦑+⦒⸣»}}}}1
    {‹⸢›in‹⸣›}{{{\ID{«⸢⦑in⦒⸣»}}}}1
    {‹⸢›out‹⸣›}{{{\ID{«⸢⦑out⦒⸣»}}}}2
    {`}{{{\`{}}}}1
    {get-‹¢›}{{{\ID{\emph{get-\$}}}}}5
    {put-‹¢›}{{{\ID{\emph{put-\$}}}}}5
    {update-‹¢›}{{{\ID{\emph{update-\$}}}}}8
    {=}{{{\ID{=}}}}1
  ,classoffset=1
  ,keywords=
    {'add1,'sub1,'+,'-,'*,'/,'failure,'N
    ,\#t,\#f,0,1,2,3,4,5,6,7,8,9
    }
  ,keywordstyle=\color{ValueColor}
  }
\lstdefinestyle{result}
  {basicstyle=\ttfamily\color{ResultColor}
  ,keywordstyle=\color{ResultColor}
  ,identifierstyle=\color{ResultColor}
  ,literate=
    {‹λ›}{{{«\uplambda»}}}1
    {‹δ›}{{{«\updelta»}}}1
    {‹σ›}{{{«\upsigma»}}}1
    {‹ς›}{{{«\varsigma»}}}1
    {‹ρ›}{{{«\uprho»}}}1
    {‹φ›}{{{«\upphi»}}}1
    {‹θ›}{{{«\uptheta»}}}1
    {‹Σ›}{{{\ssmall«Σ»}}}1
    {‹∅›}{{{«∅»}}}1
    {‹←›}{{{«←»}}}1
    {‹≔›}{{{«≔»}}}1
    {‹₀›}{{{«\,⸤⦑0⦒⸥»}}}1
    {‹₁›}{{{«\,⸤⦑1⦒⸥»}}}1
    {‹₂›}{{{«\,⸤⦑2⦒⸥»}}}1
    {‹′›}{{{«\,′»}}}1
    {‹¢›}{{{\$}}}1
    {‹∈›}{{{«∈»}}}1
    {‹×›}{{{\ssmall«×»}}}1
    {‹⊥›}{{{\ssmall«⊥»}}}1
    {‹¬›}{{{\ssmall«¬»}}}1
    {timeout}{{{\EO{\emph{\textbf{timeout}}}}}}7
  }

\newcommand{\rfloat}[1]{\begin{flushright}\fbox{#1}\end{flushright}\vspace{-2.1em}} % -2.1em}}
\newcommand{\resultskip}{\vspace{-0.8em}} % -1.2em}}
\newcommand{\figskip}{\vspace{0pt}} % -1.3em}}
\newcommand{\captionskip}[1]{\vspace{-0.75em}\caption{#1}\vspace{-1em}}
\newcommand{\up}[1]{\overbrace{\vphantom{X^X}#1}}
\newcommand{\mathgobble}{-0.25em}
\newcommand{\monadgobble}{-0.5em}

\makeatletter\if@ACM@journal\makeatother
%% Journal information (used by PACMPL format)
%% Supplied to authors by publisher for camera-ready submission
\acmJournal{PACMPL}
\acmVolume{1}
\acmNumber{1}
\acmArticle{1}
\acmYear{2017}
\acmMonth{1}
\acmDOI{10.1145/nnnnnnn.nnnnnnn}
\startPage{1}
\else\makeatother
%% Conference information (used by SIGPLAN proceedings format)
%% Supplied to authors by publisher for camera-ready submission
\acmConference[PL'17]{ACM SIGPLAN Conference on Programming Languages}{January 01--03, 2017}{New York, NY, USA}
\acmYear{2017}
\acmISBN{978-x-xxxx-xxxx-x/YY/MM}
\acmDOI{10.1145/nnnnnnn.nnnnnnn}
\startPage{1}
\fi


%% Copyright information
%% Supplied to authors (based on authors' rights management selection;
%% see authors.acm.org) by publisher for camera-ready submission
\setcopyright{none}             %% For review submission
%\setcopyright{acmcopyright}
%\setcopyright{acmlicensed}
%\setcopyright{rightsretained}
%\copyrightyear{2017}           %% If different from \acmYear


%% Bibliography style
\bibliographystyle{ACM-Reference-Format}
%% Citation style
%% Note: author/year citations are required for papers published as an
%% issue of PACMPL.
%\citestyle{acmauthoryear}  %% For author/year citations
%\citestyle{acmnumeric}     %% For numeric citations
%\setcitestyle{nosort}      %% With 'acmnumeric', to disable automatic
                            %% sorting of references within a single citation;
                            %% e.g., \cite{Smith99,Carpenter05,Baker12}
                            %% rendered as [14,5,2] rather than [2,5,14].
%\setcitesyle{nocompress}   %% With 'acmnumeric', to disable automatic
                            %% compression of sequential references within a
                            %% single citation;
                            %% e.g., \cite{Baker12,Baker14,Baker16}
                            %% rendered as [2,3,4] rather than [2-4].

\begin{document}

\title{Definitional Abstract Interpreters for Higher-Order Programming Languages}
%\title{Abstracting Definitional Interpreters} % for Higher-Order Programming Languages}
\subtitle{Functional Pearl}

\author{David Darais}
\orcid{nnnn-nnnn-nnnn-nnnn}             %% \orcid is optional
\affiliation{
  %\position{Position1}
  %\department{Department1}              %% \department is recommended
  \institution{University of Maryland}            %% \institution is required
  %\streetaddress{Street1 Address1}
  \city{College Park}
  \state{Maryland}
  %\postcode{Post-Code1}
  %\country{Country1}
}
\email{darais@cs.umd.edu}          %% \email is recommended

\author{Nicholas Labich}
\orcid{nnnn-nnnn-nnnn-nnnn}             %% \orcid is optional
\affiliation{
  %\position{Position1}
  %\department{Department1}              %% \department is recommended
  \institution{University of Maryland}            %% \institution is required
  %\streetaddress{Street1 Address1}
  \city{College Park}
  \state{Maryland}
  %\postcode{Post-Code1}
  %\country{Country1}
}
\email{labichn@cs.umd.edu}          %% \email is recommended

\author{Phúc C. Nguyễn}
\orcid{nnnn-nnnn-nnnn-nnnn}             %% \orcid is optional
\affiliation{
  %\position{Position1}
  %\department{Department1}              %% \department is recommended
  \institution{University of Maryland}            %% \institution is required
  %\streetaddress{Street1 Address1}
  \city{College Park}
  \state{Maryland}
  %\postcode{Post-Code1}
  %\country{Country1}
}
\email{pcn@cs.umd.edu}          %% \email is recommended


\author{David {Van Horn}}
\orcid{nnnn-nnnn-nnnn-nnnn}             %% \orcid is optional
\affiliation{
  %\position{Position1}
  %\department{Department1}              %% \department is recommended
  \institution{University of Maryland}            %% \institution is required
  %\streetaddress{Street1 Address1}
  \city{College Park}
  \state{Maryland}
  %\postcode{Post-Code1}
  %\country{Country1}
}
\email{dvanhorn@cs.umd.edu}          %% \email is recommended

%% \authorinfo{Phúc C. Nguyễn}
%%            {University of Maryland}
%%            {pcn@cs.umd.edu}

\begin{abstract}
  In this functional pearl, we examine the use of definitional
  interpreters as a basis for abstract interpretation of higher-order
  programming languages.  As it turns out, definitional interpreters,
  especially those written in a monadic style, can provide a nice
  basis for a wide variety of collecting semantics, abstract
  interpretations, symbolic executions, and their intermixings.


  But the real insight of this story is a kind of replaying of an
  insight contained in Reynold's landmark paper, \emph{Definitional
    Interpreters for Higher-Order Programming Languages}, in which he
  observes definitional interpreters open the possibility of the
  defined-language inheriting properties of the defining-language.  We
  show the same holds true for definitional \emph{abstract}
  interpreters.  Remarkably, we observe that abstract definitional
  interpreters inherit the so-called ``pushdown control flow''
  property, wherein function calls and returns are precisely matched
  in the abstract semantics, simply by virtue of the function call
  mechanism of the defining-language.

  Past approaches first achieved this property for higher-order
  languages within the last ten years and have since been the subject
  of many papers.  These approaches start from a state-machine
  semantics and uniformly involve significant technical engineering to
  recover the precision of pushdown control flow.  In contrast,
  starting from a definitional interpreter, the pushdown control flow
  property is inherent in the meta-language and requires no further
  technical mechanism to achieve.
\end{abstract}

\maketitle

%\category{CR-number}{subcategory}{third-level}
%\terms term1, term2
%\keywords keyword1, keyword2

\section{Introduction}

An abstract interpreter is intended to soundly and effectively compute
an over-approximation to its concrete counterpart.  For higher-order
languages, these concrete interpreters tend to be formulated as
state-machines~(e.g.~\cite{dvanhorn:jagannathan-weeks-popl95,
  dvanhorn:jagannathan-etal-popl98, 
  dvanhorn:wright-jagannathan-toplas98,
  dvanhorn:Might:2006:DeltaCFA,
  dvanhorn:midtgaard-jensen-sas-08,
  dvanhorn:Midtgaard2009Controlflow,
  dvanhorn:Might2011Family, dvanhorn:Sergey2013Monadic}).  There are
several reasons for this choice: they operate with simple transfer
functions defined over similarly simple data structures, they make
explicit all aspects of the state of a computation, and computing
fixed-points in the set of reachable states is straightforward.
%
The essence of the state-machine based approach was distilled by Van
Horn and Might in their ``abstracting abstract machines'' (AAM)
technique, which provides a systematic method for constructing
abstract interpreters from standard abstract machines like the CEK- or
Krivine-machines \cite{dvanhorn:VanHorn2010Abstracting}.  Language
designers who would like to build abstract interpreters and program
analysis tools for their language can now, in principle at least,
first build a state-machine interpreter and then turn the crank to
construct the approximating abstract counterpart.

A natural pair of questions that arise from this past work is to
wonder:
\begin{enumerate}
  \item can a systematic abstraction technique similar to AAM  be
carried out for interpreters written, \emph{not} as state-machines,
but instead as high-level definitional interpreters, i.e. recursive,
compositional evaluators?
\item  is such a perspective fruitful?
\end{enumerate}
In this functional pearl we seek to answer both questions in the
affirmative.

For the first question, we show the AAM recipe can be applied
to definitional interpreters in a straightforward adaptation of the
original method. The primary technical challenge in this new setting
is handling interpreter fixpoints in a way that is both sound and
always terminates---a naive abstraction of fixpoints will be sound but
isn't always terminating, and a naive use of caching for fixpoints
will guarantee termination but is inherently unsound. We address this
technical challenge with a straightforward caching fixpoint-finding
algorithm which is both sound guaranteed to terminate when abstracting
arbitrary definitional interpreters.

For the second question, we claim that the abstract definitional
interpreter perspective is fruitful in two regards.  The first is
unsurprising: high-level abstract interpreters offer the usual
beneficial properties of their concrete counterparts in terms of being
re-usable and extensible.  In particular, we show that abstract
interpreters can be structured with monad transformers to good effect.
The second regard is more surprising, and we consider its observation
to be the main contribution of this pearl.

Definitional interpreters, in contrast to abstract machines, can leave
aspects of computation implicit, relying on the semantics of the
defin\emph{ing}-language to define the semantics of the
defin\emph{ed}-language, an observation made by Reynolds in his landmark
paper, \emph{Definitional Interpreters for Higher-order Programming
  Languages}~\cite{dvanhorn:reynolds-acm72}.  For example, Reynolds showed it is
possible to write a definitional interpreter such that it defines a
call-by-value language when the metalanguage is call-by-value, and
defines a call-by-name language when the metalanguage is call-by-name.
Inspired by Reynolds, we show that \emph{abstract} definitional interpreters can likewise
inherit properties of the metalanguage.  In particular we construct an
abstract definitional interpreter where there is no explicit
representation of continuations or a call stack.  Instead the
interpreter is written in a straightforward recursive style, and the
call stack is implicitly handled by the metalangauge.  What emerges
from this construction is a total abstract evaluation function that
soundly approximates all possible concrete executions of a given
program.  But remarkably, since the abstract evaluator relies on the
metalanguage to manage the call stack implicitly, it is easy to
observe that it introduces no approximation in the matching of calls
and returns, and therefore implements a ``pushdown'' analysis, all
without the need for any explicit machinery to do so.

\subsection*{Outline}

In the remainder of this pearl, we present an adaptation of the AAM
method to the setting of recursively-defined, compositional evaluation
functions, a.k.a.~definitional interpreters.  We first briefly review
the basic ingredients in the AAM recipe (\S\ref{s:aam}) and then
define our definitional interpreter (\S\ref{s:interp}).  The
interpreter is largely standard, but is written in a monadic and
extensible style, so as to be re-usable for various forms of semantics
we examine.  The AAM technique applies in a basically straightforward
way by store-allocating bindings and soundly finitizing the heap.  But
when naively run, the interpreter will not always terminate.  To solve
this problem we introduce a caching strategy and a simple fixed-point
computation to ensure the interpreter terminates (\S\ref{s:cache}).
It is at this point that we observe the interpreter we have built
enjoys the ``pushdown'' property \emph{\`a la} Reynolds---it is
inherited from the defining language of our interpreter and requires
no explicit mechanism (\S\ref{s:reynolds}).

Having established the main results, we then explore some variations
in brief vignettes that showcase the flexibility of our definitional
abstract interpreter approach.  First we consider the widely used
technique of so-called ``store-widening,'' which trades precision for
efficiency by modelling the abstract store globally instead of locally
(\S\ref{s:widening}).  Thanks to our monadic formulation of the
interpreter, this is achieved by a simple re-ordering of the monad
transformer stack.  We also explore some alternative abstractions,
showing that due to the extensible construction, it's easy to
experiment with alternative components for the abstract interpreter.
In particular, we define an alternative interpretation of the primitive
operations that remains completely precise until forced by joins in
the store to introduce approximation (\S\ref{s:alt-abstraction}).  As
another variation, we explore computing a form of symbolic execution
as yet another instance of our interpreter (\S\ref{s:symbolic}).  Lastly, we
show how to incorporate so-called ``abstract garbage collection,'' a
well-known technique for improving the precision of abstract
interpretation by clearing out unreachable store locations, thus
avoiding future joins which cause imprecision (\S\ref{s:gc}).  This
last variation is significant because it demonstrates that even though
we have no explicit representation of the stack, it is possible to
compute analyses that typically require such explicit representations
in order to calculate root sets for garbage collection.

Finally, we place our work in the context of the prior literature on
higher-order abstract interpretation (\S\ref{s:related-work}) and draw
some conclusions (\S\ref{s:conclusion}).



%% In his landmark paper, \emph{Definitional Interpreters for Higher-order
%% Programming Languages}~\cite{dvanhorn:reynolds-acm72}, Reynolds observed that
%% when a programming language is defined by way of an interpreter, it is possible
%% to inherit semantic characteristics of the defining metalanguage. For example,
%% it is possible to define a definitional interpreter which is call-by-value when
%% the metalanguage is call-by-value, and call-by-name when the metalanguage is
%% call-by-name.

%% We expand on Reynolds's observation in the setting of abstract interpreters and
%% discover the following:
%% \begin{itemize}
%% \item Definitional interpreters, written in monadic style, can simultaneously
%%   define a language's semantics as well as safe approximations of those
%%   semantics, \emph{i.e.} abstract interpreters.
%% \item These definitional \emph{abstract} interpreters can inherit
%%   characteristics of the defining language.  In particular, precise
%%   call-and-return matching can be inherited, yielding a pushdown
%%   analysis~\cite{dvanhorn:Earl2010Pushdown,local:vardoulakis-diss12}.
%% \end{itemize}
%% Furthermore, we contribute:
%% \begin{itemize}
%% \item A systematic methodology for designing abstract interpreters via
%%   definitional interpreters; and
%% \item A soundness framework establishing the correctness of abstract
%%   interpreters defined in definitional style.
%% \end{itemize}

%% \paragraph{A Compositional Approach to Program Analysis}
%% There are many approaches to designing program analyzers for programming
%% languages. For \emph{higher-order} programming languages, two popular
%% approaches are to use abstract machines~\cite{dvanhorn:VanHorn2010Abstracting}
%% and constraint systems~\cite{dvanhorn:Neilson:1999}. Other foundations exist
%% for designing program analyzers, but no approach to-date utilizes big-step
%% operational semantics or definitional interpreters for program analysis of
%% higher-order languages. This is unfortunate because big-step semantics and
%% definitional interpreters are more compositional and high-level than state
%% machines or constraint systems. Big-step and denotational approaches exist for
%% first-order programming languages; the gap is in extending these ideas to the
%% higher-order setting.

%% We bridge this gap, providing the first framework for program analysis of
%% higher-order languages which builds upon big-step semantics and their
%% corresponding definitional interpreters, which are compositional by nature. 

%% Our key insights are to design definitional interpreters in monadic,
%% open-recursive style (§~\ref{s:interp}), and to design a
%% novel fixpoint algorithm tailored specifically to the setting of higher-order
%% definitional interpreters (§~\ref{s:cache}). The extensible nature of the
%% interpreter allows us to recover a wide-range of analyses through its
%% instantiation, including widening techniques (§~\ref{s:widening}), precision
%% preserving abstractions (§~\ref{s:alt-abstraction}), and symbolic execution for
%% program verification (§~\ref{s:symbolic}). Our implementation is freely
%% available through a suite of embedded languages in Racket (§~\ref{s:try-it}).
%% Finally, we prove the approach sound w.r.t. a derived big-step collecting and
%% abstract semantics (§~\ref{s:formalism}), where the key insight in the
%% formalism is to model not only standard big-step \emph{evaluation} relations,
%% but also big-step \emph{reachability} relations.

%% \paragraph{Pushdown Precision in Program Analysis}
%% A common problem with traditional approaches to control flow analysis is the
%% inability to properly match a function call with its return in the abstract
%% semantics. This leads to infeasible program (abstract) executions in which a
%% call is made from one point in the program, but control returns to another.
%% The PDCFA analysis of Earl \emph{et al}~\cite{dvanhorn:Earl2010Pushdown} and
%% CFA2 analysis of Vardoulakis and Shivers~\cite{dvanhorn:Vardoulakis2011CFA2}
%% were the first approaches to overcome this limitation in the higher-order
%% setting. In essence, these analyses replaces the traditional finite automata
%% abstractions of programs with pushdown automata, an approach pioneered by Reps
%% \emph{et al}~\cite{dvanhorn:Reps1995Precise} in the first-order setting.

%% We realize a new technique for defining abstract interpreters with pushdown
%% precision, meaning the analysis precisely matches function calls to returns. In
%% the setting of definitional interpreters, this property is inherited from the
%% defining metalanguage and requires no instrumentation to the analysis \emph{at
%% all} (§~\ref{s:reynolds}).

%% A technical difference between small-step and big-step approaches to semantics
%% are that small-step methods must model the execution context of evaluation,
%% whether through evaluation contexts~\cite{local:felleisen-TCS1992} or stack
%% frames~\cite{dvanhorn:Felleisen1987Calculus}. In big-step methods, there is no
%% model for the context; it is implicit in the definition of evaluation rules, or
%% in the case of definitional interpreters, implicit in the call-and-return
%% semantics of the defining programming language. Because a defining programming
%% language is precise in its call/return behavior, so is a definitional abstract
%% interpreter embedded within it.

\section{From Machines to Compositional Evaluators}
\label{s:aam}

In recent years, there has been considerable effort in the systematic
construction of abstract interpreters for higher-order languages using abstract
machines---first-order transition systems---as a semantic basis.  The so-called
\emph{Abstracting Abstract Machines} (AAM) approach to abstract
interpretation~\cite{dvanhorn:VanHorn2010Abstracting} is a recipe for
transforming a machine semantics into an easily abstractable form. The
transformation includes the following ingredients:
\begin{itemize}
\item Allocating continuations in the store;
\item Allocating variable bindings in the store;
\item Using a store that maps addresses to \emph{sets} of values;
\item Interpreting store updates as a join; and
\item Interpreting store dereference as a non-deterministic choice.
\end{itemize}
These transformations are semantics-preserving due to the original and derived
machines operating in a lock-step correspondence.  After transforming the
semantics in this way, a \emph{computable} abstract interpreter is achieved by:
\begin{itemize}
\item Bounding store allocation to a finite set of addresses; and
\item Widening base values to some abstract domain.
\end{itemize}
After performing these transformations, the soundness and computability of the
resulting abstract interpreter are then self-evident and easily proved.

The AAM approach has been applied to a wide variety of languages and
applications, and given the success of the approach it's natural to wonder what
is essential about its use of low-level machines. It is not at all clear
whether a similar approach is possible with a higher-level formulation of the
semantics, such as a compositional evaluation function defined recursively over
the syntax of expressions.

This paper shows that the essence of the AAM approach can be applied to a
high-level semantic basis.  We show that compositional evaluators written in
monadic style can express similar abstractions to that of AAM, and like AAM,
the design remains systematic.  Moreover, we show that the high-level semantics
offers a number of benefits not available to the machine model.  

\begin{figure} %{{{ f:syntax
\begin{mdframed}
\begin{alignat*}{4}
   e ∈ &&\mathrel{}   exp ⩴ &\mathrel{} 𝔥⸨(vbl⸩\ x𝔥⸨)⸩         &\hspace{3em} [⦑\emph{variable}⦒]
\\[\mathgobble]     &&\mathrel{}       ∣ &\mathrel{} 𝔥⸨(num⸩\ n𝔥⸨)⸩         &\hspace{3em} [⦑\emph{number}⦒]
\\[\mathgobble]     &&\mathrel{}       ∣ &\mathrel{} 𝔥⸨(lam⸩\ x\ e𝔥⸨)⸩      &\hspace{3em} [⦑\emph{lambda}⦒]
\\[\mathgobble]     &&\mathrel{}       ∣ &\mathrel{} 𝔥⸨(if0⸩\ e\ e\ e𝔥⸨)⸩   &\hspace{3em} [⦑\emph{conditional}⦒]
\\[\mathgobble]     &&\mathrel{}       ∣ &\mathrel{} 𝔥⸨(op2⸩\ b\ e\ e𝔥⸨)⸩   &\hspace{3em} [⦑\emph{binary op}⦒]
\\[\mathgobble]     &&\mathrel{}       ∣ &\mathrel{} 𝔥⸨(app⸩\ e\ e𝔥⸨)⸩      &\hspace{3em} [⦑\emph{application}⦒]
\\[\mathgobble]     &&\mathrel{}       ∣ &\mathrel{} 𝔥⸨(rec⸩\ x\ e\ e𝔥⸨)⸩   &\hspace{3em} [⦑\emph{letrec}⦒]
\\[\mathgobble] x ∈ &&\mathrel{}   var ≔ &\mathrel{} ❴𝔥⸨x⸩, 𝔥⸨y⸩, …❵        &\hspace{3em} [⦑\emph{variable names}⦒]
%\\[\mathgobble] u ∈ &&\mathrel{}  unop ≔ &\mathrel{} ❴𝔥⸨add1⸩, …❵           &\hspace{3em} [⦑\emph{unary prim}⦒]
\\[\mathgobble] b ∈ &&\mathrel{} binop ≔ &\mathrel{} ❴𝔥⸨+⸩, 𝔥⸨-⸩, …❵        &\hspace{3em} [⦑\emph{binary prim}⦒]
\end{alignat*}
\captionskip{Programming Language Syntax}
\label{f:syntax}
\end{mdframed}
\end{figure} %}}}

There is a rich body of work concerning tools and techniques for
\emph{extensible} interpreters~\cite{dvanhorn:Liang1995Monad,
  local:jaskelioff2009lifting, local:kiselyov2012typed}, all of which
applies to high-level semantics.  By putting abstract interpretation
for higher-order languages on a high-level semantic basis, we can
bring these results to bear on the construction of extensible abstract
interpreters.

\section{A Definitional Interpreter}\label{s:interp}

We begin by constructing a definitional interpreter for a small but
representative higher-order, functional language.  As our defining language, we
use an applicative subset of Racket, a dialect of Scheme.\footnote{This choice
is largely immaterial: any functional language would do.} The abstract syntax
of the language is defined in Figure~\ref{f:syntax}; it includes variables,
numbers, unary and binary operations on numbers, conditionals, {\tt letrec}
expressions, functions and applications.

\begin{figure} %{-{
\rfloat{⸨ev@⸩}
\begin{lstlisting}
¦ (define ((ev ev) e)
¦   (match e
¦     [(num n) (return n)]
¦     [(vbl x)
¦      (do ρ ← ask-env
¦          (find (ρ x)))]    
¦     [(ifz e₀ e₁ e₂) 
¦      (do v  ← (ev e₀)
¦          z? ← (zero? v)
¦          (ev (if z? e₁ e₂)))]
¦     [(op1 o e₀)
¦      (do v ← (ev e₀)
¦          (δ o v))]   
¦     [(op2 o e₀ e₁)
¦      (do v₀ ← (ev e₀)
¦          v₁ ← (ev e₁)
¦          (δ o v₀ v₁))]
¦     [(lrc f e₀ e₁) 
¦      (do ρ  ← ask-env
¦          a  ← (alloc f)
¦          ρ′ ≔ (ρ f a)
¦          (ext a (cons e₀ ρ′))
¦          (local-env ρ′
¦            (ev e₁)))]
¦     [(lam x e₀)
¦      (do ρ ← ask-env
¦          (return (cons (lam x e₀) ρ)))]
¦     [(app e₀ e₁)
¦      (do (cons (lam x e₂) ρ) ← (ev e₀)
¦          v₁ ← (ev e₁)
¦          a  ← (alloc x)         
¦          (ext a v₁)
¦          (local-env (ρ x a) 
¦            (ev e₂)))]))
\end{lstlisting}
\caption{The Extensible Definitional Interpreter}
\label{f:interpreter}
\end{figure} %}-}

The interpreter for the language is defined in Figure~\ref{f:interpreter}. At
first glance, it has many conventional aspects:
\begin{itemize}
\item It is compositionally defined by structural recursion on the syntax of
expressions.
\item It represents function values as a closure data structure which pairs the
function body with the evaluation environment.
\item It is structured monadically and uses monad operations to interact with
the environment and store.
\item It relies on a helper function ⸨δ⸩ to interpret primitive operations.
\end{itemize}
There are a few superficial aspects that deserve a quick note:
environments ⸨ρ⸩ are finite maps and ⸨(ρ«\ »x)⸩ denotes
«ρ(x)» while ⸨(ρ«\ »x a)⸩ denotes «ρ[x↦a]».  The
⸨do⸩-notation is just shorthand for ⸨bind⸩, as usual:

% LAYOUT
\begin{alignat*}{2}
   𝔥⸨(do x ←«\ »e . r)⸩ ≡ &\mathrel{} 𝔥⸨(bind e (λ«\ »(x) (do . r)))⸩
\\        𝔥⸨(do e . r)⸩ ≡ &\mathrel{} 𝔥⸨(bind e (λ«\ »(_) (do . r)))⸩
\\ 𝔥⸨(do x ≔«\ »e . r)⸩ ≡ &\mathrel{} 𝔥⸨(let ((x e)) (do . r))⸩
\\            𝔥⸨(do b)⸩ ≡ &\mathrel{} 𝔥⸨b⸩
\end{alignat*}
Finally, there are two unconventional aspects worth noting.

First, the interpreter is written in an \emph{open recursive style}; the
evaluator does not call itself recursively, instead it takes as an argument a
function ⸨ev⸩—shadowing the name of the function ⸨ev⸩ being defined—and ⸨ev⸩
(the argument) is called instead of self-recursion.  This is a standard
encoding for recursive functions in a setting without recursive binding.  It is
up to an external function, such as the Y-combinator, to close the recursive
loop.  This open recursive form is crucial for our purposes in that it allows
intercepting recursive calls to perform “deep” instrumentation of the
interpreter.

Second, the code is clearly \emph{incomplete}.  There are a number of free
variables, noted in italics, which must implement the following:
\begin{itemize}
\item The underlying monad of the interpreter: ⸨return⸩ and ⸨bind⸩;
\item An interpretation of primitives: ⸨δ⸩ and ⸨zero?⸩;
\item Environment operations: ⸨ask-env⸩ for retrieving the
environment and ⸨local-env⸩ for installing an environment;
\item Store operations: ⸨ext⸩ for updating the store, and ⸨find⸩ for
dereferencing locations; and
\item An operation for ⸨alloc⸩ating new store locations.
\end{itemize}
Going forward, we make frequent use of definitions involving free variables,
and we call such a collection of such definitions a \emph{component}. We assume
components can be named (in this case, we've named the component ⸨ev@⸩,
indicated by the box in the upper-right corner) and linked together to
eliminate free variables.\footnote{We use Racket
\emph{units}~\cite{local:flatt-pldi98} to model components in our
implementation.}

\begin{figure} %{-{
\rfloat{⸨monad@⸩}
\begin{alignat*}{1}
          & 𝔥⸨(define-monad⸩
\\[-0.5em]& ␣␣𝔥⸨(⸩\!\up{𝔥⸨ReaderT⸩}⸢env⸣\ 𝔥⸨(⸩\!\up{𝔥⸨FailT⸩}⸢errors⸣\ 𝔥⸨(⸩\!\up{𝔥⸨StateT⸩}⸢store⸣\ 𝔥⸨ID))))⸩
\end{alignat*}
\figskip\rfloat{⸨δ@⸩}
\begin{lstlisting}
¦ (define (δ . ovs)
¦   (match ovs
¦     [(list 'add1 n)  (return (add1 n))]
¦     [(list 'sub1 n)  (return (sub1 n))]
¦     [(list '- n)     (return (- n))]
¦     [(list '+ n₀ n₁) (return (+ n₀ n₁))]
¦     [(list '- n₀ n₁) (return (- n₀ n₁))]
¦     [(list '* n₀ n₁) (return (* n₀ n₁))]
¦     [(list 'quotient n₀ n₁)
¦      (if (= 0 n₁)
¦          fail
¦          (return (quotient n₀ n₁)))]))
¦ (define (zero? v)
¦   (return (= 0 v)))
\end{lstlisting}
\figskip\rfloat{⸨store@⸩}
\begin{lstlisting}
¦ (define (find a)
¦   (do σ ← get-store
¦       (return (σ a))))
¦ (define (ext a v) 
¦   (update-store (λ (σ) (σ a v))))
\end{lstlisting}
\figskip\rfloat{⸨alloc@⸩}
\begin{lstlisting}
¦ (define (alloc x)
¦   (do σ ← get-store
¦       (return (size σ))))
\end{lstlisting}
\caption{Components for Definitional Interpreters}
\label{f:concrete-components}
\end{figure} %}-}

\subsection{Instantiating the Interpreter}

Next we examine a set of components which complete the definitional
interpreter, defined in Figure~\ref{f:concrete-components}. The first component
is ⸨monad@⸩, which uses our ⸨define-monad⸩ macro to generate a set of bindings
based on a monad transformer stack.  For this interpreter, we use a failure
monad to model divide-by-zero errors, a state monad to model the store, and a
reader monad to model the environment.  The ⸨define-monad⸩ form generates
bindings for ⸨return⸩, ⸨bind⸩, ⸨ask-env⸩, ⸨local-env⸩, ⸨get-store⸩ and
⸨update-store⸩, and their definitions are
standard~\cite{dvanhorn:Liang1995Monad}. 

We also add the ⸨mrun⸩ operation for running monadic computations, starting
with the empty environment and store ⸨∅⸩:
\begin{lstlisting}
¦ (define (mrun m)
¦   (run-StateT ∅ (run-ReaderT ∅ m)))
\end{lstlisting}
While the ⸨define-monad⸩ form is hiding some details, this component could have
equivalently been written out explicitly. For example, ⸨return⸩ and ⸨bind⸩ can
be defined as:
\begin{lstlisting}
¦ (define (((return a) r) s) (cons a s))
¦ (define (((bind ma f) r) s)
¦   (match ((ma r) s)
¦     [(cons a s′) (((f a) r) s′)]
¦     ['failure 'failure]))
\end{lstlisting}
The remaining operations are similarly straightforward.  So far our use of
monad transformers can be seen as a mere convenience, but the monad abstraction
will become essential for deriving new analyses later on.

The ⸨δ@⸩ component defines the interpretation of primitives, which is given in
terms of the underlying monad.  Finally the ⸨alloc@⸩ component provides a
definition of ⸨alloc⸩, which fetches the store and uses its size to return a
fresh address (assuming the invariant «𝔥⸨(∈«\ »a σ)⸩ ⇔ 𝔥⸨a⸩ < 𝔥⸨(size σ)⸩»).
The ⸨store@⸩ component defines ⸨find⸩ and ⸨ext⸩ for finding and extending the
store in terms of the monadic operations.

The only remaing pieces of the puzzle are a fixed-point combinator, which is
straightforward to define:
\begin{lstlisting}
¦ (define ((fix f) x) ((f (fix f)) x))
\end{lstlisting}
and the main entry-point for the interpreter:
\begin{lstlisting}
¦ (define (eval e) (mrun ((fix ev) e)))
\end{lstlisting}
By taking advantage of Racket's languages-as-libraries
features~\cite{dvanhorn:TobinHochstadt2011Languages}, we construct REPLs for
interacting with this interpreter.  Here are a few examples, which make use of
a concrete syntax for more succinctly writing expressions. The identity
function evaluates to an answer consisting of a closure over the empty
environment together with the empty store:
ℑ⁅
¦ > (λ (x) x)
ℑ,
¦ '(((λ (x) x) . ()) . ())
ℑ⁆
Here's an example showing a non-empty environment and store:
ℑ⁅
¦ > ((λ (x) (λ (y) x)) 4)
ℑ,
¦ '(((λ (y) x) . ((x . 0))) . ((0 . 4)))
ℑ⁆
Primitive operations work as expected:
ℑ⁅
¦ > (* (+ 3 4) 9)
ℑ,
¦ '(63 . ())
ℑ⁆
And divide-by-zero errors result in failures:
ℑ⁅
¦ > (quotient 5 (- 3 3))
ℑ,
¦ '(failure . ())
ℑ⁆
Because our monad stack places ⸨FailT⸩ above ⸨StateT⸩, the answer includes the
(empty) store at the point of the error. Had we changed ⸨monad@⸩ to use:
\begin{alignat*}{1}
& ␣␣𝔥⸨(⸩\!\up{𝔥⸨ReaderT⸩}⸢env⸣\ 𝔥⸨(⸩\!\up{𝔥⸨StateT⸩}⸢store⸣\ 𝔥⸨(⸩\!\up{𝔥⸨FailT⸩}⸢errors⸣\ 𝔥⸨ID⸩))))
\end{alignat*}
failures would not include the store:
ℑ⁅
¦ > (quotient 5 (- 3 3))
ℑ,
¦ 'failure
ℑ⁆
At this point we've defined a simple definitional interpreter, although the
extensible components involved—monadic operations and open recursion—will allow
us to instantiate the same interpreter to achieve a wide range of useful
abstract interpretations.

\subsection{Collecting Variations}\label{s:collecting}

The formal development of abstract interpretation often starts from a so-called
``non-standard collecting semantics.''  A common form of collecting semantics
is a trace semantics, which collects streams of states the interpreter reaches.
Figure~\ref{f:trace} shows the monad stack for a tracing interpreter and a
``mix-in'' for the evaluator.  The monad stack adds ⸨WriterT⸩ using ⸨List⸩,
which provides a new operation ⸨tell⸩ for writing items to the stream of
reached states.  The ⸨ev-trace⸩ function is a wrapper around an underlying
⸨ev₀⸩ unfixed evaluator, and interposes itself between each recursive call by
⸨tell⸩ing the current state of the evaluator, that is the current expression,
environment, and store.  The top-level evaluation function is then:
\begin{lstlisting}
¦ (define (eval e) 
¦   (mrun ((fix (ev-tell ev)) e)))
\end{lstlisting}

\begin{figure} %{-{
\rfloat{⸨trace-monad@⸩}
\begin{alignat*}{1}
          & 𝔥⸨(define-monad⸩
\\[-0.5em]& ␣␣𝔥⸨(⸩\!\up{𝔥⸨ReaderT⸩}⸢env⸣\ 𝔥⸨(⸩\!\up{𝔥⸨FailT⸩}⸢errors⸣\ 𝔥⸨(⸩\!\up{𝔥⸨StateT⸩}⸢store⸣\ 𝔥⸨(⸩\!\up{𝔥⸨WriterT List⸩}⸢traces⸣\ 𝔥⸨ID)))))⸩
\end{alignat*}
\figskip\rfloat{⸨ev-tell@⸩}
\begin{lstlisting}
¦ (define (((ev-tell ev₀) ev) e)
¦   (do ρ ← ask-env
¦       σ ← get-store
¦       (tell (list e ρ σ))
¦       ((ev₀ ev) e)))
\end{lstlisting}
\caption{Trace Collecting Semantics}
\label{f:trace}
\end{figure} %}-}

Now when an expression is evaluated, we get the resulting answer and a list of
all the states seen by the evaluator, in the order in which they were seen. For
example:
ℑ⁅
¦ > (* (+ 3 4) 9)
ℑ,
¦ '((63 . ())
¦   ((* (+ 3 4) 9)()())
¦   ((+ 3 4) () ())
¦   (3 () ())
¦   (4 () ())
¦   (9 () ()))
ℑ⁆
% ℑ⁅
% ¦ > ((λ (x) (λ (y) x)) 4)
% ℑ,
% ¦ '((((λ (y) x) . ((x . 0))) . ((0 . 4)))
% ¦   (((λ (x) (λ (y) x)) 4) () ())
% ¦   ((λ (x) (λ (y) x)) () ())
% ¦   (4 () ())
% ¦   ((λ (y) x) ((x . 0)) ((0 . 4))))
% ℑ⁆
Were we to swap ⸨List⸩ with ⸨Set⸩ in the monad stack, we would obtain a
\emph{reachable} state semantics, another common form of collecting semantics,
that loses the order and repetition of states.

As another collecting semantics variant, we show how to collect the \emph{dead
code} in a program.  Here we use a monad stack that has an additional state
component (with operations named ⸨put-dead⸩ and ⸨get-dead⸩) which stores the
set of dead expressions.  Initially this will contain all of the subexpressions
of the program.  As the interpreter recurs through expressions it will remove
them from the dead set.

Figure~\ref{f:dead} defines the monad stack for the dead code collecting
semantics and the ⸨ev-dead@⸩ component, another mix-in for an ⸨ev₀⸩ evaluator
to remove the given subexpression before recurring.  Since computing the dead
code requires an outer wrapper that sets the initial set of dead code to be all
of the subexpressions in the program, we define ⸨eval-dead@⸩ which consumes a
\emph{closed evaluator}, i.e. something of the form ⸨(fix ev)⸩.

Putting these pieces together, the dead code collecting semantics is defined:
\begin{lstlisting}
¦ (define (eval e)
¦   (mrun ((eval-dead (fix (ev-dead ev))) e)))
\end{lstlisting}
Running a program with the dead code interpreter produces an answer and the set
of expressions that were not evaluated during the running of a program:
ℑ⁅
¦ > (if0 0 1 2)
ℑ,
¦ (cons '(1 . ()) (set 2))
ℑ;
¦ > (* (+ 3 4) 9)
ℑ,
¦ (cons '(63 . ()) (set))
ℑ;
¦ > (λ (x) x)
ℑ,
¦ (cons '(((λ (x) x) . ()) . ()) (set 'x))
ℑ;
¦ > (if0 (quotient 1 0) 2 3)]
ℑ,
¦ (cons '(failure . ()) (set 3 2))
ℑ⁆

\begin{figure} %{-{
\rfloat{⸨dead-monad@⸩}
\begin{alignat*}{1}
          & 𝔥⸨(define-monad⸩
\\[-0.5em]& ␣␣𝔥⸨(⸩\!\up{𝔥⸨ReaderT⸩}⸢env⸣\ 𝔥⸨(⸩\!\up{𝔥⸨StateT⸩}⸢store⸣\ 𝔥⸨(⸩\!\up{𝔥⸨StateT⸩}⸢dead⸣\ 𝔥⸨(⸩\!\up{𝔥⸨FailT⸩}⸢errors⸣\ 𝔥⸨ID)))))⸩
\end{alignat*}
\figskip\rfloat{⸨ev-dead@⸩}
\begin{lstlisting}
¦ (define (((ev-dead ev₀) ev) e)
¦   (do θ  ← get-dead       
¦       (put-dead (set-remove θ e))
¦       ((ev₀ ev) e)))
\end{lstlisting}
\figskip\rfloat{⸨eval-dead@⸩}
\begin{lstlisting}
¦ (define ((eval-dead eval) e₀)
¦   (do (put-dead (subexps e₀))
¦       (eval e₀)))
\end{lstlisting}
\caption{Dead Code Collecting Semantics}
\label{f:dead}
\end{figure} %}-}

Our setup makes it easy not only to express the concrete interpreter, but also
different forms of collecting semantics. Let us now look at abstractions.

\subsection{Abstracting Base Values}\label{s:base}

Our interpreter must become decidable before it can be considered an analysis,
and the first step towards decidability is to abstract the base types of the
language to something finite. We do this for our number base type by
introducing a new \emph{abstract} number, written ⸨'N⸩, which represents the
set of all numbers. Abstract numbers are introduced by an alternative
interpretation of primitive operations, given in Figure~\ref{f:abs-delta},
which simply produces ⸨'N⸩ in all cases. 

Some care must be taken in the abstract interpretation of ⸨'quotient⸩. If the
denominator is the abstract number ⸨'N⸩, then it is possible the program could
fail as a result of divide-by-zero, since ⸨0⸩ is contained in the
representation of ⸨'N⸩. Therefore there are \emph{two} possible answers when
the denominator is ⸨'N⸩: ⸨'N⸩ and ⸨'failure⸩. Both answers are ⸨return⸩ed by
introducing non-determinism ⸨NondetT⸩ into the monad stack:
\begin{alignat*}{1}
& 𝔥⸨(⸩\!\up{𝔥⸨ReaderT⸩}⸢env⸣\ 𝔥⸨(⸩\!\up{𝔥⸨FailT⸩}⸢errors⸣\ 𝔥⸨(⸩\!\up{𝔥⸨StateT⸩}⸢store⸣𝔥\ 𝔥⸨(⸩\!\up{𝔥⸨NondetT⸩}⸢branching⸣\ 𝔥⸨ID))))⸩
\end{alignat*}
Adding non-determinism provides the ⸨mplus⸩ operation for combining multiple
answers. Non-determinism is also used in the implementation of ⸨zero?⸩, which
returns both true and false on ⸨'N⸩.

\begin{figure} %{-{
\rfloat{⸨δ^@⸩}
\begin{lstlisting}
¦ (define (δ . ovs)
¦   (match ovs
¦     [(list 'add1 n)  (return 'N)]
¦     [(list 'sub1 n)  (return 'N)]
¦     [(list '+ n₀ n₁) (return 'N)]
¦     [(list '- n₀ n₁) (return 'N)]
¦     [(list '* n₀ n₁) (return 'N)]
¦     [(list 'quotient n₀ (? number? n₁))
¦      (if (= 0 n₁) fail (return 'N))]
¦     [(list 'quotient n₀ 'N)
¦      (mplus fail (return 'N))]))
¦ (define (zero? v)
¦   (match v
¦     ['N (mplus (return #t) (return #f))]
¦     [_  (return (= 0 v))]))
\end{lstlisting}
\caption{Abstracting Primitive Operations}
\label{f:abs-delta}
\end{figure} %}-}

By linking together ⸨δ^@⸩ and the monad stack with non-determinism, we obtain
an evaluator that produces a set of results:
ℑ⁅
¦ > (* (+ 3 4) 9)
ℑ,
¦ '((N . ()))
ℑ;
¦ > (quotient 5 (add1 2))
ℑ,
¦ '((failure . ()) (N . ()))
ℑ;
¦ > (if0 (add1 0) 3 4)
ℑ,
¦ '((3 . ()) (4 . ()))
ℑ⁆

If we link ⸨δ^@⸩ with the \emph{tracing} monad stack plus non-determinism:
\begin{alignat*}{1}
  & 𝔥⸨(⸩\!\up{𝔥⸨ReaderT⸩}⸢env⸣𝔥⸨(⸩\!\up{𝔥⸨FailT⸩}⸢errors⸣𝔥⸨(⸩\!\up{𝔥⸨StateT⸩}⸢store⸣
    𝔥⸨(⸩\!\up{𝔥⸨WriterT List⸩}⸢traces⸣𝔥⸨(⸩\!\up{𝔥⸨NondetT⸩}⸢branching⸣\ 𝔥⸨ID)⸩\!…\!𝔥⸨)⸩
\end{alignat*}
we get an evaluator that produces sets of traces:
ℑ⁅
¦ > (if0 (add1 0) 3 4)
ℑ,
¦ (set
¦  '((3 . ())
¦    ((if0 (add1 0) 3 4) () ())
¦    ((add1 0) () ())
¦    (0 () ())
¦    (3 () ()))
¦  '((4 . ())
¦    ((if0 (add1 0) 3 4) () ())
¦    ((add1 0) () ())
¦    (0 () ())
¦    (4 () ())))
ℑ⁆

It should be clear that the interpreter will only ever see a finite set of
numbers (including ⸨'N⸩), but it's definitely not true that the interpreter
halts on all inputs.  First, it's still possible to generate an infinite number
of closures.  Second, there's no way for the interpreter to detect when it sees
a loop.  To make a terminating abstract interpreter requires tackling both.  We
look next at abstracting closures.

\subsection{Abstracting Closures}

Closures consist of code---a lambda term---and an environment---a finite map
from variables to addresses.  Since the set of lambda terms and variables is
bounded by the program text, it suffices to abstract closures by abstracting
the set of addresses.  Following the AAM approach, we can do this by modifying
the allocation function to always produce elements drawn from a finite set.  In
order to retain soundness in the semantics, we will need to modify the store to
map addresses to \emph{sets} of values and model store update as a join and
dereference as a non-deterministic choice.

Any abstraction of the allocation function that produces a finite set will do,
but the choice of abstraction will determine the precision of the resulting
analysis.  A simple choice is to allocate variable bindings by using a
variable's name as its address.  This gives a monomorphic, or 0CFA-like,
abstraction.

Figure~\ref{f:0cfa-abs} shows an alternative component for finite
allocation that uses variables names as the notion of addresses and a
component for the derived operations ⸨find⸩ and ⸨ext⸩
when the store uses a \emph{set} as its range.  The
⸨for/monad+⸩ form is just a convenience for combining a set of
computations with ⸨mplus⸩; in other words, ⸨find⸩
returns \emph{all} of the values in the store at a given address.  The
⸨ext⸩ function joins whenever an address is already allocated,
otherwise it maps the address to a singleton set.

\begin{figure}
\rfloat{⸨alloc^@⸩}
\begin{lstlisting}
(define (alloc x)
  (return x))
\end{lstlisting}
\figskip\rfloat{⸨store-nd@⸩}
\begin{lstlisting}
¦ (define (find a)
¦   (do σ ← get-store
¦       (for/monad+ ([v (σ a)])
¦         (return v))))
¦ (define (ext a v)
¦   (update-store
¦     (λ (σ) (σ a (if (∈ a σ) 
¦                     (set-add (σ a) v) 
¦                     (set v))))))
\end{lstlisting}
\caption{Abstracting Allocation: 0CFA}
\label{f:0cfa-abs}
\end{figure}

By linking these components together with the same monad stack from
Section~\ref{s:base}, we obtain an interpreter that loses precision whenever
variables are bound to multiple values.  For example, this program
binds ⸨x⸩ to both ⸨0⸩ and ⸨1⸩ and therefore
produces both answers when run:
ℑ⁅
¦ > (let f (λ (x) x)
¦     (let _ (f 0) (f 1)))]
ℑ,
¦ '((0 . ((x 1 0) (f ((λ (x) x) . ()))))
¦   (1 . ((x 1 0) (f ((λ (x) x) . ())))))
ℑ⁆

We've now taken care of making a sound, finite abstraction of the
space of all closures that arise during evaluation.  It would seem we
are very close to having a sound, total abstract interpretation
function.


\section{Caching and Finding Fixed-points}\label{s:cache}

At this point, the interpreter obtained by linking together ⸨monad^@⸩, ⸨δ^@⸩,
⸨alloc^@⸩ and ⸨store-nd@⸩ components will only ever visit a finite number of
configurations for a given program. A configuration (⸨ς⸩) consists of an
expression (⸨e⸩), environment (⸨ρ⸩) and store (⸨σ⸩). This configuration is
finite because: expressions are finite in the given program; environments are
maps from variables (again, finite in the program) to addresses; the addresses
are finite thanks to ⸨alloc^⸩; the store maps addresses to sets of values; base
values are abstracted to a finite set by ⸨δ^⸩; and closures consist of an
expression and environment, which are both finite.

Although the interpreter will only ever see a finite set of inputs, it
\emph{doesn't know it}.  A simple loop will cause the interpreter to diverge:
ℑ⁅
¦ > (rec f (λ (x) (f x)) (f 0))
ℑ,
¦ timeout
ℑ⁆
To solve this problem, we introduce a \emph{cache} (⸨¢⸢in⸣⸩) as input to the
algorithm, which maps from configurations (⸨ς⸩) to sets of value-and-store
pairs (⸨v×σ⸩). When a configuration is reached for the second time, rather than
re-evaluating the expression and entering an infinite loop, the result is
looked up from ⸨¢⸢in⸣⸩, which acts as an oracle. It is important that the cache
is used co-inductively: it is only safe to use ⸨¢⸢in⸣⸩ as an oracle so long as
some progress has been made first. 

The results of evaluation are then stored in an output cache (⸨¢⸢out⸣⸩), which
after the end of evaluation is “more defined” than the input cache (⸨¢⸢in⸣⸩),
again following a co-inductive argument. The least fixed-point ⸨¢⁺⸩ of an
evaluator which transforms an oracle ⸨¢⸢in⸣⸩ and outputs a more defined oracle
⸨¢⸢out⸣⸩ is then a sound approximation of the program, because it
over-approximates all finite unrollings of the unfixed evaluator. 

The co-inductive caching algorithm is shown in Figure~\ref{f:caching}, along
with the monad transformer stack ⸨monad-cache@⸩ which has two new components:
⸨ReaderT⸩ for the input cache ⸨¢⸢in⸣⸩, and ⸨StateT+⸩ for the output cache
⸨¢⸢out⸣⸩. We use a ⸨StateT+⸩ instead of ⸨WriterT⸩ monad transformer in the
output cache so it can double as tracking the set of seen states. The ⸨+⸩ in
⸨StateT+⸩ signifies that caches for multiple non-deterministic branches will be
merged automatically, producing a set of results and a single cache, rather
than a set of results paired with individual caches.

\begin{figure} %{-{
\begin{mdframed}
\rfloat{⸨monad-cache@⸩}
\begin{flalign*}
& 𝔥⸨(define-monad (⸩\!\up{𝔥⸨ReaderT⸩}⸢env⸣\ 𝔥⸨(⸩\!\up{𝔥⸨FailT⸩}⸢errors⸣\ 𝔥⸨(⸩\!\up{𝔥⸨StateT⸩}⸢store⸣\ 𝔥⸨(⸩\!\up{𝔥⸨NondetT⸩}⸢mplus⸣\ 𝔥⸨(⸩\!\up{𝔥⸨ReaderT⸩}⸢in-\$⸣\ 𝔥⸨(⸩\!\up{𝔥⸨StateT+⸩}⸢out-\$⸣\ 𝔥⸨ID)))))))⸩ &
\end{flalign*}
\figskip\rfloat{⸨ev-cache@⸩}
\begin{lstlisting}
¦ (define (((ev-cache ev₀) ev) e)
¦   (do ρ   ← ask-env  σ ← get-store
¦       ς   ≔ (list e ρ σ)
¦       ¢⸢out⸣ ← get-cache-out
¦       (if (∈ ς ¢⸢out⸣)
¦           (for/monad+ ([v×σ (¢⸢out⸣ ς)])
¦             (do (put-store (cdr v×σ))
¦                 (return (car v×σ))))
¦           (do ¢⸢in⸣    ← ask-cache-in
¦               v×σ₀  ≔ (if (∈ ς ¢⸢in⸣) (¢⸢in⸣ ς) ∅)
¦               (put-cache-out (¢⸢out⸣ ς v×σ₀))
¦               v     ← ((ev₀ ev) e)
¦               σ′    ← get-store
¦               v×σ′  ≔ (cons v σ′)
¦               (update-cache-out (λ (¢⸢out⸣) (¢⸢out⸣ ς (set-add (¢⸢out⸣ ς) v×σ′))))
¦               (return v)))))
\end{lstlisting}
\captionskip{Co-inductive Caching Algorithm}
\label{f:caching}
\end{mdframed}
\end{figure} %}-}

In the algorithm, when a configuration ⸨ς⸩ is first encountered, we place an
entry in the output cache mapping ⸨ς⸩ to «𝔥⸨(¢⸢in⸣⸩\ 𝔥⸨ς)⸩», which is the
“oracle” result. Also, whenever we finish computing the result ⸨v×σ'⸩ of
evaluating a configuration ⸨ς⸩, we place an entry in the output cache mapping
⸨ς⸩ to ⸨v×σ′⸩. Finally, whenever we reach a configuration ⸨ς⸩ for which a
mapping in the output cache exists, we use it immediately, ⸨return⸩ing each
result using the ⸨for/monad+⸩ iterator. Therefore, every “cache hit” on
⸨¢⸢out⸣⸩ is in one of two possible states: 1) we have already seen the
configuration, and the result is the oracle result, as desired; or 2) we have
already computed the “improved” result (w.r.t. the oracle), and need not
recompute it.

To compute the least fixed-point ⸨¢⁺⸩ for the evaluator ⸨ev-cache⸩ we perform a
standard Kleene fixed-point iteration starting from the empty map, the bottom
element for the cache, as shown in Figure~\ref{f:fixing}.

\begin{figure} %{-{
\begin{mdframed}
\rfloat{⸨fix-cache@⸩}
\begin{lstlisting}
¦ (define ((fix-cache eval) e)  
¦   (do ρ ← ask-env  σ ← get-store
¦       ς ≔ (list e ρ σ)
¦       ¢⁺ ← (mlfp (λ (¢) (do (put-cache-out ∅)
¦                             (put-store σ)
¦                             (local-cache-in ¢ (eval e))
¦                             get-cache-out)))
¦       (for/monad+ ([v×σ (¢⁺ ς)])
¦         (do (put-store (cdr v×σ))
¦             (return (car v×σ))))))
¦ (define (mlfp f)
¦   (let loop ([x ∅])
¦     (do x′ ← (f x)
¦         (if (equal? x′ x) (return x) (loop x′)))))
\end{lstlisting}
\captionskip{Finding Fixed-Points in the Cache}
\label{f:fixing}
\end{mdframed}
\end{figure} %}-}

The algorithm runs the caching evaluator ⸨eval⸩ on the given program ⸨e⸩ from
the initial environment and store. This is done inside of ⸨mlfp⸩, a monadic
least fixed-point finder. After finding the least fixed-point, the final values
and store for the initial configuration ⸨ς⸩ are extracted and returned.

Termination of the least fixed-point is justified by the monotonicity of the
evaluator (it always returns an “improved” oracle), and the finite domain of
the cache, which maps abstract configurations to pairs of values and stores,
all of which are finite.


With these peices in place we construct a complete interpreter:
\begin{alignat*}{1}
& 𝔥⸨(define (eval e) (mrun ((fix-cache (fix (ev-cache ev))) e)))⸩
\end{alignat*}
When linked with ⸨δ^⸩ and ⸨alloc^⸩, this abstract interpreter is sound and
computable, as demonstrated on the following examples:
ℑ⁅
¦ > (rec f (λ (x) (f x))
¦     (f 0))
ℑ,
¦ '()
ℑ;
¦ > (rec f (λ (n) (if0 n 1 (* n (f (- n 1)))))
¦     (f 5))
ℑ,
¦ '(N)
ℑ;
¦ > (rec f (λ (x) (if0 x 0 (if0 (f (- x 1)) 2 3)))
¦      (f (+ 1 0)))
ℑ,
¦ '(0 2 3)
ℑ⁆

\subsection*{Formal soundness and termination}\label{s:cache:formalism}

In this pearl, we have focused on the code and its intuitions rather
than rigorously establishing the usual formal properties of our
abstract interpreter, but this is just a matter of presentation: the
interpreter is indeed proven sound and computable.  We have formalized
this co-inductive caching algorithm in Appendix~\ref{s:formalism},
where we prove both that it always terminates, and that it computes a
sound over-approximation of concrete evaluation. Here, we give a short
summary of our metatheory approach.

In formalising the soundness of this caching algorithm, we extend a standard
big-step evaluation semantics into a \emph{big-step reachability} semantics,
which characterizes all intermediate configurations which are seen between the
evaluation of a single expression and its eventual result. These two
notions—\emph{evaluation} which relates expressions to fully evaluated results,
and \emph{reachability} which characterizes intermediate configuration
states—remain distinct throughout the formalism.

After specifying evaluation and reachability for concrete evaluation, we
develop a \emph{collecting} semantics which gives a precise specification for
any abstract interpreter, and an \emph{abstract} semantics which partially
specifies a sound, over-approximating algorithm w.r.t. the collecting
semantics.

The final step is to compute an oracle for the \emph{abstract evaluation
relation}, which maps individual configurations to abstractions of the values
they evaluate to. To construct this cache, we \emph{mutually} compute the
least-fixed point of both the evaluation and reachability relations: based on
what is evaluated, discover new things which are reachable, and based on what
is reachable, discover new results of evaluation. The caching algorithm
developed in this section is a slightly more efficient strategy for solving the
mutual fixed-point, by taking a deep exploration of the reachability relation
(up-to seeing the same configuration twice) rather than applying just a single
rule of inference.

\section{Pushdown \emph{à la} Reynolds}\label{s:reynolds}

By combining the finite abstraction of base values and closures with the
termination-guaranteeing cache-based fixed-point algorithm, we have obtained a
terminating abstract interpreter.  But what kind of abstract interpretation did
we get?

We have followed the basic recipe of AAM, but adapted to a compositional
evaluator instead of an abstract machine.  However, we did manage to skip over
one of the key steps in the AAM method: we never store-allocated continuations.
\begin{center}
\emph{In fact, there are no continuations at all!}
\end{center}
The abstract machine formulation of the semantics models the object-level stack
explicitly as an inductively defined data structure.  Because stacks may be
arbitrarily large, they must be finitized like base values and closures.  Like
closures, the AAM trick is to thread them through the store and then finitize
the store.  But in the definitional interpreter approach, the stack is implicit
and inherited from the meta-language.

But here is the remarkable thing: since the stack is inherited from the
meta-language, the abstract interpreter inherits the ``call-return matching''
of the meta-language, which is to say there is no loss of precision of in the
analysis of the control stack.  This is a property that usually comes at
considerable effort and engineering in the formulations of higher-order flow
analysis that model the stack explicitly.  So-called higher-order ``pushdown''
analysis has been the subject of multiple publications and a
dissertation~\cite%
{dvanhorn:Vardoulakis2011CFA2%
,dvanhorn:Earl2010Pushdown%
,local:vardoulakis-diss12%
,dvanhorn:VanHorn2012Systematic%
,dvanhorn:Earl2012Introspective%
,dvanhorn:Johnson2014Abstracting%
,dvanhorn:Johnson2014Pushdown%
,local:p4f%
}. Yet when formulated in the definitional interpreter style, the pushdown
property requires no mechanics and is simply inherited from the meta-language.

Reynolds, in his celebrated paper \emph{Definitional Interpreters for
Higher-order Programming Languages}~\cite{dvanhorn:reynolds-acm72}, first
observed that when the semantics of a programming language is presented as a
definitional interpreter, the defined language could inherit semantic
properties of the defining metalanguage.  We have now shown this observation
can be extended to \emph{abstract} interpretation as well, namely in the
important case of the pushdown property.

In the remainder of this paper, we explore a few natural extensions and
variations on the basic pushdown abstract interpreter we have established up to
this point.

\section{Widening the Store}\label{s:widening}

The abstract interpreter we've constructed so far uses a
store-per-program-state abstraction, which is precise but prohibitively
expensive. A common technique to combat this cost is to use a global
``widened'' store~\cite{dvanhorn:might-phd,dvanhorn:Shivers:1991:CFA}, which over-approximates each individual store in the
current set-up. This change is achieved easily in the monadic setup by
re-ordering the monad stack, a technique due to \citet{local:darais-oopsla2015}. Whereas before we had ⸨monad-cache@⸩ we
instead swap the order of ⸨StateT⸩ for the store and ⸨NondetT⸩:
\begin{alignat*}{1}
& 𝔥⸨(⸩\!\up{𝔥⸨ReaderT⸩}⸢env⸣\ 𝔥⸨(⸩\!\up{𝔥⸨FailT⸩}⸢errors⸣\ 𝔥⸨(⸩\!\up{𝔥⸨NondetT⸩}⸢mplus⸣\ 𝔥⸨(⸩\!\up{𝔥⸨StateT+⸩}⸢store⸣\ 𝔥⸨(⸩\!\up{𝔥⸨ReaderT⸩}⸢in-\$⸣\ 𝔥⸨(⸩\!\up{𝔥⸨StateT+⸩}⸢out-\$⸣\ 𝔥⸨ID))))))⸩
\end{alignat*}
we get a store-widened variant of the abstract interpreter. Because ⸨StateT⸩
for the store appears underneath nondeterminism, it will be automatically
widened. We write ⸨StateT+⸩ to signify that the cell of state supports such
widening. 

% To see the difference, here is an example without store-widening:
% ℑ⁅
% ¦ (let x (+ 1 0)
% ¦   (let y (if0 x 1 2)
% ¦     (let z (if0 x 3 4)
% ¦       (if0 x y z))))
% ℑ,
% ¦ '((4 . ((x N) (y 2) (z 4)))
% ¦   (1 . ((x N) (y 1) (z 3)))
% ¦   (2 . ((x N) (y 2) (z 3)))
% ¦   (3 . ((x N) (y 1) (z 3)))
% ¦   (1 . ((x N) (y 1) (z 4)))
% ¦   (3 . ((x N) (y 2) (z 3)))
% ¦   (2 . ((x N) (y 2) (z 4)))
% ¦   (4 . ((x N) (y 1) (z 4))))
% ℑ⁆
% and with:
% ℑ⁅
% ¦ (let x (+ 1 0)
% ¦   (let y (if0 x 1 2)
% ¦     (let z (if0 x 3 4)
% ¦       (if0 x y z))))
% ℑ,
% ¦ '((1 3 2 4) . ((x N) (y 1 2) (z 3 4)))
% ℑ⁆
% Notice that before widening, the result is a set of value, store
% pairs.  After widening the result is a pair of a set of values and a
% store.  Importantly, the cache, which bounds the overall run-time of
% the abstract interpreter, is potentially exponential without
% store-widening, but collapses to polynomial after store-widening.

\begin{figure} %{-{
\begin{mdframed}
\rfloat{⸨precise-δ@⸩}
\begin{lstlisting}
¦ (define (δ o n₀ n₁)
¦   (match* (o n₀ n₁)
¦     [('+ (? num?) (? num?)) (return (+ n₀ n₁))]
¦     [('+ _        _       ) (return 'N)] ... ))
¦ (define (zero? v)
¦   (match v
¦     ['N (mplus (return #t) (return #f))]
¦     [_  (return (zero? v))]))
\end{lstlisting}
\figskip\rfloat{⸨store-crush@⸩}
\begin{lstlisting}
¦ (define (find a)
¦   (do σ ← get-store
¦       (for/monad+ ([v (σ a)]) (return v))))
¦ (define (crush v vs)
¦   (if (closure? v)
¦       (set-add vs v)
¦       (set-add (set-filter closure? vs) 'N)))
¦ (define (ext a v)
¦   (update-store (λ (σ) (if (∈ a σ)
¦                            (σ a (crush v (σ a)))
¦                            (σ a (set v))))))
\end{lstlisting}
\captionskip{An Alternative Abstraction for Precise Primitives}
\label{f:pres-delta}
\end{mdframed}
\end{figure} %}-}

\section{An Alternative Abstraction}\label{s:alt-abstraction}

In this section, we demonstrate how easy it is to experiment with alternative
abstraction strategies by swapping out components.  In particular we look at an
alternative abstraction of primitive operations and store joins.

\begin{figure} %{-{
\rfloat{⸨symbolic-monad@⸩}
\begin{flalign*}
                  & 𝔥⸨(define-monad⸩
  & \\[\monadgobble]& ␣␣𝔥⸨(⸩\!\up{𝔥⸨ReaderT⸩}⸢env⸣\ 𝔥⸨(⸩\!\up{𝔥⸨FailT⸩}⸢errors⸣\ 𝔥⸨(⸩\!\up{𝔥⸨StateT⸩}⸢store⸣\ 𝔥⸨(⸩\!\up{𝔥⸨StateT⸩}⸢path⸣\ 𝔥⸨(⸩\!\up{𝔥⸨NondetT⸩}⸢mplus⸣\ 𝔥⸨ID))))))⸩
\end{flalign*}
\figskip\rfloat{⸨ev-symbolic@⸩}
\begin{lstlisting}
¦ (define (((ev-symbolic ev₀) ev) e)
¦   (match e [(sym x) (return x)]
¦            [e       ((ev₀ ev) e)]))
\end{lstlisting}
\figskip\rfloat{⸨δ-symbolic@⸩}
\begin{lstlisting}
¦ (define (δ o n₀ n₁)
¦   (match* (o n₀ n₁)
¦     [('/ n₀ n₁)
¦      (do z? ← (zero? n₁)
¦          (cond [z? fail]
¦                [(and (num? n₀) (num? n₁))
¦                 (return (/ n₀ n₁))]
¦                [else (return `(/ ,n₀ ,n₁))]))] ... ))
¦ (define (zero? v)
¦   (do φ ← get-path-cond
¦       (match v
¦         [(? num? n)             (return (= 0 n))]
¦         [v #:when (∈ v φ)       (return #t)]
¦         [v #:when (∈ `(¬ ,v) φ) (return #f)]
¦         [v (mplus (do (refine v)       (return #t))
¦                   (do (refine `(¬ ,v)) (return #f)))])))
\end{lstlisting}
\vspace{-0.75em}
\caption{Symbolic Execution Variant}
\label{f:symbolic}
\vspace{-1em}
\end{figure} %}-}

Figure~\ref{f:pres-delta} defines two new components: ⸨precise-δ@⸩ and
⸨store-crush@⸩.  The first is an alternative interpretation for primitive
operations that is \emph{precision preserving}.  Unlike ⸨δ^@⸩, it does not
introduce abstraction, it merely propagates it.  When two concrete
numbers are added together, the result will be a concrete number, but if either
number is abstract then the result is abstract.

This interpretation of primitive operations clearly doesn't impose a finite
abstraction on its own, because the state space for concrete numbers is
infinite. If ⸨precise-δ@⸩ is linked with the ⸨store-nd@⸩ implementation of the
store, termination is therefore not guaranteed.  

The ⸨store-crush@⸩ operations are designed to work with ⸨precise-δ@⸩ by
performing \emph{widening} when joining multiple concrete values into the
store. This abstraction offers a high-level of precision; for example,
``straight-line'' arithmetic operations are computed with full precision:
ℑ⁅
¦ > (* (+ 3 4) 9)
ℑ,
¦ '(63)
ℑ⁆
Even linear binding and arithmetic preserves precision:
ℑ⁅
¦ > ((λ (x) (* x x)) 5)
ℑ,
¦ '(25)
ℑ⁆
It's only when the approximation of binding structure comes in to
contact with base values that we see a loss in precision:
ℑ⁅
¦ > (let f (λ (x) x)
¦     (* (f 5) (f 5)))
ℑ,
'(N)
ℑ⁆
This combination of ⸨precise-δ@⸩ and ⸨store-crush@⸩ allows termination for most
programs, but still not all. In the following example, ⸨id⸩ is eventually
applied to a widened argument ⸨'N⸩, which makes both conditional branches
reachable. The function returns ⸨0⸩ in the base case, which is propagated to
the recursive call and added to ⸨1⸩, which yields the concrete answer ⸨1⸩.
This results in a cycle where the intermediate sum returns ⸨2⸩, ⸨3⸩, ⸨4⸩ when
applied to ⸨1⸩, ⸨2⸩, ⸨3⸩, etc.
ℑ⁅
¦ > (rec id (λ (n) (if0 n 0 (+ 1 (id (- n 1)))))
¦     (id 3))
ℑ,
¦ timeout
ℑ⁆
To ensure termination for all programs, we assume all references to
primitive operations are $η$-expanded, so that store-allocations also
take place at primitive applications, ensuring widening at repeated
bindings. In fact, all programs terminate when using ⸨precise-δ@⸩,
⸨store-crush@⸩ and «η»-expanded primitives, which means we have a
achieved a computable and uniquely precise abstract interpreter.

Here we see one of the strengths of the extensible, definitional approach to
abstract interpreters. The combination of added precision and widening is
encoded quite naturally. In contrast, it's hard to imagine how such a
combination could be formulated as, say, a constraint-based flow analysis.

\section{Symbolic execution}\label{s:symbolic}

\begin{figure} %{-{
\rfloat{⸨δ^-symbolic@⸩}
\begin{lstlisting}
¦ (define (δ o n₀ n₁)
¦   (match* (o n₀ n₁)
¦     [('/ n₀ n₁) (do z? ← (zero? n₁)
¦                     (cond [z? fail]
¦                           [(member 'N (list n₀ n₁)) (return 'N)]
¦                           ... ))]
¦     ... ))
¦ (define (zero? v)
¦   (do φ ← get-path-cond
¦       (match v ['N (mplus (return #t) (return #f))] ... )))
\end{lstlisting}
\captionskip{Symbolic Execution with Abstract Numbers}
\label{f:symbolic-widen}
\end{figure} %}-}

As a final exercise in applying our definitional abstract interpretation
framework, we develop a symbolic execution engine, and use it to perform sound
program verification. First we describe the monad stack and metafunctions that
implement a symbolic executor~\cite{dvanhorn:King1976Symbolic}, then show how
abstractions discussed in previous sections can be applied to enforce
termination, turning a traditional symbolic execution into a path-sensitive
verification engine.

%\subsection{Symbolic Execution}
To support symbolic execution, first we extend the syntax of the language to
support symbolic numbers:
\begin{alignat*}{4}
   e ∈ &&\mathrel{}     exp ⩴ &\mathrel{} … ∣ 𝔥⸨(sym⸩\ x𝔥⸨)⸩ &\hspace{1em} [⦑\emph{symbolic number}⦒]
\\ ε ∈ &&\mathrel{}    pexp ⩴ &\mathrel{} e ∣ ¬e             &\hspace{1em} [⦑\emph{path expression}⦒]
\\ φ ∈ &&\mathrel{}    pcon ≔ &\mathrel{} ℘(pexp)   &\hspace{1em} [⦑\emph{path condition}⦒]
\end{alignat*}
Figure~\ref{s:symbolic} shows the units needed to turn the existing interpreter
into a symbolic executor. Primitives such as ⸨'/⸩ now also take as input and
return symbolic values. As standard, symbolic execution employs a
path-condition accumulating assumptions made at each branch, allowing the
elimination of infeasible paths and construction of test cases. We represent
the path-condition ⸨φ⸩ as a set of symbolic values or their negations.
If ⸨e⸩ is in ⸨φ⸩, ⸨e⸩ is assumed to evaluate to ⸨0⸩;
if ⸨¬ e⸩ is in ⸨φ⸩, ⸨e⸩ is assumed to evaluate to non-⸨0⸩.
This set is another state component provided by ⸨StateT⸩ in the monad
transformer stack. Monadic operations ⸨get-path-cond⸩ and ⸨refine⸩ reference
and update the path-condition. The metafunction ⸨zero?⸩ works similarly to the
concrete counterpart, but also uses the path-condition to prove that some
symbolic numbers are definitely ⸨0⸩ or non-⸨0⸩. In case of uncertainty, ⸨zero?⸩
returns both answers instead of refining the path-condition with the assumption
made.

In the following example, the symbolic executor recognizes that result ⸨3⸩ and
division-by-0 error are not feasible:
ℑ⁅
¦ > (if0 'x (if0 'x 2 3) (/ 5 'x))
ℑ,
¦ (set (cons '(/ 5 x) (set '(¬ x)))
¦      (cons 2 (set 'x)))
ℑ⁆
A scaled up symbolic executor could implement ⸨zero?⸩ by calling out to an SMT
solver for interesting arithmetics, or extend the language with symbolic
functions and blame semantics for sound higher-order symbolic
execution~\cite{dvanhorn:TobinHochstadt2012Higherorder,dvanhorn:Nguyen2015Relatively}.

\newcommand{\lamif}{«λ⦑IF⦒» }

\paragraph{From Symbolic Execution to Verification}

Traditional symbolic executors mainly aim to find bugs and do not provide
termination guarantee. However, when we apply to this symbolic executor the
finite abstractions presented in previous sections, namely base value widening
and finite allocation (Section~\ref{s:base}), and caching and fixing
(Section~\ref{s:cache}), we turn the symbolic execution into a sound,
path-sensitive program verification.

Operations on symbolic values introduce a new source of infinity in the
state-space, because the space of symbolic values is not finite. We therefore
widen a symbolic value to the abstract number ⸨'N⸩ when it shares an address
with a different number, similarly to the precision-preserving abstraction from
Section~\ref{s:alt-abstraction}. Figure~\ref{f:symbolic-widen} shows extension
to ⸨δ⸩ and ⸨zero?⸩ in the presence of ⸨'N⸩. The different treatments of ⸨'N⸩
and symbolic values clarifies that abstract values are not symbolic values: the
former stands for a set of multiple values, whereas the latter stands for an
single unknown value. Tests on abstract number ⸨'N⸩ do not strengthen the
path-condition; it is unsound to accumulate any assumption about ⸨'N⸩.


\section{Try It Out}\label{s:try-it}

All of the components discussed in this paper have been implemented as
units~\cite{local:flatt-pldi98} in Racket~\cite{dvanhorn:plt-tr1}.  We have
also implemented a ⸨#lang⸩ language so that composing and experimenting with
these interpreters is easy.  Assuming Racket is installed, you can install the
⸨monadic-eval⸩ package with (URL redacted for double-blind).
%\begin{center}
%\begin{verbatim}
%raco pkg install https://github.com/plum-umd/monadic-eval.git
%\end{verbatim}
%\end{center}

A ⸨#lang monadic-eval⸩ program starts with a list of components, which are
linked together, and an expression producing an evaluator.  Subsequent forms
are interpreted as expressions when run. Programs can be run from the
command-line or interactively in the DrRacket IDE.

\section{Related Work}

This work draws upon and re-presents many existing ideas from the literature on
abstract interpretation for higher-order languages.  In particular, it closely
follows the abstracting abstract
machine~\cite{dvanhorn:VanHorn2010Abstracting,dvanhorn:VanHorn2012Systematic}
approach to deriving abstract interpreters from semantics for higher-order
languages.  The key difference here is that we have done it in the setting of a
monadic definitional interpreter instead of an abstract machine.  This involved
a novel caching mechanism and fixed-point algorithm, but otherwise followed the
same recipe.  Remarkably, the pushdown property is simply inherited from the
meta-language rather than require explicit mechanisms within the abstract
interpreter.

The use of monads and monad transformers to make extensible (concrete)
interpreters is a well-known
idea~\cite{davdar:Moggi:1989:Monads,local:steele-popl94,dvanhorn:Liang1995Monad},
which we have extended to work for compositional abstract interpreters.  The
use of monads and monad transformers in machine based-formulatons of abstract
interpreters has previously been explored by Sergey, \emph{et
al.}~\cite{dvanhorn:Sergey2013Monadic} and Darais \emph{et
al.}~\cite{local:darais-oopsla2015}, respectively.  Darais has also shown that
certain monad transformers are also \emph{Galois transformers}, i.e. they
compose to form monads that are Galois connections.  This idea may pave a path
forward for having both componential code \emph{and proofs} for abstract
interpreters in the style presented here.

The caching mechanism used to ensure termination in our abstract interpreter is
similar to that used by Johnson and Van
Horn~\cite{dvanhorn:Johnson2014Abstracting}.  They use a local- and
meta-memoization table in a machine-based interpreter to ensure termination for
a pushdown abstract interpreter.  This mechanism is in turn reminiscent of
Glück's use of memoization in an interpreter for two-way non-deterministic
pushdown automata~\cite{local:gluck-schmidtfest13}.

Vardoulakis, who was the first to develop the idea of a pushdown abstraction
for higher-order flow analysis~\cite{dvanhorn:Vardoulakis2011CFA2}, formalized
CFA2 using a CPS model, which is similar in spirit to a machine-based model.
However, in his dissertation~\cite{local:vardoulakis-diss12} he sketches an
alternative presentation dubbed ``Big CFA2'' which is a big-step operational
semantics for doing pushdown analysis quite similar in spirit to the approach
presented here.  One key difference is that Big CFA2 fixes a particular coarse
abstraction of base values and closures---for example, both branches of a
conditional are always evaluated.  Consequently, it only uses a single
iteration of the abstract evaluation function, and avoids the need for the
cache-based fixed-point of Section~\ref{s:cache}.  We don't believe Big CFA2 as
stated is unsound, however if the underlying abstractions were tightened, it
appears it would run in to the same issues identified here.

Our formulation of a pushdown abstract interpreter computes an abstraction
similar to the many existing variants of pushdown flow analysis~\cite%
{dvanhorn:Vardoulakis2011CFA2%
,dvanhorn:Earl2010Pushdown%
,local:vardoulakis-diss12%
,dvanhorn:VanHorn2012Systematic%
,dvanhorn:Earl2012Introspective%
,dvanhorn:Johnson2014Abstracting%
,dvanhorn:Johnson2014Pushdown%
,local:p4f%
}.
% @;{ Our incorporation of an
% abstract garbage collector into a pushdown abstract interpreter
% achieves a similar goal as that of so-called @emph{introspective}
% pushdown abstract interpreters@~cite[earl-icfp12 johnson-jfp14].  }
The mixing of symbolic execution and abstract intrepretation is similar in
spirit to the \emph{logic flow analysis} of Might~\cite{local:might-popl07},
albeit in a pushdown setting and with a stronger notion of negation; generally,
our presentation resembles traditional formulations of symbolic execution more
closely.  Our approach to symbolic execution only handles the first-order case
of symbolic values, as is traditional.  However, Nguyễn's work on higher-order
symbolic execution~\cite{dvanhorn:Nguyen2015Relatively} demonstrates how to
scale to behavioral symbolic values.  In principle, it should be possible to
handle this case in our approach by adapting Nguyễn's method to a formulation
in a compositional evaluator.

We have eschewed soundness proofs in this paper.  This is done in part to
emphasize the pearly intuitions and constructions of abstract definitional
interpreters and in part because it is far less clear how to prove soundness
when compared to the machine-based formulations. Part of the difficulty stems
from the set-up to support extensibility. As mentioned previously, perhaps
Galois transformers~\cite{local:darais-oopsla2015} can help with this aspect.
But even if we fixed a particular set of components and monad transformer
stack, we run up against the challenge of having to prove soundness in the
presence of concrete computations which may not terminate. Handling this in the
small-step setting is easy using a preservation argument, but it's not clear
how to do it with our approach.  Rompf and Amin's recent work on proving type
soundness with definitional interpreters~\cite{local:rompf-arxiv2015} appears
revelant and perhaps paves a way forward.

Now that we have abstract interpreters formulated with a basis in abstract
machines and with a basis in monadic interpreters, an obvious question is can
we obtain a correspondence between them similar to the functional
correspondence between their concrete
counterparts~\cite{dvanhorn:Ager2005Functional}.  An interesting direction for
future work is to try to apply the usual tools of defunctionalization, CPS, and
refocusing to see if we can interderive these abstract semantic artifacts.

\section{Conclusions}\label{s:conclusion}

We have shown that a definitional interpreter written in monadic style can
express a wide variety of semantics, such as the usual concrete semantics,
collecting semantics, abstract interpretations, symbolic execution, and several
combinations thereof. 

Remarkably, our abstract interpreter implements a form of pushdown abstraction
in which calls and returns are always properly matched in the abstract
semantics.  True to the definitional style of Reynolds, the evaluator involves
no explicit mechanics to achieve this property; it is simply inherited from the
defining language.

We believe this formulation of abstract interpretation offers a promising new
foundation towards re-usable components for the static analysis and
verification of higher-order programs.


\begin{acks}
  We thank Sam Tobin-Hochstadt and Dionna Glaze for several fruitful
  conversations while developing the ideas in this work.  We thank
  Ilya Sergey for comments and in particular, bringing Jones and
  Nielsen's two-level metalanguage for abstract interpretation to our
  attention~\cite{local:jones-ai1995}.
\end{acks}

%\balance
%\bibliographystyle{abbrvnat}
\bibliography{davdar,dvanhorn,local}

\appendix
\section{Formalism}\label{s:formalism}

In this section we formalize our approach to designing definitional abstract
interpreters. We begin with a ``ground truth'' big-step semantics and concludes
with the fixpoint iteration strategy described in Section~\ref{s:cache}, which
we prove sound and computable w.r.t. a synthesized abstract semantics. The
design is systematic, and applies to arbitrary developments which use big-step
operational semantics. We demonstrate the systematic process as applied to a
subset of the language described in Figure~\ref{f:syntax}, which we call
\lamif:
\begin{alignat*}{1}
e ∈ exp ⩴ n ∣ x ∣ λx.e ∣ e(e) ∣ ⟬if0⟭(e)❴e❵❴e❵ ∣ b(e,e) 
\end{alignat*}
\vspace{-2.0em}
\begin{alignat*}{6}
                n ∈&&\mathrel{} nums  \mathrel{\hphantom{≔}} &\mathrel{}             &\quad\quad ℓ ∈ &&\mathrel{}  addr ≔ &\mathrel{} var × time 
\\[\mathgobble] x ∈&&\mathrel{} vars  \mathrel{\hphantom{≔}} &\mathrel{}             &\quad\quad σ ∈ &&\mathrel{} store ≔ &\mathrel{} addr → val⸤⊥⸥ 
\\[\mathgobble] b ∈&&\mathrel{} binop                    ≔   &\mathrel{} ❴plus,…❵    &\quad\quad v ∈ &&\mathrel{}   val ⩴ &\mathrel{} n ∣ ⟨λx.e,ρ⟩ 
\\[\mathgobble] τ ∈&&\mathrel{}  time                    ≔   &\mathrel{} ℕ           &\quad\quad ρ ∈ &&\mathrel{}   env ≔ &\mathrel{} var → addr⸤⊥⸥ 
\end{alignat*}

\paragraph{Concrete Semantics}

\begin{figure*} %{{{
\begin{flushright}[\emph{Concrete Evaluation}]\quad\fbox{«ρ,τ⊢e,σ⇓v,σ′»}\end{flushright}
\vspace{-0.75em}
\begin{mathpar}
  \inferrule*[left=(Lit)]{ }{ρ , τ ⊢ n , σ ⇓ n , σ}

  \inferrule*[left=(Var)]{ }{ρ , τ ⊢ x , σ ⇓ σ(ρ(x)) , σ}

  \inferrule*[left=(Lam)]{ }{ρ , τ ⊢ λx.e , σ ⇓ ⟨λx.e,ρ⟩ , σ}\vspace{-0.75em}

  \inferrule*[left=(Bin)]{
  ρ , τ ⊢ e₁ , σ  ⇓ v₁ , σ₁ \\
  ρ , τ ⊢ e₂ , σ₁ ⇓ v₂ , σ₂}
  {ρ , τ ⊢ b(e₁,e₂) , σ ⇓ ⟦b⟧(v₁,v₂) , σ₂}\vspace{-0.75em}

  \inferrule*[left=(App),right={\begin{minipage}{2em}\ssmall
    \begin{alignat*}{1}
    \begin{alignedat}{2} 
⟨λx.e′,ρ′⟩ = &\mathrel{} v₁ \\[-0.5em]
        ℓ  = &\mathrel{} ⟨x,τ′⟩ \\[-0.5em]
        τ′ \mathrel{\hphantom{=}} &\mathrel{} ⦑\emph{fresh}⦒
      \end{alignedat}
    \end{alignat*}
  \end{minipage}}]{
  ρ       , τ  ⊢ e₁    , σ        ⇓ v₁ , σ₁ \\
  ρ       , τ  ⊢ e₂    , σ₁       ⇓ v₂ , σ₂ \\
  ρ′[x↦ℓ] , τ′ ⊢ e′    , σ₂[ℓ↦v₂] ⇓ v′ , σ₃}
  {ρ      , τ ⊢ e₁(e₂) , σ        ⇓ v′ , σ₃}\vspace{-0.75em}

  \inferrule*[left=(IfT),right={\ssmall «n=0»}]{
  ρ , τ ⊢ e₁ , σ ⇓ n , σ₁ \\
  ρ , τ ⊢ e₂ , σ₁ ⇓ v , σ₂}
  {ρ , τ ⊢ ⟬if0⟭(e₁)❴e₂❵❴e₃❵ , σ ⇓ v , σ₂}

  \inferrule*[left=(IfF),right={\ssmall «n≠0»}]{
    ρ , τ ⊢ e₁ , σ  ⇓ n , σ₁ \\
    ρ , τ ⊢ e₃ , σ₁ ⇓ v , σ₂}
  {ρ , τ ⊢ ⟬if0⟭(e₁)❴e₂❵❴e₃❵ , σ ⇓ v , σ₂}
\end{mathpar}
\begin{flushright}[\emph{Concrete Reachability}]\quad\fbox{«ρ,τ⊢e,σ⇑ς»}\end{flushright}
\vspace{-0.25em}
\begin{mathpar}
  \inferrule*[left=(Refl)]{ }{ρ,τ⊢e,σ⇑⟨e,ρ,σ,τ⟩}

  \inferrule*[left=(RBin1)]
  {ρ,τ⊢e₁,σ⇑ς}
  {ρ,τ⊢b(e₁,e₂),σ⇑ς}

  \inferrule*[left=(RBin2)]
  {  ρ,τ⊢e₁,σ⇓v₁,σ₁
  \\ ρ,τ⊢e₂,σ₁⇑ς}
  {ρ,τ⊢b(e₁,e₂),σ⇑ς}\vspace{-0.75em}

  \inferrule*[left=(RApp1)]
   {ρ,τ⊢e₁,σ⇑ς}
   {ρ,τ⊢e₁(e₂),σ⇑ς}

   \inferrule*[left=(RApp2),right={\ssmall «⟨λx.e′,ρ′⟩=v₁»}]
   {  ρ,τ⊢e₁,σ⇓v₁,σ₁
   \\ ρ,τ⊢e₂,σ₁⇑ς}
   {ρ,τ⊢e₁(e₂),σ⇑ς}\vspace{-0.75em}

   \inferrule*[left=(RApp3),right={\begin{minipage}{2em}\ssmall
       \begin{alignat*}{1}
       \begin{alignedat}{2} 
    ⟨λx.e′,ρ′⟩ = &\mathrel{} v₁ \\[-0.5em]
           ℓ  = &\mathrel{} ⟨x,τ′⟩ \\[-0.5em]
           τ′ \mathrel{\hphantom{=}} &\mathrel{} ⦑\emph{fresh}⦒
         \end{alignedat}
       \end{alignat*}
     \end{minipage}}]
  {  ρ,τ⊢e₁,σ ⇓v₁,σ₁
  \\ ρ,τ⊢e₂,σ₁⇓v₂,σ₂
  \\ ρ′[x↦ℓ],τ′⊢e′,σ₂[ℓ↦v₂]⇑ς}
  {ρ,τ⊢e₁(e₂),σ⇑ς}\vspace{-0.75em}

  \inferrule*[left=(RIf1)]
  {ρ,τ⊢e₁,σ⇑ς}
  {ρ,τ⊢⟬if0⟭(e₁)❴e₂❵❴e₃❵,σ⇑ς}

  \inferrule*[left=(RIfT),right={\ssmall «n=0»}]
  {  ρ,τ⊢e₁,σ⇓n,σ₁
  \\ ρ,τ⊢e₂,σ₁⇑ς}
  {ρ,τ⊢⟬if0⟭(e₁)❴e₂❵❴e₃❵,σ⇑ς}\vspace{-0.75em}

  \inferrule*[left=(RIfF),right={\ssmall «n≠0»}]
  {  ρ,τ⊢e₁,σ⇓n,σ₁
  \\ ρ,τ⊢e₃,σ₁⇑ς}
  {ρ,τ⊢⟬if0⟭(e₁)❴e₂❵❴e₃❵,σ⇑ς}
\end{mathpar}
\caption{\lamif{} Big-step Concrete Evaluation and Reachability Semantics}
\label{f:lamif-concrete}
\end{figure*} %}}}

We begin with the concrete semantics of \lamif as a big-step evaluation
relation «ρ,τ⊢e,σ⇓v,σ′», shown in Figure~\ref{f:lamif-concrete}. The definition
is mostly standard: «ρ» and «σ» are the environment and store, «e» is the
initial expression, and «v» is the resulting value. The argument «τ» represents
``time,'' which when abstracted supports modeling execution contexts like
call-site sensitivity. Concretely time is modelled as a natural number, and all
that is required is that ``fresh'' numbers are available for allocating values
in the store.

\begin{figure*} %{-{
\begin{flushright}[\emph{Collecting Evaluation}]\quad\fbox{«ρ,τ⊢e,∿{σ}⇓∿{v},∿{σ}′»}\end{flushright}
\vspace{-0.75em}
\begin{mathpar}
  \inferrule*[left=(OLit)]{ }{ρ,τ⊢n,∿{σ}⇓❴n❵,∿{σ}}

  \inferrule*[left=(OVar)]{ }{ρ,τ⊢x,∿{σ}⇓∿{σ}(ρ(x)),∿{σ}}

  \inferrule*[left=(OLam)]{ }{ρ,τ⊢λx.e,∿{σ}⇓❴⟨λx.e,ρ⟩❵,∿{σ}}\vspace{-0.75em}

    \inferrule*[left=(OBin)]
  {  ρ,τ⊢e₁,∿{σ}⇓∿{v}₁,∿{σ}₁
  \\ ρ,τ⊢e₂,∿{σ}₁⇓∿{v}₂,∿{σ}₂}
  {  ρ,τ⊢b(e₁,e₂),∿{σ}⇓∿{⟦b⟧}(∿{v}₁,∿{v}₂),∿{σ}₂}\vspace{-0.75em}

  \inferrule*[left=(OApp),right={\begin{minipage}{2em}\ssmall
    \begin{alignat*}{1}
    \begin{alignedat}{2} 
              ⟨λx.e′,ρ′⟩ ∈ &\mathrel{} ∿{v}₁ \\[-0.5em]
                      ℓ  = &\mathrel{} ⟨x,τ′⟩ \\[-0.5em]
                      τ′ \mathrel{\hphantom{=}} &\mathrel{} ⦑\emph{fresh}⦒
      \end{alignedat}
    \end{alignat*}
  \end{minipage}}]
  {  ρ      ,τ ⊢e₁    ,∿{σ} ⇓∿{v}₁,∿{σ}₁
  \\ ρ      ,τ ⊢e₂    ,∿{σ}₁⇓∿{v}₂,∿{σ}₂
  \\ ρ′[x↦ℓ],τ′⊢e′    ,∿{σ}₂[ℓ↦∿{v}₂]⇓∿{v}′,∿{σ}₃}
  {ρ      ,τ ⊢e₁(e₂),∿{σ} ⇓∿{v}′,∿{σ}₃}\vspace{-0.75em}

  \inferrule*[left=(OIfT),right={\ssmall «0∈∿{v}₁»}]
  {  ρ,τ⊢e₁,∿{σ}⇓∿{v}₁,∿{σ}₁
  \\ ρ,τ⊢e₂,∿{σ}₁⇓∿{v},∿{σ}₂}
  {  ρ,τ⊢⟬if0⟭(e₁)❴e₂❵❴e₃❵,∿{σ}⇓∿{v},∿{σ}₂}

  \inferrule*[left=(OIfF),right={\begin{minipage}{2em}\ssmall
      \begin{alignat*}{2}
        n ∈ &\mathrel{} ∿{v}₁ \\[-0.5em]
        n ≠ &\mathrel{} 0
      \end{alignat*}
    \end{minipage}}]
    {  ρ,τ⊢e₁,∿{σ}⇓∿{v}₁,∿{σ}₁
    \\ ρ,τ⊢e₃,∿{σ}₁⇓∿{v},∿{σ}₂}
    {  ρ,τ⊢⟬if0⟭(e₁)❴e₂❵❴e₃❵,∿{σ}⇓∿{v},∿{σ}₂}
\end{mathpar}
\begin{flushright}[\emph{Collecting Reachability}]\quad\fbox{«ρ,τ⊢e,∿{σ}⇑∿{ς}»}\end{flushright}
\vspace{-0.25em}
\begin{mathpar}
  \inferrule*[left=(ORefl)]{ }{ρ,τ⊢e,∿{σ}⇑⟨e,ρ,∿{σ},τ⟩}

  \inferrule*[left=(ORBin1)]
  {ρ,τ⊢e₁,∿{σ}⇑ς}
  {ρ,τ⊢b(e₁,e₂),∿{σ}⇑∿{ς}}

  \inferrule*[left=(ORBin2)]
  {  ρ,τ⊢e₁,∿{σ}⇓∿{v}₁,∿{σ}₁
  \\ ρ,τ⊢e₂,∿{σ}₁⇑∿{ς}}
  {ρ,τ⊢b(e₁,e₂),∿{σ}⇑∿{ς}}\vspace{-0.75em}

  \inferrule*[left=(ORApp1)]
   {ρ,τ⊢e₁,∿{σ}⇑∿{ς}}
   {ρ,τ⊢e₁(e₂),∿{σ}⇑∿{ς}}

   \inferrule*[left=(ORApp2),right={\ssmall «⟨λx.e′,ρ′⟩∈∿{v}₁»}]
   {  ρ,τ⊢e₁,∿{σ}⇓∿{v}₁,∿{σ}₁
   \\ ρ,τ⊢e₂,∿{σ}₁⇑∿{ς}}
   {ρ,τ⊢e₁(e₂),∿{σ}⇑∿{ς}}\vspace{-0.75em}

   \inferrule*[left=(ORApp3),right={\begin{minipage}{2em}\ssmall
       \begin{alignat*}{1}
       \begin{alignedat}{2} 
  ⟨λx.e′,ρ′⟩ ∈ &\mathrel{} ∿{v}₁ \\[-0.5em]
           ℓ  = &\mathrel{} ⟨x,τ′⟩ \\[-0.5em]
           τ′ \mathrel{\hphantom{=}} &\mathrel{} ⦑\emph{fresh}⦒
         \end{alignedat}
       \end{alignat*}
     \end{minipage}}]
  {  ρ,τ⊢e₁,∿{σ} ⇓∿{v}₁,∿{σ}₁
  \\ ρ,τ⊢e₂,∿{σ}₁⇓∿{v}₂,∿{σ}₂
  \\ ρ′[x↦ℓ],τ′⊢e′,∿{σ}₂[ℓ↦∿{v}₂]⇑∿{ς}}
  {ρ,τ⊢e₁(e₂),∿{σ}⇑∿{ς}}\vspace{-0.75em}

  \inferrule*[left=(ORIf1)]
  {ρ,τ⊢e₁,∿{σ}⇑∿{ς}}
  {ρ,τ⊢⟬if0⟭(e₁)❴e₂❵❴e₃❵,∿{σ}⇑∿{ς}}

  \inferrule*[left=(ORIfT),right={\ssmall «0∈∿{v}₁»}]
  {  ρ,τ⊢e₁,∿{σ}⇓∿{v}₁,∿{σ}₁
  \\ ρ,τ⊢e₂,∿{σ}₁⇑∿{ς}}
  {ρ,τ⊢⟬if0⟭(e₁)❴e₂❵❴e₃❵,∿{σ}⇑∿{ς}}\vspace{-0.75em}

  \inferrule*[left=(ORIfF),right={\begin{minipage}{2em}\ssmall 
    \begin{alignat*}{2}
      n ∈ &\mathrel{} ∿{v}₁ \\[-0.5em]
      n ≠ &\mathrel{} 0
    \end{alignat*}
    \end{minipage}}]
    {  ρ,τ⊢e₁,∿{σ}⇓∿{v}₁,∿{σ}₁
    \\ ρ,τ⊢e₃,∿{σ}₁⇑∿{ς}}
    {ρ,τ⊢⟬if0⟭(e₁)❴e₂❵❴e₃❵,∿{σ}⇑∿{ς}}
\end{mathpar}
\caption{Big-step Collecting Evaluation and Reachability Semantics}
\label{f:lamif-collecting}
\end{figure*} %}-}

\paragraph{Reachability}

The primary limitation of using big-step semantics as a starting point for
abstraction is that intermediate computations are not represented in the model
for evaluation. For example, consider the program that applies the identity
function to an expression that loops, which we notate «Ω»:
\[ (λx.x)(Ω) \]
A big-step evaluation relation can only describe results of terminating
computations, and because this program never terminates, such a relation says
nothing about the behavior of the program. A good static analyzer will explore
the behavior of «Ω» to (possibly) discover that it loops, or more importantly,
to provide analysis results (like data-flow or side-effects) for intermediate
computation states.

The need to analyze intermediate states is the primary reason that big-step
semantics are overlooked as a starting point for abstract interpretation. To
remedy the situation, while remaining in a big-step setting, we introduce a
big-step \emph{reachability} relation, notated «ρ,τ⊢e,σ⇑ς» and also shown in
Figure~\ref{f:lamif-concrete}. Configurations «ς» are tuples «⟨e,ρ,σ,τ⟩»,
and are reachable when evaluation passes through the configuration at any point
on its way to a final value, or during an infinite loop.

The complete big-step semantics of an expression («e») under environment («ρ»),
store («σ») and time («τ»), which we notate «⟦e⟧⸢bs⸣(ρ,σ,τ)», is then the set
of all \emph{reachable evaluations}:
\begin{alignat*}{2}
                ⟦e⟧⸢bs⸣(ρ,σ,τ) ≔ ❴⟨v,σ″⟩ ∣ &\mathrel{} ρ,σ,τ⇑⟨e′,ρ′,σ′,τ′⟩ 
\\[\mathgobble]                          ∧ &\mathrel{} ρ′,τ′⊢e′,σ′⇓v,σ″❵
\end{alignat*}
We construct a formal bridge between the big-step and small-step worlds through
the complete big-step semantics («⟦e⟧⸢bs⸣») and a complete small-step semantics
«↝⸢*⸣», which is traditionally used as the starting point of abstraction for
program analysis:
\begin{alignat*}{1}
                & ⟦e⟧⸢ss⸣(ρ,σ,τ) ≔ 
\\[\mathgobble] & ␣\begin{alignedat}{2} ❴⟨v,σ″⟩ ∣ &\mathrel{} ∀κ. ⟨e,ρ,σ,τ,κ⟩ ↝⸢*⸣ ⟨e′,ρ′,σ′,τ′,κ′\!⧺\!κ⟩ 
                   \\[\mathgobble]              ∧ &\mathrel{} ⟨e′,ρ′,σ′,τ′,κ′⧺κ⟩ ↝⸢*⸣ ⟨v,ρ″,σ″,τ″,κ′\!⧺\!κ⟩❵
                   \end{alignedat}
\end{alignat*}
We connect the complete big-step and small-step semantics through the following
theorem:
\begin{theorem}[Complete Big-step/Small-step Equivalence]
  \[ ⟦e⟧⸢bs⸣(ρ,σ,τ) = ⟦e⟧⸢ss⸣(ρ,σ,τ) \]
\end{theorem}
The proof is by induction on the big-step derivation for «⊆», and on the
transitive small-step derivation for «⊇».

\paragraph{Collecting Semantics}

Before abstracting the semantics—in pursuit of a sound static analysis
algorithm—we pass through a big-step collecting evaluation and reachability
semantics, notated «ρ,τ⊢e,∿{σ}⇓∿{v},∿{σ}» and «ρ,τ⊢e,∿{σ}⇑∿{ς}» and shown in
Figure~\ref{f:lamif-collecting}, where «∿{v}», «∿{σ}» and «∿{ς}» range over
collecting state spaces:
\begin{alignat*}{3}
                ∿{v} ∈ &&\mathrel{} ∿{val}      &\mathrel{} ≔ ℘(val) 
\\[\mathgobble] ∿{σ} ∈ &&\mathrel{} ∿{store}    &\mathrel{} ≔ addr ↦ ∿{val} 
\\[\mathgobble] ∿{ς} ∈ &&\mathrel{} ∿{config}   &\mathrel{} ≔ exp × env × ∿{store} × time
\end{alignat*}
and the denotation for binary operators («⟦b⟧») is lifted to a collecting
denotation operator «∿{⟦b⟧}»:
\[ ∿{⟦b⟧}(∿{v}₁,∿{v}₂) ≔ ❴⟦b⟧(v₁,v₂) ∣ v₁ ∈ ∿{v}₁ ∧ v₂ ∈ ∿{v}₂❵ \]

The big-step collecting and reachability relations are structurally similar to
the concrete semantics. The primary differences are the use of set containment
(«∈») in place of equality («=») when branching on application and conditional
expressions.

The big-step collecting reachability semantics is a sound approximation of the
big-step concrete reachability semantics:
\begin{theorem}[Collecting Reachability Semantics Soundness]
\begin{alignat*}{2}
                & ⦑If⦒    &&\quad ρ,τ⊢e,σ⇑⟨e′,ρ′,σ′,τ′⟩ \quad ⦑and⦒ \quad ρ′,τ′⊢e′,σ′⇓v,σ″
\\[\mathgobble] & ⦑where⦒ &&\quad η(σ) ⊑ ∿{σ}
\\[\mathgobble] & ⦑then⦒  &&\quad ρ,τ⊢e,∿{σ}⇑⟨e′,ρ′,∿{σ}′,τ′⟩ \quad ⦑and⦒ \quad ρ′,τ′⊢e,∿{σ}′⇓∿{v},∿{σ}″ 
\\[\mathgobble] & ⦑where⦒ &&\quad η(σ′) ⊑ ∿{σ}′ \quad ⦑and⦒ \quad v ∈ ∿{v} \quad ⦑and⦒ \quad η(σ″) ⊑ ∿{σ}″
\end{alignat*}
\end{theorem}
The proof is by induction on the concrete big-step derivation. The extraction
function «η» is defined separately for stores («σ») and configurations («ς»):
\begin{alignat*}{1}
   & η(σ)(ℓ) ≔ ❴σ(ℓ)❵ \quad\quad η(⟨e,ρ,σ,τ⟩) ≔ ⟨e,ρ,η(σ),τ⟩
\end{alignat*}
and the partial ordering on stores and configurations is pointwise:
\begin{alignat*}{1}
  & ∿{σ}₁ ⊑ ∿{σ}₂ \quad \mathrel{⦑\emph{iff}⦒} \quad ∀ℓ.\mathrel{} ∿{σ}₁(ℓ) ⊆ ∿{σ}₂(ℓ)
  \\[\mathgobble] & ⟨e₁,ρ₁,∿{σ}₁,τ₁⟩ ⊑ ⟨e₂,ρ₂,∿{σ}₂,τ₂⟩ \quad \mathrel{⦑\emph{iff}⦒} 
  \\[\mathgobble] & \hspace{2em} e₁ = e₂ ∧ ρ₁ = ρ₂ ∧ ∿{σ}₁ ⊑ ∿{σ}₂ ∧ τ₁ = τ₂
\end{alignat*}

\paragraph{Finite Abstraction}

\begin{figure*} %{-{
\begin{flushright}[\emph{Abstract Evaluation}]\quad\fbox{«♯{ρ},♯{τ}⊢e,♯{σ}⇓♯{v},♯{σ}′»}\end{flushright}
\vspace{-0.75em}
\begin{mathpar}
  \inferrule*[left=(ALit)]{ }{♯{ρ},♯{τ}⊢n,♯{σ}⇓♯{η}(n),♯{σ}}

  \inferrule*[left=(AVar)]{ }{♯{ρ},♯{τ}⊢x,♯{σ}⇓♯{σ}(♯{ρ}(x)),♯{σ}}

  \inferrule*[left=(ALam)]{ }{♯{ρ},♯{τ}⊢λx.e,♯{σ}⇓♯{η}(⟨λx.e,♯{ρ}⟩),♯{σ}}\vspace{-0.75em}

    \inferrule*[left=(ABin)]
    {  ♯{ρ},♯{τ}⊢e₁,♯{σ}⇓♯{v}₁,♯{σ}₁
    \\ ♯{ρ},♯{τ}⊢e₂,♯{σ}₁⇓♯{v}₂,♯{σ}₂}
    {  ♯{ρ},♯{τ}⊢b(e₁,e₂),♯{σ}⇓♯{⟦b⟧}(♯{v}₁,♯{v}₂),♯{σ}₂}\vspace{-0.75em}

  \inferrule*[left=(AApp),right={\begin{minipage}{2em}\ssmall
    \begin{alignat*}{1}
    \begin{alignedat}{2} 
           ⟨λx.e′,♯{ρ}′⟩ ∈ &\mathrel{} ⌊γ⌋⸤clo⸥(♯{v}₁) \\[-0.5em]
                   ♯{ς}  = &\mathrel{} ⟨e₁(e₂),♯{ρ},♯{σ},♯{τ}⟩ \\[-0.5em]
                   ♯{ℓ}  = &\mathrel{} ⟨x,♯{τ}′⟩ \\[-0.5em]
                   ♯{τ}′ = &\mathrel{} ♯{next}(♯{τ},♯{ς})
      \end{alignedat}
    \end{alignat*}
  \end{minipage}}]
  {  ♯{ρ}      ,♯{τ} ⊢e₁    ,♯{σ} ⇓♯{v}₁,♯{σ}₁
  \\ ♯{ρ}      ,♯{τ} ⊢e₂    ,♯{σ}₁⇓♯{v}₂,♯{σ}₂
  \\ ♯{ρ}′[x↦♯{ℓ}],♯{τ}′⊢e′    ,♯{σ}₂⊔[♯{ℓ}↦♯{v}₂]⇓♯{v}′,♯{σ}₃}
  {♯{ρ}      ,♯{τ} ⊢e₁(e₂),♯{σ} ⇓♯{v}′,♯{σ}₃}\vspace{-0.75em}

  \inferrule*[left=(AIfT),right={\ssmall «0∈⌊γ⌋⸤0⸥(♯{v}₁)»}]
  {  ♯{ρ},♯{τ}⊢e₁,♯{σ}⇓♯{v}₁,♯{σ}₁
  \\ ♯{ρ},♯{τ}⊢e₂,♯{σ}₁⇓♯{v},♯{σ}₂}
  {  ♯{ρ},♯{τ}⊢⟬if0⟭(e₁)❴e₂❵❴e₃❵,♯{σ}⇓♯{v},♯{σ}₂}
  \;
  \inferrule*[left=(AIfF),right={\ssmall «¬0∈⌊γ⌋⸤¬0⸥(♯{v}₁)»}]
  {  ♯{ρ},♯{τ}⊢e₁,♯{σ}⇓♯{v}₁,♯{σ}₁
  \\ ♯{ρ},♯{τ}⊢e₃,♯{σ}₁⇓♯{v},♯{σ}₂}
  {  ♯{ρ},♯{τ}⊢⟬if0⟭(e₁)❴e₂❵❴e₃❵,♯{σ}⇓♯{v},♯{σ}₂}
\end{mathpar}
\begin{flushright}[\emph{Abstract Reachability}]\quad\fbox{«♯{ρ},♯{τ}⊢e,♯{σ}⇑♯{ς}»}\end{flushright}
\vspace{-0.25em}
\begin{mathpar}
  \inferrule*[left=(ARefl)]{ }{♯{ρ},♯{τ}⊢e,♯{σ}⇑⟨e,♯{ρ},♯{σ},♯{τ}⟩}

  \inferrule*[left=(ARBin1)]
  {♯{ρ},♯{τ}⊢e₁,♯{σ}⇑ς}
  {♯{ρ},♯{τ}⊢b(e₁,e₂),♯{σ}⇑♯{ς}}

  \inferrule*[left=(ARBin2)]
  {  ♯{ρ},♯{τ}⊢e₁,♯{σ}⇓♯{v}₁,♯{σ}₁
  \\ ♯{ρ},♯{τ}⊢e₂,♯{σ}₁⇑♯{ς}}
  {♯{ρ},♯{τ}⊢b(e₁,e₂),♯{σ}⇑♯{ς}}\vspace{-0.75em}

  \inferrule*[left=(ARApp1)]
  {♯{ρ},♯{τ}⊢e₁,♯{σ}⇑♯{ς}}
  {♯{ρ},♯{τ}⊢e₁(e₂),♯{σ}⇑♯{ς}}

  \inferrule*[left=(ARApp2),right={\ssmall «⟨λx.e′,♯{ρ}′⟩∈⌊γ⌋⸤clo⸥(♯{v}₁)»}]
  {  ♯{ρ},♯{τ}⊢e₁,♯{σ}⇓♯{v}₁,♯{σ}₁
  \\ ♯{ρ},♯{τ}⊢e₂,♯{σ}₁⇑♯{ς}}
  {♯{ρ},♯{τ}⊢e₁(e₂),♯{σ}⇑♯{ς}}\vspace{-0.75em}

   \inferrule*[left=(ARApp3),right={\begin{minipage}{2em}\ssmall
       \begin{alignat*}{1}
       \begin{alignedat}{2} 
         ⟨λx.e′,♯{ρ}′⟩ ∈ &\mathrel{} ⌊γ⌋⸤clo⸥(♯{v}₁) \\[-0.5em]
            ♯{ς} = &\mathrel{} ⟨e₁(e₂),♯{ρ},♯{σ},♯{τ}⟩ \\[-0.5em]
           ♯{ℓ}  = &\mathrel{} ⟨x,♯{τ}′⟩ \\[-0.5em]
           ♯{τ}′ = &\mathrel{} ♯{next}(♯{τ},♯{ς})
         \end{alignedat}
       \end{alignat*}
     \end{minipage}}]
     {  ♯{ρ},♯{τ}⊢e₁,♯{σ} ⇓♯{v}₁,♯{σ}₁
       \\ ♯{ρ},♯{τ}⊢e₂,♯{σ}₁⇓♯{v}₂,♯{σ}₂
     \\ ♯{ρ}′[x↦♯{ℓ}],♯{τ}′⊢e′,♯{σ}₂⊔[♯{ℓ}↦♯{v}₂]⇑♯{ς}}
     {♯{ρ},♯{τ}⊢e₁(e₂),♯{σ}⇑♯{ς}}\vspace{-0.75em}

  \inferrule*[left=(ARIf1)]
  {♯{ρ},♯{τ}⊢e₁,♯{σ}⇑♯{ς}}
  {♯{ρ},♯{τ}⊢⟬if0⟭(e₁)❴e₂❵❴e₃❵,♯{σ}⇑♯{ς}}

  \inferrule*[left=(ARIfT),right={\ssmall «0∈⌊γ⌋⸤0⸥(♯{v}₁)»}]
  {  ♯{ρ},♯{τ}⊢e₁,♯{σ}⇓♯{v}₁,♯{σ}₁
  \\ ♯{ρ},♯{τ}⊢e₂,♯{σ}₁⇑♯{ς}}
  {♯{ρ},♯{τ}⊢⟬if0⟭(e₁)❴e₂❵❴e₃❵,♯{σ}⇑♯{ς}}\vspace{-0.75em}

  \inferrule*[left=(ARIfF),right={\ssmall «¬0∈⌊γ⌋⸤¬0⸥(♯{v}₁)»}]
    {  ♯{ρ},♯{τ}⊢e₁,♯{σ}⇓♯{v}₁,♯{σ}₁
    \\ ♯{ρ},♯{τ}⊢e₃,♯{σ}₁⇑♯{ς}}
    {♯{ρ},♯{τ}⊢⟬if0⟭(e₁)❴e₂❵❴e₃❵,♯{σ}⇑♯{ς}}
\end{mathpar}
\caption{Big-step Abstract Evaluation and Reachability Semantics}
\label{f:lamif-abstract}
\end{figure*} %}-}

The next step towards a computable static analysis is an abstract semantics
with a finite state space that approximates the big-step collecting semantics,
notated «♯{ρ},♯{τ}⊢e,♯{σ}⇓♯{v},♯{σ}» and «♯{ρ},♯{τ}⊢e,♯{σ}⇑♯{ς}» and shown in
Figure~\ref{f:lamif-abstract}, where «♯{ρ}», «♯{τ}», «♯{v}», «♯{σ}» and «♯{ς}»
are finite abstractions of their collecting counterparts:
\begin{alignat*}{3}
                ♯{ρ} ∈ &&\mathrel{} ♯{env}    &\mathrel{} ≔ var ↦ ♯{addr}⸤⊥⸥ 
\\[\mathgobble] ♯{ℓ} ∈ &&\mathrel{} ♯{addr}   &\mathrel{} ≔ var × ♯{time} 
\\[\mathgobble] ♯{τ} ∈ &&\mathrel{} ♯{time}   &\mathrel{} ≔ … 
\\[\mathgobble] ♯{v} ∈ &&\mathrel{} ♯{val}    &\mathrel{} ≔ … 
\\[\mathgobble] ♯{σ} ∈ &&\mathrel{} ♯{store}  &\mathrel{} ≔ ♯{addr} ↦ ♯{val} 
\\[\mathgobble] ♯{ς} ∈ &&\mathrel{} ♯{config} &\mathrel{} ≔ exp × ♯{env} × ♯{store} × ♯{time}
\end{alignat*}
The primary structural difference from the collecting semantics is the use of
join when updating the store («♯{σ}⊔[♯{ℓ}↦♯{v}]») rather than strict
replacement («∿{σ}[ℓ↦∿{v}]»). This is to preserve soundness in the presence of
address reuse, which occurs from the finite size of the address space.

The abstract denotation («♯{⟦b⟧}») is any over-approximation of the collecting
denotation («∿{⟦b⟧}») w.r.t. a Galois connection «∿{val}⇄{α}{γ}♯{val}»:
\[ ♯{⟦b⟧}(♯{v}₁,♯{v}₂) ⊒ α(∿{⟦b⟧}(γ(♯{v}₁),γ(♯{v}₂))) \]
Concretization functions «⌊γ⌋⸤clo⸥», «⌊γ⌋⸤0⸥» and «⌊γ⌋⸤¬0⸥» are computable
finite subsets of the full concretization function «γ» s.t.:
\begin{alignat*}{2}
                ⌊γ⌋⸤clo⸥(♯{v}) ≔ &\mathrel{} ❴⟨λx.e,♯{ρ}⟩ ∣ ⟨λx.e,♯{ρ}⟩ ∈ γ(♯{v})❵ 
\\[\mathgobble] ⌊γ⌋⸤0⸥(♯{v})   ≔ &\mathrel{} ❴0 ∣ 0 ∈ γ(♯{v})❵ 
\\[\mathgobble] ⌊γ⌋⸤¬0⸥(♯{v})  ≔ &\mathrel{} ❴¬0 ∣ n ∈ γ(♯{v}) ∧ n≠0❵
\end{alignat*}
Abstract sets «♯{time}» and «♯{val}» are left as parameters to the analysis
along with their operations «♯{next}», «♯{⟦b⟧}», «⌊γ⌋⸤clo⸥», «⌊γ⌋⸤0⸥»,
«⌊γ⌋⸤¬0⸥» and «⊔⸢♯{val}⸣».

The abstract semantics is a sound approximation of the collecting semantics,
which we establish through the theorem:
\begin{theorem}[Abstract Reachability Semantics Soundness]
\begin{alignat*}{2}
               & ⦑If⦒    &&\quad ρ,τ⊢e,∿{σ}⇑⟨e′,ρ′,∿{σ}′,τ′⟩ \quad ⦑and⦒ \quad ρ′,τ′⊢e′,∿{σ}′⇓∿{v},∿{σ}″
\\[\mathgobble]& ⦑where⦒ &&\quad η(ρ) ⊑ ♯{ρ} \quad ⦑and⦒ \quad η(τ) ⊑ ♯{τ} \quad ⦑and⦒ \quad η(∿{σ}) ⊑ ♯{σ}
\\[\mathgobble]& ⦑then⦒  &&\quad ♯{ρ},♯{τ}⊢e,♯{σ}⇑⟨e′,♯{ρ}′,♯{σ}′,♯{τ}′⟩ \quad ⦑and⦒ \quad ♯{ρ}′,♯{τ}′⊢e,♯{σ}′⇓♯{v},♯{σ}″ 
\\[\mathgobble]& ⦑where⦒ &&\quad η(ρ′) ⊑ ♯{ρ}′⦑,⦒\ η(τ′) ⊑ ♯{τ}′⦑,⦒\ η(∿{σ}′) ⊑ ♯{σ}′⦑,⦒\ v ∈ ∿{v}⦑,⦒\ η(σ″) ⊑ ∿{σ}″
\end{alignat*}
\end{theorem}
The proof is by induction on the big-step derivation. The extraction function
«η» is defined separately for environments («ρ»), time («τ»), collecting stores
(«∿{σ}»), values («∿{v}») and configurations («∿{ς}»). «η(τ)» and «η(∿{v})» are
given with parameters «♯{time}» and «♯{val}». «η(ρ)», «η(∿{σ})» and «η(∿{ς})»
are defined pointwise:
\begin{alignat*}{1}
  & η(ρ)(x) ≔ η(ρ(x)) \quad\quad η(∿{σ})(♯{ℓ}) ≔ ⨆⸤ℓ ∈ γ(♯{ℓ})⸥η(∿{σ}(ℓ)) 
\\[\mathgobble] & η(⟨e,ρ,τ,∿{σ}⟩) ≔ ⟨e,η(ρ),η(τ),η(∿{σ})⟩
\end{alignat*}

\paragraph{Computing the Analysis}

An analysis for the program «e₀» w.r.t. the abstract semantics is some cache
«\$ ∈ ♯{config} ↦ ℘(♯{val} × ♯{store})» that maps all configurations reachable
from the initial configuration «⟨e₀,♯{ρ}₀,♯{σ}₀,♯{τ}₀⟩» to their final values
and stores «♯{v},♯{σ}», which we notate «\$ ⊨ e₀»:
\begin{alignat*}{2}
  \$ ⊨ e₀ \quad\quad \mathrel{⦑\emph{iff}⦒} \quad\quad & 
    \begin{alignedat}{2}
                   & ⦑\emph{If}⦒     &&\quad ♯{ρ}₀,♯{τ}₀⊢e₀,♯{σ₀}⇑⟨e,♯{ρ},♯{σ},♯{τ}⟩ 
    \\[\mathgobble]& ⦑\emph{and}⦒    &&\quad ♯{ρ},♯{τ}⊢e,♯{σ}⇓♯{v},♯{σ}′  
    \\[\mathgobble]& ⦑\emph{then}⦒   &&\quad ⟨♯{v},♯{σ}′⟩ ∈ \$(⟨e,♯{ρ},♯{σ},♯{τ}⟩)
      \end{alignedat}
\end{alignat*}
The \emph{best} cache «\$⁺» is then computed as the least fixed point of the
functional «ℱ»:
\begin{alignat*}{1}
  & ℱ ∈ (♯{config} ↦ ℘(♯{val}×♯{store})) → (♯{config} ↦ ℘(♯{val}×♯{store})) 
\\[\mathgobble] & ℱ ≔ λ\$.  
\\[\mathgobble] &  \hspace{1em} ⨆⸤⟨e,♯{ρ},♯{σ},♯{τ}⟩∈\$⸥ \begin{cases}
         ❴ ⟨e,♯{ρ},♯{σ},♯{τ}⟩ ↦ ❴⟨♯{v},♯{σ}′⟩❵ ∣ ♯{ρ},♯{τ}⊢e,♯{σ}⇓⸢\$⸣♯{v},♯{σ}′ ❵ 
      \\[\mathgobble] ❴ ♯{ς} ↦ ❴❵ ∣ ♯{ρ},♯{τ}⊢e,♯{σ}⇑⸢\$⸣♯{ς}❵
   \end{cases}
\end{alignat*}
which also includes the initial configuration:
\[ \$⁺ ≔ ⦑\emph{lfp}⦒ (λ\$.\mathrel{} ℱ(\$) ⊔ ❴⟨e₀,η(ρ₀),η(σ₀),η(τ₀)⟩ ↦ ❴❵❵) \]
The relations «♯{ρ},♯{τ}⊢e,♯{σ}⇓⸢\$⸣♯{v},♯{σ}′» and «♯{ρ},♯{τ}⊢e,♯{σ}⇑⸢\$⸣♯{ς}»
are modified versions of the original abstract semantics, but with recursive
judgements replaced by «⟨♯{v},♯{σ}′⟩ ∈ \$(e,♯{ρ},♯{σ},♯{τ})» and «♯{ς} ∈
\$(e,♯{ρ},♯{σ},♯{τ})» respectively. Therefore «ℱ» is not recursive; the
recursion in the relations is lifted to the outer fixpoint of the analysis.
Because the state space «♯{config} ↦ ℘(♯{val}×♯{store})» is finite and «ℱ» is
monotonic, «\$⁺» can be computed algorithmically in finite time by Kleene
fixed-point iteration. See Nielson et al~\cite{dvanhorn:Neilson:1999} for more
background and examples of static analyzers computed in this style, and from
which the current development was largely inspired.
\begin{theorem}[Algorithm Correctness]
  «\$⁺» is a valid analysis for «e₀», that is: «\$⁺ ⊨ e₀».
\end{theorem}
The proof is by induction on the assumed derivations
«♯{ρ}₀,♯{τ}₀⊢e₀,♯{σ}₀⇑⟨♯{e},♯{ρ},♯{σ},♯{τ}⟩» and «♯{ρ},♯{τ}⊢e,♯{σ}⇓♯{v},♯{σ}′»,
and utilizes the fact that «\$⁺» is a fixed point, that is: «ℱ(\$⁺) = \$⁺». Our
final theorem relates the analysis cache «\$⁺» back to the concrete semantics
of the initial program as a sound approximation:
\begin{theorem}[Algorithm Soundness]
\begin{alignat*}{2}
                &⦑If ⦒  &&\quad ρ₀,τ₀⊢e₀,σ₀⇑⟨e,ρ,σ,τ⟩ \quad ⦑and⦒ \quad ρ,τ⊢e,σ⇓v,σ′  
\\[\mathgobble] &⦑then⦒ &&\quad ⟨♯{v},♯{σ}′⟩ ∈ \$⁺(⟨e,♯{ρ},♯{σ},♯{τ}⟩)  
\\[\mathgobble] &⦑where⦒&&\quad η(ρ) ⊑ ♯{ρ}⦑,⦒\ η(τ) ⊑ ♯{τ}⦑,⦒\ η(σ) ⊑ ♯{σ}⦑,⦒\ η(v) ⊑ ♯{v}⦑,⦒\ η(σ′) ⊑ ♯{σ}′
\end{alignat*}
\end{theorem}
The proof follows by composing Theorems~1-4.

\paragraph{Computing with Definitional Interpreters}

The algorithm described in Section~\ref{s:cache} is a more efficient strategy
for computing «\$⁺» using an extensible open-recursive definitional
interpreter. This technique is general, and bridges the gap between the
big-step abstract semantics formalized in this section and the definitional
interpreters we wish to execute to obtain analyses.

An extensible open-recursive definitional interpreter for \lamif (the small
language formalized in this section) has domain:
\begin{alignat*}{1}
  & ℰ ∈ Σ → Σ \quad ⦑\emph{where}⦒ \quad Σ ≔ ♯{config} → ℘(♯{val}×♯{store})
\end{alignat*}
and is defined such that its denotational-fixpoint («Y(ℰ)») recovers concrete
interpretation when instantiated with the concrete state-space. For example,
the recursive case for binary operator expressions is defined:
\begin{alignat*}{1}
  & ℰ(ℰ′)(⟨b(e₁,e₂),♯{ρ},♯{σ},♯{τ}) ≔  
\\[\mathgobble] & \hspace{1em} ❴\mathrel{} ♯{⟦b⟧}(♯{v}₁,♯{v₂}) 
\\[\mathgobble] & \hspace{1em} ∣\mathrel{} ⟨♯{v}₁,♯{σ}₁⟩ ∈ ℰ′(⟨e₁,♯{ρ},♯{σ},♯{τ}⟩) ∧ ⟨♯{v}₂,♯{σ}₂⟩ ∈ ℰ′(⟨e₂,♯{ρ},♯{σ}₁,♯{τ}⟩) ❵
\end{alignat*}
The iteration strategy to analyze the program «e₀» is then to run «e₀» using
«ℰ», but intercepting recursive calls to:
\begin{enumerate}
  \item Cache results for all intermediate configurations «♯{ς}»; and
  \item Cache seen states to prevent infinite loops.
\end{enumerate}
(1) is required to fulfill the specification that «\$⁺» include results for all
reachable configurations from «e₀», and (2) is required to reach a fixed point
of the analysis. To track this extra information we add functional state to the
interpreter (which was done through a monad transformer in
Section~\ref{s:cache}) of type:
\[ ♯{cache} ≔ ♯{config} ↦ ℘(♯{val}×♯{store}) \]
such that the open-recursive evaluator has type:
\begin{alignat*}{1}
  & ℰ ∈ Σ → Σ \quad ⦑\emph{where}⦒ 
\\[\mathgobble] & \hspace{1em} Σ ≔ ♯{config}×♯{cache} → ℘(♯{val}×♯{store})×♯{cache}
\end{alignat*}
The iteration to compute «\$⁺» given «ℰ» is then defined:
\begin{alignat*}{1}
  & \hspace{0em} \$⁺ ≔ ⦑\emph{lfp}⦒(λ\$ᵒ. 
\\[\mathgobble] & \hspace{1em} \mathrel{⟬let⟭} ℰ⋆ ≔ Y(λℰ′.\mathrel{} ℰ(λ⟨♯{ς},\$ⁱ⟩. 
\\[\mathgobble] & \hspace{3em}      \mathrel{⟬if⟭} ♯{ς} ∈ \$ⁱ \mathrel{⟬then⟭} ⟨\$ⁱ(♯{ς}),\$ⁱ⟩ \mathrel{⟬else⟭} 
\\[\mathgobble] & \hspace{3em}      \mathrel{⟬let⟭} ⟨♯{VS},\$⸢i\prime⸣⟩ ≔ ℰ′(♯{ς},\$ⁱ[♯{ς}↦\$ᵒ(♯{ς})]) 
\\[\mathgobble] & \hspace{3em}      \mathrel{⟬in⟭} ⟨♯{VS},\$⸢i\prime⸣[♯{ς}↦♯{VS}]⟩)) 
\\[\mathgobble] & \hspace{1em} \mathrel{⟬in⟭} π₂(ℰ⋆(⟨e₀,♯{ρ}₀,♯{σ}₀,♯{τ}₀⟩,❴❵)))
\end{alignat*}
The fixed interpreter «ℰ⋆» calls the unfixed interpreter «ℰ», but intercepts
recursive calls to perform (1) and (2) described above. When loops are
detected, the results from the previous complete result «\$ᵒ» is used, and the
outer fixpoint computes the least fixed point of this «\$ᵒ».

The end result is that, rather than compute analysis results and reachable
states naively with Kleene fixpoint iteration, we are able to reuse the
standard definitional interpreter—written in open-recursive form—to
simultaneously explore reachable states, cache intermediate configurations, and
iterate towards a least fixpoint solution for the analysis. This method is more
efficient, and reuses an extensible definitional interpreter which can recover
a wide range of analyses, including concrete interpretation.

\paragraph{Widening}

Two forms of widening can be employed to the semantics and iteration algorithm
to achieve acceptable performance for the abstract interpreter.

The first form of widening is to widen the store in the result set
«℘(♯{val}×♯{store})» to «℘(♯{val})×♯{store}» in the evaluator «ℰ»:
\begin{alignat*}{1}
  & ℰ ∈ Σ → Σ \quad ⦑\emph{where}⦒ 
\\[\mathgobble] & \hspace{1em} Σ ≔ ♯{config} × ♯{cache} → ℘(♯{val})×♯{store}×♯{cache}
\end{alignat*}
We perform this widening systematically and with no added effort through the
use of Galois Transformers~\cite{local:darais-oopsla2015} in
Section~\ref{s:widening}. The iteration strategy for this widened state space
is the same as before, which computes a fixed point of the outer cache «\$ᵒ».

The next form of widening is to pull the store out of the configuration space
\emph{entirely}, that is:
\begin{alignat*}{2}
  ♯{ς} ∈ &\mathrel{} ♯{config} ≔ exp × ♯{env} × ♯{time} 
\\[\mathgobble] \$ ∈ &\mathrel{} ♯{cache} ≔ ♯{config} ↦ ℘(♯{val})
\end{alignat*}
and:
\begin{alignat*}{1}
  & ℰ ∈ Σ → Σ \quad ⦑\emph{where}⦒ 
\\[\mathgobble] & \hspace{1em} Σ ≔ ♯{config}×♯{store}×♯{cache} → ℘(♯{val})×♯{store}×♯{cache}
\end{alignat*}
The fixed point iteration then finds a mutual least fixed-point of both the
outer cache «\$ᵒ» \emph{and} the store «♯{σ}»:
\begin{alignat*}{1}
  & \hspace{0em} ⟨\$⁺,♯{σ}⁺⟩ ≔ ⦑\emph{lfp}⦒(λ⟨\$ᵒ,♯{σ}⟩. 
\\[\mathgobble] & \hspace{1em} \mathrel{⟬let⟭} ℰ⋆ ≔ Y(λℰ′. \mathrel{} ℰ(λ⟨♯{ς},♯{σ}ⁱ,\$ⁱ⟩. 
\\[\mathgobble] & \hspace{3em}      \mathrel{⟬if⟭} ♯{ς} ∈ \$ⁱ \mathrel{⟬then⟭} ⟨\$ⁱ(♯{ς}),σⁱ,\$ⁱ⟩ \mathrel{⟬else⟭} 
\\[\mathgobble] & \hspace{3em}      \mathrel{⟬let⟭} ⟨♯{V},♯{σ}⸢i\prime⸣,\$⸢i\prime⸣⟩ ≔ ℰ′(♯{ς},♯{σ}ⁱ,\$ⁱ[♯{ς}↦\$ᵒ(♯{ς})]) 
\\[\mathgobble] & \hspace{3em}      \mathrel{⟬in⟭} ⟨♯{V},♯{σ}⸢i\prime⸣,\$⸢i\prime⸣[♯{ς}↦♯{V}]⟩)) 
\\[\mathgobble] & \hspace{1em} \mathrel{⟬in⟭} π⸤2×3⸥(ℰ⋆(⟨e₀,♯{ρ}₀,♯{τ}₀⟩,♯{σ},❴❵)))
\end{alignat*}
This second version of widening, which computes a fixpoint also over the store,
recovers a so-called \emph{flow-insensitive} analysis. In this model, all
program states are re-analyzed in the store resulting from execution. Also, the
cache («\$») does not index over store states «♯{σ}» in its domain, greatly
reducing its size, and leading to a much more efficient (although less precise)
static analyzer.

\paragraph{Recovering Classical 0CFA}

From the fully widened static analyzer, which computes a mutual fixpoint
between a cache and store, we can easily recover a classical 0CFA analysis. We
do this by instantiating «♯{time}» to the singleton abstraction «❴•❵», as was
shown in Section~\ref{s:interp}. In this setting, the lexical environment «ρ»
is uniquely determined by the program expression «e», and can therefore be
eliminated, resulting in the analysis state space:
\begin{alignat*}{2}
  ♯{ς} ∈ &\mathrel{} ♯{config} ≔ exp 
\\[\mathgobble] \$ ∈ &\mathrel{} ♯{cache} ≔ exp ↦ ℘(♯{val}) 
\\[\mathgobble] ♯{σ} ∈ &\mathrel{} ♯{store} ≔ var ↦ ℘(♯{val})
\end{alignat*}
The specification for the analysis and the fully store-widened least
fixed-point iteration for computing it recovers the constraint-based
description of 0CFA given by Nielson \emph{et al}
in~\cite{dvanhorn:Neilson:1999}, where 0CFA is defined as the smallest cache
(«\$») and store («σ») which satisfy a co-inductively defined judgment: «\$,σ ⊨
e».

\paragraph{Recovering Pushdown Analysis}

We borrow from the recent result in pushdown analysis by Gilray \emph{et
al}~\cite{local:p4f} which shows that full pushdown precision can be achieved
in a small-step store-widened abstract semantics by allocating continuations
using a particular address space: program expressions paired with abstract
environments («⟨e,♯{ρ}⟩»). In other words, «⟨e,♯{ρ}⟩» is sufficient to achieve
full pushdown precision because the tuple uniquely identifies the evaluation
context up to the final result of evaluation.

Our fully widened semantics recovers pushdown precision because the cache maps
tuples «⟨e,♯{ρ},♯{τ}⟩», which contains «⟨e,♯{ρ}⟩». We then see that abstract
time «♯{τ}» is redundant and eliminate it from the cache, resulting in a
smaller domain for the same analysis:
\begin{alignat*}{2}
  ♯{ς} ∈ &\mathrel{} ♯{config} ≔ exp × ♯{env} × ♯{time} 
\\[\mathgobble] \$ ∈ &\mathrel{} ♯{cache} ≔ exp × ♯{env} ↦ ℘(♯{val}) 
\\[\mathgobble] ♯{σ} ∈ &\mathrel{} ♯{store} ≔ var × ♯{addr} ↦ ℘(♯{val})
\end{alignat*}
An advantage of our setting is that we recover pushdown analysis also for
varying degrees of store-widening, which is not the case in Gilray \emph{et
al}~\cite{local:p4f}, although pushdown precision for non-widened semantics has
been achieved by Johnson and Van Horn~\cite{dvanhorn:Johnson2014Abstracting}.
Furthermore, the implementation of our analyzer inherits this precision through
precise call-return matching in the defining metalanguage, requiring no added
instrumentation to the state-space of the analyzer.

Going back to Nielson et al~\cite{dvanhorn:Neilson:1999}, it would be
interesting to redevelop their constraint-based analysis descriptions of kCFA
in a form that recovers pushdown precision. Such an exercise would amount to
translating our big-step abstract semantics instantiated to kCFA to a
constraint system. The resulting system would differ from classical kCFA by the
addition of environments «♯{ρ}» (which Nielson et al call context environments)
to the domain of the cache. In this way our formal framework is able to bridge
the gap between results in pushdown analysis described via small-step machines
\emph{a la} Van Horn and Might~\cite{dvanhorn:VanHorn2010Abstracting}, and
constraint-based systems \emph{a la} Nielson et al for which pushdown analysis
has yet to be described effectively.


\end{document}

%\lstset
% {language=Lisp
% ,extendedchars=true
% ,inputencoding=utf8
% ,mathescape=true
% ,escapechar=`
% ,upquote=true
% ,basicstyle=\ttfamily\color{ParenColor}
% ,commentstyle=\color{CommentColor}
% ,alsoletter=+-*/?'\#0123456789^
% ,identifierstyle=\color{IdentifierColor}
% ,emph=%
%  {return,bind
%  ,zero?
%  ,ask-env,local-env
%  ,ext,find,alloc
%  ,get-store,put-store,update-store
%  ,tell
%  ,get-dead,put-dead,update-dead
%  ,fail,mplus
%  ,get-,put-,update-
%  ,ask-,local-
%  }
% ,emphstyle=\color{IdentifierColor}\emph
% ,classoffset=1
% ,keywords={'add1,'sub1,'+,'-,'*,'/,'failure,'N,\#t,\#f,0,1,2,3,4,5,6,7,8,9}
% ,keywordstyle=\color{ValueColor}
% }

