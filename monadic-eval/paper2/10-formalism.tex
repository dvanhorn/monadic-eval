\newcommand{\lamif}{«λ⦑IF⦒» }

\section{Formalism}\label{s:formalism}

In this section we formalize our approach to designing static analyzers via
definitional interpreters. The development begins with a ``ground truth''
big-step semantics and concludes with the fixpoint iteration strategy described
in Section~\ref{s:cache}, which we prove sound w.r.t. a synthesized
abstract semantics. The design is systematic, and applies to arbitrary
programming languages described via big-step operational semantics. We
demonstrate the systematic process as applied to a subset of the language
described in Figure~\ref{f:syntax}, which we call \lamif.

\paragraph{Concrete Semantics}

We begin with the concrete semantics of \lamif as a big-step relation
«ρ,τ⊢e,σ⇓v,σ′», shown in Figures~\ref{f:lamif-syntax}
and~\ref{f:lamif-concrete}. The definition is mostly standard: «ρ» and «σ» are
the environment and store, «e» is the initial expression, and «v» is the
resulting value. The argument «τ» represents ``time,'' which when abstracted
supports modeling execution contexts like call-site sensitivity. Concretely
time is modelled as a natural number and represents the number of steps of
execution.

\begin{figure} %{-{
\begin{alignat*}{3}
                n ∈ &&\mathrel{} lits \mathrel{\hphantom{≔}} & 
\\[\mathgobble] x ∈ &&\mathrel{} vars \mathrel{\hphantom{≔}} & 
\\[\mathgobble] b ∈ &&\mathrel{}  binop ≔ &\mathrel{} ❴⟬plus⟭, …❵ 
\\[\mathgobble] e ∈ &&\mathrel{}    exp ⩴ &\mathrel{} n ∣ x ∣ λx.e ∣ e(e) ∣ ⟬if0⟭(e)❴e❵❴e❵ ∣ b(e,e) 
\\[\mathgobble] ρ ∈ &&\mathrel{}    env ≔ &\mathrel{} var → addr⸤⊥⸥ 
\\[\mathgobble] σ ∈ &&\mathrel{}  store ≔ &\mathrel{} addr → val⸤⊥⸥ 
\\[\mathgobble] v ∈ &&\mathrel{}    val ⩴ &\mathrel{} n ∣ ⟨λx.e,ρ⟩ 
\\[\mathgobble] ℓ ∈ &&\mathrel{}   addr ≔ &\mathrel{} var × time 
\\[\mathgobble] τ ∈ &&\mathrel{}   time ≔ &\mathrel{} ℕ
\end{alignat*}
\caption{\lamif{} Syntactic Categories}
\label{f:lamif-syntax}
\end{figure} %}-}

\begin{figure} %{-{
\begin{mathpar}

  \inferrule*[left=(Lit)]{ }{ρ , τ ⊢ n , σ ⇓ n , σ}

  \inferrule*[left=(Var)]{ }{ρ , τ ⊢ x , σ ⇓ σ(ρ(x)) , σ}

  \inferrule*[left=(Lam)]{ }{ρ , τ ⊢ λx.e , σ ⇓ ⟨λx.e,ρ⟩ , σ}

  \inferrule*[left=(App),right={\begin{minipage}{2em}\ssmall
    \begin{alignat*}{1}
    \begin{alignedat}{2} 
⟨λx.e′,ρ′⟩ = &\mathrel{} v₁ \\[-0.5em]
        τ′ = &\mathrel{} τ+1 \\[-0.5em]
        ℓ  = &\mathrel{} ⟨x,τ′⟩ \\
      \end{alignedat}
    \end{alignat*}
  \end{minipage}}]{
  ρ       , τ  ⊢ e₁    , σ        ⇓ v₁ , σ₁ \\
  ρ       , τ  ⊢ e₂    , σ₁       ⇓ v₂ , σ₂ \\
  ρ′[x↦ℓ] , τ′ ⊢ e′    , σ₂[ℓ↦v₂] ⇓ v′ , σ₃}
  {ρ      , τ ⊢ e₁(e₂) , σ        ⇓ v′ , σ₃}

  \inferrule*[left=(If-T),right={\ssmall «n=0»}]{
  ρ , τ ⊢ e₁ , σ ⇓ n , σ₁ \\
  ρ , τ ⊢ e₂ , σ₁ ⇓ v , σ₂}
  {ρ , τ ⊢ ⟬if0⟭(e₁)❴e₂❵❴e₃❵ , σ ⇓ v , σ₂}

  \inferrule*[left=(If-F),right={\ssmall «n≠0»}]{
    ρ , τ ⊢ e₁ , σ  ⇓ n , σ₁ \\
    ρ , τ ⊢ e₃ , σ₁ ⇓ v , σ₂}
  {ρ , τ ⊢ ⟬if0⟭(e₁)❴e₂❵❴e₃❵ , σ ⇓ v , σ₂}

  \inferrule*[left=(Bin)]{
  ρ , τ ⊢ e₁ , σ  ⇓ v₁ , σ₁ \\
  ρ , τ ⊢ e₂ , σ₁ ⇓ v₂ , σ₂}
  {ρ , τ ⊢ b(e₁,e₂) , σ ⇓ ⟦b⟧(v₁,v₂) , σ₂}
\end{mathpar}
\caption{\lamif{} Big-step Concrete Semantics}
\label{f:lamif-concrete}
\end{figure} %}-}

\paragraph{Reachability}

The primary limitation of using big-step semantics as a starting point for
abstraction is that intermediate computations are not represented in the model.
For example, consider the program that applies the identity function to an
expression that loops, which we notate «Ω»:
\[ (λx.x)(Ω) \]
A big-step semantics can only describe results of terminating computations, and
because this program never terminates, our big-step semantics relation says
nothing about the behavior of the program. A good static analyzer will explore
the behavior of «Ω» to (possibly) discover that it loops, but more importantly,
to provide analysis results (like data-flow or side-effects) for intermediate
computation states.

The need to analyze intermediate states is the primary reason that big-step
semantics are overlooked as a starting point for abstract interpretation. To
remedy the situation, while remaining in a big-step setting, we introduce a
big-step \emph{reachability semantics}, described by the relation «ρ,τ⊢e,σ⇑ς»
shown in Figure~\ref{f:lamif-reachability}. Configurations «ς» are tuples
«⟨e,ρ,σ,τ⟩», and are reachable when evaluation passes through the configuration
at any point on its way to a final value, or during an infinite loop.

\begin{figure} %{-{
\begin{mathpar}
  \inferrule*[left=(Refl)]{ }{ρ,τ⊢e,σ⇑⟨e,ρ,σ,τ⟩}

  \inferrule*[left=(RApp1)]
   {ρ,τ⊢e₁,σ⇑ς}
   {ρ,τ⊢e₁(e₂),σ⇑ς}

   \inferrule*[left=(RApp2),right={\ssmall «⟨λx.e′,ρ′⟩=v₁»}]
   {  ρ,τ⊢e₁,σ⇓v₁,σ₁
   \\ ρ,τ⊢e₂,σ₁⇑ς}
   {ρ,τ⊢e₁(e₂),σ⇑ς}

   \inferrule*[left=(RApp3),right={\begin{minipage}{2em}\ssmall
       \begin{alignat*}{1}
       \begin{alignedat}{2} 
    ⟨λx.e′,ρ′⟩ = &\mathrel{} v₁ \\[-0.5em]
           τ′ = &\mathrel{} τ+1 \\[-0.5em]
           ℓ  = &\mathrel{} ⟨x,τ′⟩ \\
         \end{alignedat}
       \end{alignat*}
     \end{minipage}}]
  {  ρ,τ⊢e₁,σ ⇓v₁,σ₁
  \\ ρ,τ⊢e₂,σ₁⇓v₂,σ₂
  \\ ρ′[x↦ℓ],τ′⊢e′,σ₂[ℓ↦v₂]⇑ς}
  {ρ,τ⊢e₁(e₂),σ⇑ς}

  \inferrule*[left=(RIf1)]
  {ρ,τ⊢e₁,σ⇑ς}
  {ρ,τ⊢⟬if0⟭(e₁)❴e₂❵❴e₃❵,σ⇑ς}

  \inferrule*[left=(RIf-T),right={\ssmall «n=0»}]
  {  ρ,τ⊢e₁,σ⇓n,σ₁
  \\ ρ,τ⊢e₂,σ₁⇑ς}
  {ρ,τ⊢⟬if0⟭(e₁)❴e₂❵❴e₃❵,σ⇑ς}

  \inferrule*[left=(RIf-F),right={\ssmall «n≠0»}]
  {  ρ,τ⊢e₁,σ⇓n,σ₁
  \\ ρ,τ⊢e₃,σ₁⇑ς}
  {ρ,τ⊢⟬if0⟭(e₁)❴e₂❵❴e₃❵,σ⇑ς}

  \inferrule*[left=(RBin1)]
  {ρ,τ⊢e₁,σ⇑ς}
  {ρ,τ⊢b(e₁,e₂),σ⇑ς}

  \inferrule*[left=(RBin2)]
  {  ρ,τ⊢e₁,σ⇓v₁,σ₁
  \\ ρ,τ⊢e₂,σ₁⇑ς}
  {ρ,τ⊢b(e₁,e₂),σ⇑ς}

\end{mathpar}
\caption{\lamif{} Big-step Reachability Semantics}
\label{f:lamif-reachability}
\end{figure} %}-}

The complete concrete semantics of an expression («e») under environment («ρ»),
store («σ») and time («τ»), which we notate «⟦e⟧⸢bs⸣(ρ,σ,τ)», is then a pairing
of the set of resulting values and final stores («v,σ′»), and the reachable
configurations («ς»):
\begin{alignat*}{1}
                & ⟦e⟧⸢bs⸣(ρ,σ,τ) ≔ 
\\[\mathgobble] & \hspace{1em}⟨❴⟨v,σ′⟩ ∣ ρ,τ⊢e,σ⇓v,σ′❵,❴ς ∣ ρ,τ⊢e,σ⇑ς❵⟩
\end{alignat*}
We then construct a formal bridge between the complete concrete semantics
(«⟦e⟧⸢bs⸣») and a complete small step semantics, which is traditionally used as
the starting point of abstraction for program analysis:
\begin{alignat*}{1}
                & ⟦e⟧⸢ss⸣(ρ,σ,τ) ≔ 
\\[\mathgobble] & \hspace{1em}⟨❴⟨v,σ′⟩ ∣ ∀κ. ⟨e,ρ,σ,τ,κ⟩ ↝⸢*⸣ ⟨v,ρ′,σ′,τ′,κ⟩❵ 
\\[\mathgobble] & \hspace{1em},❴⟨e′,ρ′,σ′,τ′⟩ ∣ ∀κ. ⟨e,ρ,σ,τ,κ⟩ ↝⸢*⸣ ⟨e′,ρ′,σ′,τ′,κ′⧺κ⟩❵⟩
\end{alignat*}
We connect the complete big-step and small-step semantics through the following theorem:
\begin{theorem}[Complete Big-step/Small-step Equivalence]
  \[ ⟦e⟧⸢bs⸣(ρ,σ,τ) = ⟦e⟧⸢ss⸣(ρ,σ,τ) \]
\end{theorem}
The proof is by induction on the big-step derivation in the «⊆» direction, and
on the transitive small-step derivation in the «⊇» direction.

\paragraph{Collecting Semantics}

Before abstracting the semantics—in pursuit of a sound static analysis
algorithm—we pass through a big-step collecting and reachability semantics,
«ρ,τ⊢e,∿{σ}⇓∿{v},∿{σ}» and «ρ,τ⊢e,∿{σ}⇑∿{ς}», shown in
Figure~\ref{f:lamif-collecting}, where «∿{v}», «∿{σ}» and «∿{ς}» range over
collecting state spaces:
\begin{alignat*}{3}
                ∿{v} ∈ &&\mathrel{} ∿{val}      &\mathrel{} ≔ ℘(val) 
\\[\mathgobble] ∿{σ} ∈ &&\mathrel{} ∿{store}    &\mathrel{} ≔ addr ↦ ∿{val} 
\\[\mathgobble] ∿{ς} ∈ &&\mathrel{} ∿{config}   &\mathrel{} ≔ exp × env × ∿{store} × time
\end{alignat*}
The denotation for binary operators («⟦b⟧») is lifted to a collecting
denotation operator «∿{⟦b⟧}»:
\[ ∿{⟦b⟧}(∿{v}₁,∿{v}₂) ≔ ❴⟦b⟧(v₁,v₂) ∣ v₁ ∈ ∿{v}₁ ∧ v₂ ∈ ∿{v}₂❵ \]

\begin{figure*} %{-{
\begin{mathpar}
  \inferrule*[left=(OLit)]{ }{ρ,τ⊢n,∿{σ}⇓❴n❵,∿{σ}}

  \inferrule*[left=(OVar)]{ }{ρ,τ⊢x,∿{σ}⇓∿{σ}(ρ(x)),∿{σ}}

  \inferrule*[left=(OLam)]{ }{ρ,τ⊢λx.e,∿{σ}⇓❴⟨λx.e,ρ⟩❵,∿{σ}}

  \inferrule*[left=(OApp),right={\begin{minipage}{2em}\ssmall
    \begin{alignat*}{1}
    \begin{alignedat}{2} 
              ⟨λx.e′,ρ′⟩ ∈ &\mathrel{} ∿{v}₁ \\[-0.5em]
                      τ′ = &\mathrel{} τ+1 \\[-0.5em]
                      ℓ  = &\mathrel{} ⟨x,τ′⟩
      \end{alignedat}
    \end{alignat*}
  \end{minipage}}]
  {  ρ      ,τ ⊢e₁    ,∿{σ} ⇓∿{v}₁,∿{σ}₁
  \\ ρ      ,τ ⊢e₂    ,∿{σ}₁⇓∿{v}₂,∿{σ}₂
  \\ ρ′[x↦ℓ],τ′⊢e′    ,∿{σ}₂[ℓ↦∿{v}₂]⇓∿{v}′,∿{σ}₃}
  {ρ      ,τ ⊢e₁(e₂),∿{σ} ⇓∿{v}′,∿{σ}₃}

  \inferrule*[left=(OIf-T),right={\ssmall «0∈∿{v}₁»}]
  {  ρ,τ⊢e₁,∿{σ}⇓∿{v}₁,∿{σ}₁
  \\ ρ,τ⊢e₂,∿{σ}₁⇓∿{v},∿{σ}₂}
  {  ρ,τ⊢⟬if0⟭(e₁)❴e₂❵❴e₃❵,∿{σ}⇓∿{v},∿{σ}₂}

  \inferrule*[left=(OIf-F),right={\begin{minipage}{2em}\ssmall
      \begin{alignat*}{2}
        n ∈ &\mathrel{} ∿{v}₁ \\[-0.5em]
        n ≠ &\mathrel{} 0
      \end{alignat*}
    \end{minipage}}]
    {  ρ,τ⊢e₁,∿{σ}⇓∿{v}₁,∿{σ}₁
    \\ ρ,τ⊢e₃,∿{σ}₁⇓∿{v},∿{σ}₂}
    {  ρ,τ⊢⟬if0⟭(e₁)❴e₂❵❴e₃❵,∿{σ}⇓∿{v},∿{σ}₂}

    \inferrule*[left=(OBin)]
  {  ρ,τ⊢e₁,∿{σ}⇓∿{v}₁,∿{σ}₁
  \\ ρ,τ⊢e₂,∿{σ}₁⇓∿{v}₂,∿{σ}₂}
  {  ρ,τ⊢b(e₁,e₂),∿{σ}⇓∿{⟦b⟧}(∿{v}₁,∿{v}₂),∿{σ}₂}

  \inferrule*[left=(ORefl)]{ }{ρ,τ⊢e,∿{σ}⇑⟨e,ρ,∿{σ},τ⟩}

  \inferrule*[left=(ORApp1)]
   {ρ,τ⊢e₁,∿{σ}⇑∿{ς}}
   {ρ,τ⊢e₁(e₂),∿{σ}⇑∿{ς}}

   \inferrule*[left=(ORApp2),right={\ssmall «⟨λx.e′,ρ′⟩∈∿{v}₁»}]
   {  ρ,τ⊢e₁,∿{σ}⇓∿{v}₁,∿{σ}₁
   \\ ρ,τ⊢e₂,∿{σ}₁⇑∿{ς}}
   {ρ,τ⊢e₁(e₂),∿{σ}⇑∿{ς}}

   \inferrule*[left=(ORApp3),right={\begin{minipage}{2em}\ssmall
       \begin{alignat*}{1}
       \begin{alignedat}{2} 
  ⟨λx.e′,ρ′⟩ ∈ &\mathrel{} ∿{v}₁ \\[-0.5em]
           τ′ = &\mathrel{} τ+1 \\[-0.5em]
           ℓ  = &\mathrel{} ⟨x,τ′⟩ \\
         \end{alignedat}
       \end{alignat*}
     \end{minipage}}]
  {  ρ,τ⊢e₁,∿{σ} ⇓∿{v}₁,∿{σ}₁
  \\ ρ,τ⊢e₂,∿{σ}₁⇓∿{v}₂,∿{σ}₂
  \\ ρ′[x↦ℓ],τ′⊢e′,∿{σ}₂[ℓ↦∿{v}₂]⇑∿{ς}}
  {ρ,τ⊢e₁(e₂),∿{σ}⇑∿{ς}}

  \inferrule*[left=(ORIf1)]
  {ρ,τ⊢e₁,∿{σ}⇑∿{ς}}
  {ρ,τ⊢⟬if0⟭(e₁)❴e₂❵❴e₃❵,∿{σ}⇑∿{ς}}

  \inferrule*[left=(ORIf-T),right={\ssmall «0∈∿{v}₁»}]
  {  ρ,τ⊢e₁,∿{σ}⇓∿{v}₁,∿{σ}₁
  \\ ρ,τ⊢e₂,∿{σ}₁⇑∿{ς}}
  {ρ,τ⊢⟬if0⟭(e₁)❴e₂❵❴e₃❵,∿{σ}⇑∿{ς}}

  \inferrule*[left=(ORIf-F),right={\begin{minipage}{2em}\ssmall 
    \begin{alignat*}{2}
      n ∈ &\mathrel{} ∿{v}₁ \\[-0.5em]
      n ≠ &\mathrel{} 0
    \end{alignat*}
    \end{minipage}}]
    {  ρ,τ⊢e₁,∿{σ}⇓∿{v}₁,∿{σ}₁
    \\ ρ,τ⊢e₃,∿{σ}₁⇑∿{ς}}
    {ρ,τ⊢⟬if0⟭(e₁)❴e₂❵❴e₃❵,∿{σ}⇑∿{ς}}

  \inferrule*[left=(ORBin1)]
  {ρ,τ⊢e₁,∿{σ}⇑ς}
  {ρ,τ⊢b(e₁,e₂),∿{σ}⇑∿{ς}}

  \inferrule*[left=(ORBin2)]
  {  ρ,τ⊢e₁,∿{σ}⇓∿{v}₁,∿{σ}₁
  \\ ρ,τ⊢e₂,∿{σ}₁⇑∿{ς}}
  {ρ,τ⊢b(e₁,e₂),∿{σ}⇑∿{ς}}

\end{mathpar}
\caption{Big-step Collecting Reachability Semantics}
\label{f:lamif-collecting}
\end{figure*} %}-}

The big-step collecting and reachability relations are structurally similar to
the concrete semantics. The primary differences are using set containment
(«∈») in place of equality («=») when branching on application («⟨λx.e,ρ⟩» in
\textsc{(OApp)}) and conditional («n≟0» in \textsc{(OIf-T)} and
\textsc{(OIf-F)}) expressions.

The big-step collecting reachability semantics is a sound approximation of the
big-step concrete reachability semantics:
\begin{theorem}[Collecting Reachability Semantics Soundness]
  \begin{alignat*}{1}
    & ⦑If:⦒\mathrel{} ρ,τ⊢e,σ⇓v,σ′ \;∧\; η(σ) ⊑ ∿{σ} 
\\[\mathgobble] & ⦑then:⦒\mathrel{} \mathrel{∃}∿{v},∿{σ}′ \mathrel{⦑s.t.⦒} ρ,τ⊢e,∿{σ}⇓∿{v},∿{σ}′ \;∧\; v ∈ ∿{v} \;∧\; η(σ′) ⊑ ∿{σ}′ 
\\[\mathgobble] & ⦑and if:⦒\mathrel{} ρ,τ⊢e,σ⇑ς \;∧\; η(σ) ⊑ ∿{σ} 
\\[\mathgobble] & ⦑then:⦒\mathrel{} \mathrel{∃}∿{ς} \mathrel{⦑s.t.⦒} ρ,τ⊢e,∿{σ}⇑∿{ς} \;∧\; η(ς) ⊑ ∿{ς}
  \end{alignat*}
\end{theorem}
The proof is by induction on the concrete big-step derivation. The extraction
function «η» is defined separately for stores («σ») and configurations («ς»):
\begin{alignat*}{1}
   & η(σ)(ℓ) ≔ ❴σ(ℓ)❵
\\[\mathgobble] & η(⟨e,ρ,σ,τ⟩) ≔ ⟨e,ρ,η(σ),τ⟩
\end{alignat*}
and the partial ordering on collecting stores and configurations is pointwise:
\begin{alignat*}{1}
  & ∿{σ}₁ ⊑ ∿{σ}₂ \quad \mathrel{⦑\emph{iff}⦒} \quad ∀ℓ.\mathrel{} ∿{σ}₁(ℓ) ⊆ ∿{σ}₂(ℓ)
  \\[\mathgobble] & ⟨e₁,ρ₁,∿{σ}₁,τ₁⟩ ⊑ ⟨e₂,ρ₂,∿{σ}₂,τ₂⟩ \quad \mathrel{⦑\emph{iff}⦒} 
  \\[\mathgobble] & \hspace{1em} e₁ = e₂ ∧ ρ₁ = ρ₂ ∧ ∿{σ}₁ ⊑ ∿{σ}₂ ∧ τ₁ = τ₂
\end{alignat*}

\paragraph{Finite Abstraction}

The next step towards a computable static analysis is an abstract semantics
with a finite state space that approximates the big-step collecting semantics,
«♯{ρ},♯{τ}⊢e,♯{σ}⇓♯{v},♯{σ}» and «♯{ρ},♯{τ}⊢e,♯{σ}⇑♯{ς}», shown in
Figure~\ref{f:lamif-abstract}, where «♯{ρ}», «♯{τ}», «♯{v}», «♯{σ}» and «♯{ς}» are finite
abstractions of their collecting counterparts:
\begin{alignat*}{3}
  ♯{ρ} ∈ &&\mathrel{} ♯{env}    &\mathrel{} ≔ var ↦ ♯{addr}⸤⊥⸥ 
\\[\mathgobble] ♯{ℓ} ∈ &&\mathrel{} ♯{addr}   &\mathrel{} ≔ var × ♯{time} 
\\[\mathgobble] ♯{τ} ∈ &&\mathrel{} ♯{time}   &\mathrel{} ≔ … 
\\[\mathgobble] ♯{v} ∈ &&\mathrel{} ♯{val}    &\mathrel{} ≔ … 
\\[\mathgobble] ♯{σ} ∈ &&\mathrel{} ♯{store}  &\mathrel{} ≔ ♯{addr} ↦ ♯{val} 
\\[\mathgobble] ♯{ς} ∈ &&\mathrel{} ♯{config} &\mathrel{} ≔ exp × ♯{env} × ♯{store} × ♯{time}
\end{alignat*}
The primary structural difference from the collecting semantics is updating the
store with join («♯{σ}⊔[♯{ℓ}↦♯{v}]») rather than strict replacement
(«∿{σ}[ℓ↦∿{v}]»). This is to preserve soundness in the presence of address
reuse due to the finite size of the address space.

The abstract denotation («♯{⟦b⟧}») is an overapproximation of the collecting
denotation («∿{⟦b⟧}») w.r.t. a Galois connection «∿{val}⇄{α}{γ}♯{val}»:
\[ ♯{⟦b⟧}(♯{v}₁,♯{v}₂) ⊒ α(∿{⟦b⟧}(γ(♯{v}₁),γ(♯{v}₂))) \]
Concretization functions «⌊γ⌋⸤clo⸥», «⌊γ⌋⸤0⸥» and «⌊γ⌋⸤¬0⸥» are computable
finite subsets of the full concretization function «γ» s.t.:
\begin{alignat*}{1}
  & ⌊γ⌋⸤clo⸥(♯{v}) ≔ ❴⟨λx.e,♯{ρ}⟩ ∣ ⟨λx.e,♯{ρ}⟩ ∈ γ(♯{v})❵ 
\\[\mathgobble] & ⌊γ⌋⸤0⸥(♯{v}) ≔ ❴0 ∣ 0 ∈ γ(♯{v})❵ 
\\[\mathgobble] & ⌊γ⌋⸤¬0⸥(♯{v}) ≔ ❴¬0 ∣ n ∈ γ(♯{v}) ∧ n≠0❵
\end{alignat*}
Abstract sets «♯{time}» and «♯{val}» are left as parameters to the analysis
along with their operations «♯{next}», «♯{⟦b⟧}», «⌊γ⌋⸤clo⸥», «⌊γ⌋⸤0⸥»,
«⌊γ⌋⸤¬0⸥» and «⊔⸢♯{val}⸣».

The abstract semantics is a sound approximation of the collecting semantics,
which we establish through the theorem:
\begin{theorem}[Abstract Reachability Semantics Soundness]
  \begin{alignat*}{1}
    & ⦑If:⦒\mathrel{} ρ,τ⊢e,∿{σ}⇓∿{v},∿{σ}′ \;∧\; η(ρ) ⊑ ♯{ρ} \;∧\; η(τ) ⊑ ♯{τ} \;∧\; η(∿{σ}) ⊑ ♯{σ} 
\\[\mathgobble] & ⦑then:⦒\mathrel{} \mathrel{∃}♯{v},♯{σ}′ \mathrel{⦑s.t.⦒} ♯{ρ},♯{τ}⊢e,♯{σ}⇓♯{v},♯{σ}′ \;∧\; η(∿{v}) ⊑ ♯{v} \;∧\; η(∿{σ}′) ⊑ ♯{σ}′ 
\\[\mathgobble] & ⦑and if:⦒\mathrel{} ρ,τ⊢e,∿{σ}⇑∿{ς} \;∧\; η(ρ) ⊑ ♯{ρ} \;∧\; η(τ) ⊑ ♯{τ} \;∧\; η(∿{σ}) ⊑ ♯{σ} 
\\[\mathgobble] & ⦑then:⦒\mathrel{} \mathrel{∃}♯{ς} \mathrel{⦑s.t.⦒} ♯{ρ},♯{τ}⊢e,♯{σ}⇑♯{ς} \;∧\; η(∿{ς}) ⊑ ♯{ς}
  \end{alignat*}
\end{theorem}
The proof is by induction on the big-step derivation. The extraction function
«η» is defined separately for environments («ρ»), time («τ»), and collecting
stores («∿{σ}»), values («∿{v}») and configurations («∿{ς}»). «η(τ)» and
«η(∿{v})» are given with parameters «♯{time}» and «♯{val}». «η(ρ)», «η(∿{σ})»
and «η(∿{ς})» are defined pointwise:
\begin{alignat*}{1}
  & η(ρ)(x) ≔ η(ρ(x)) 
\\[\mathgobble] & η(∿{σ})(♯{ℓ}) ≔ ⨆⸤ℓ ∈ γ(♯{ℓ})⸥η(∿{σ}(ℓ)) 
\\[\mathgobble] & η(⟨e,ρ,τ,∿{σ}⟩) ≔ ⟨e,η(ρ),η(τ),η(∿{σ})⟩
\end{alignat*}

\begin{figure*} %{-{
\begin{mathpar}
  \inferrule*[left=(ALit)]{ }{♯{ρ},♯{τ}⊢n,♯{σ}⇓♯{η}(n),♯{σ}}

  \inferrule*[left=(AVar)]{ }{♯{ρ},♯{τ}⊢x,♯{σ}⇓♯{σ}(♯{ρ}(x)),♯{σ}}

  \inferrule*[left=(ALam)]{ }{♯{ρ},♯{τ}⊢λx.e,♯{σ}⇓♯{η}(⟨λx.e,♯{ρ}⟩),♯{σ}}

  \inferrule*[left=(AApp),right={\begin{minipage}{2em}\ssmall
    \begin{alignat*}{1}
    \begin{alignedat}{2} 
           ⟨λx.e′,♯{ρ}′⟩ ∈ &\mathrel{} ⌊γ⌋⸤clo⸥(♯{v}₁) \\[-0.5em]
                   ♯{ς}  = &\mathrel{} ⟨e₁(e₂),♯{ρ},♯{σ},♯{τ}⟩ \\[-0.5em]
                   ♯{τ}′ = &\mathrel{} ♯{next}(♯{τ},♯{ς}) \\[-0.5em]
                   ♯{ℓ}  = &\mathrel{} ⟨x,♯{τ}′⟩
      \end{alignedat}
    \end{alignat*}
  \end{minipage}}]
  {  ♯{ρ}      ,♯{τ} ⊢e₁    ,♯{σ} ⇓♯{v}₁,♯{σ}₁
  \\ ♯{ρ}      ,♯{τ} ⊢e₂    ,♯{σ}₁⇓♯{v}₂,♯{σ}₂
  \\ ♯{ρ}′[x↦♯{ℓ}],♯{τ}′⊢e′    ,♯{σ}₂⊔[♯{ℓ}↦♯{v}₂]⇓♯{v}′,♯{σ}₃}
  {♯{ρ}      ,♯{τ} ⊢e₁(e₂),♯{σ} ⇓♯{v}′,♯{σ}₃}

  \inferrule*[left=(AIf-T),right={\ssmall «0∈⌊γ⌋⸤0⸥(♯{v}₁)»}]
  {  ♯{ρ},♯{τ}⊢e₁,♯{σ}⇓♯{v}₁,♯{σ}₁
  \\ ♯{ρ},♯{τ}⊢e₂,♯{σ}₁⇓♯{v},♯{σ}₂}
  {  ♯{ρ},♯{τ}⊢⟬if0⟭(e₁)❴e₂❵❴e₃❵,♯{σ}⇓♯{v},♯{σ}₂}

  \inferrule*[left=(AIf-F),right={\ssmall «¬0∈⌊γ⌋⸤¬0⸥(♯{v}₁)»}]
  {  ♯{ρ},♯{τ}⊢e₁,♯{σ}⇓♯{v}₁,♯{σ}₁
  \\ ♯{ρ},♯{τ}⊢e₃,♯{σ}₁⇓♯{v},♯{σ}₂}
  {  ♯{ρ},♯{τ}⊢⟬if0⟭(e₁)❴e₂❵❴e₃❵,♯{σ}⇓♯{v},♯{σ}₂}

    \inferrule*[left=(ABin)]
    {  ♯{ρ},♯{τ}⊢e₁,♯{σ}⇓♯{v}₁,♯{σ}₁
    \\ ♯{ρ},♯{τ}⊢e₂,♯{σ}₁⇓♯{v}₂,♯{σ}₂}
    {  ♯{ρ},♯{τ}⊢b(e₁,e₂),♯{σ}⇓♯{⟦b⟧}(♯{v}₁,♯{v}₂),♯{σ}₂}

    \inferrule*[left=(ARefl)]{ }{♯{ρ},♯{τ}⊢e,♯{σ}⇑⟨e,♯{ρ},♯{σ},♯{τ}⟩}

  \inferrule*[left=(ARApp1)]
  {♯{ρ},♯{τ}⊢e₁,♯{σ}⇑♯{ς}}
  {♯{ρ},♯{τ}⊢e₁(e₂),♯{σ}⇑♯{ς}}

  \inferrule*[left=(ARApp2),right={\ssmall «⟨λx.e′,♯{ρ}′⟩∈⌊γ⌋⸤clo⸥(♯{v}₁)»}]
  {  ♯{ρ},♯{τ}⊢e₁,♯{σ}⇓♯{v}₁,♯{σ}₁
  \\ ♯{ρ},♯{τ}⊢e₂,♯{σ}₁⇑♯{ς}}
  {♯{ρ},♯{τ}⊢e₁(e₂),♯{σ}⇑♯{ς}}

   \inferrule*[left=(ARApp3),right={\begin{minipage}{2em}\ssmall
       \begin{alignat*}{1}
       \begin{alignedat}{2} 
         ⟨λx.e′,♯{ρ}′⟩ ∈ &\mathrel{} ⌊γ⌋⸤clo⸥(♯{v}₁) \\[-0.5em]
            ♯{ς} = &\mathrel{} ⟨e₁(e₂),♯{ρ},♯{σ},♯{τ}⟩ \\[-0.5em]
           ♯{τ}′ = &\mathrel{} ♯{next}(♯{τ},♯{ς}) \\[-0.5em]
           ♯{ℓ}  = &\mathrel{} ⟨x,♯{τ}′⟩ \\
         \end{alignedat}
       \end{alignat*}
     \end{minipage}}]
     {  ♯{ρ},♯{τ}⊢e₁,♯{σ} ⇓♯{v}₁,♯{σ}₁
       \\ ♯{ρ},♯{τ}⊢e₂,♯{σ}₁⇓♯{v}₂,♯{σ}₂
     \\ ♯{ρ}′[x↦♯{ℓ}],♯{τ}′⊢e′,♯{σ}₂⊔[♯{ℓ}↦♯{v}₂]⇑♯{ς}}
     {♯{ρ},♯{τ}⊢e₁(e₂),♯{σ}⇑♯{ς}}

  \inferrule*[left=(ARIf1)]
  {♯{ρ},♯{τ}⊢e₁,♯{σ}⇑♯{ς}}
  {♯{ρ},♯{τ}⊢⟬if0⟭(e₁)❴e₂❵❴e₃❵,♯{σ}⇑♯{ς}}

  \inferrule*[left=(ARIf-T),right={\ssmall «0∈⌊γ⌋⸤0⸥(♯{v}₁)»}]
  {  ♯{ρ},♯{τ}⊢e₁,♯{σ}⇓♯{v}₁,♯{σ}₁
  \\ ♯{ρ},♯{τ}⊢e₂,♯{σ}₁⇑♯{ς}}
  {♯{ρ},♯{τ}⊢⟬if0⟭(e₁)❴e₂❵❴e₃❵,♯{σ}⇑♯{ς}}

  \inferrule*[left=(ARIf-F),right={\ssmall «¬0∈⌊γ⌋⸤¬0⸥(♯{v}₁)»}]
    {  ♯{ρ},♯{τ}⊢e₁,♯{σ}⇓♯{v}₁,♯{σ}₁
    \\ ♯{ρ},♯{τ}⊢e₃,♯{σ}₁⇑♯{ς}}
    {♯{ρ},♯{τ}⊢⟬if0⟭(e₁)❴e₂❵❴e₃❵,♯{σ}⇑♯{ς}}

  \inferrule*[left=(ARBin1)]
  {♯{ρ},♯{τ}⊢e₁,♯{σ}⇑ς}
  {♯{ρ},♯{τ}⊢b(e₁,e₂),♯{σ}⇑♯{ς}}

  \inferrule*[left=(ARBin2)]
  {  ♯{ρ},♯{τ}⊢e₁,♯{σ}⇓♯{v}₁,♯{σ}₁
  \\ ♯{ρ},♯{τ}⊢e₂,♯{σ}₁⇑♯{ς}}
  {♯{ρ},♯{τ}⊢b(e₁,e₂),♯{σ}⇑♯{ς}}

\end{mathpar}
\caption{Big-step Abstract Reachability Semantics}
\label{f:lamif-abstract}
\end{figure*} %}-}

\paragraph{Computing the Analysis}

An analysis for the program «e₀» w.r.t. the abstract semantics is some cache
«\$ ∈ ♯{config} ↦ ℘(♯{val} × ♯{store})» that maps all configurations reachable
from the initial configuration «⟨e₀,♯{ρ}₀,♯{σ}₀,♯{τ}₀⟩» to their final values
and stores «♯{v},♯{σ}», which we notate «\$ ⊨ e₀»:
\begin{alignat*}{2}
  \$ ⊨ e₀ \quad\quad \mathrel{⦑\emph{iff}⦒} \quad\quad & 
    \begin{alignedat}{2}
    ⦑\emph{If:}⦒     & \mathrel{} ♯{ρ}₀,♯{τ}₀⊢e₀,♯{σ₀}⇑⟨e,♯{ρ},♯{σ},♯{τ}⟩ 
\\[\mathgobble] ⦑\emph{and:}⦒    & \mathrel{} ♯{ρ},♯{τ}⊢e,♯{σ}⇓♯{v},♯{σ}′  
\\[\mathgobble] ⦑\emph{then:}⦒   & \mathrel{} ⟨♯{v},♯{σ}′⟩ ∈ \$(⟨e,♯{ρ},♯{σ},♯{τ}⟩)
      \end{alignedat}
\end{alignat*}
The best cache «\$⁺» is then computed as the least fixed point of the
functional «ℱ»:
\begin{alignat*}{1}
  & ℱ ∈ (♯{config} ↦ ℘(♯{val}×♯{store})) → (♯{config} ↦ ℘(♯{val}×♯{store})) 
\\[\mathgobble] & ℱ ≔ λ\$.  
\\[\mathgobble] &  \hspace{1em} ⨆⸤⟨e,♯{ρ},♯{σ},♯{τ}⟩∈\$⸥ \begin{cases}
         ❴ ⟨e,♯{ρ},♯{σ},♯{τ}⟩ ↦ ❴⟨♯{v},♯{σ}′⟩❵ ∣ ♯{ρ},♯{τ}⊢e,♯{σ}⇓⸢\$⸣♯{v},♯{σ}′ ❵ 
      \\[\mathgobble] ❴ ♯{ς} ↦ ❴❵ ∣ ♯{ρ},♯{τ}⊢e,♯{σ}⇑⸢\$⸣♯{ς}❵
   \end{cases}
\end{alignat*}
and defined:
\[ \$⁺ ≔ ⦑\emph{lfp}⦒ (λ\$.\mathrel{} ℱ(\$)  ⊔ ❴⟨e₀,♯{ρ}₀,♯{σ}₀,♯{τ}₀⟩ ↦ ❴❵❵) \]
The relations «♯{ρ},♯{τ}⊢e,♯{σ}⇓⸢\$⸣♯{v},♯{σ}′» and «♯{ρ},♯{τ}⊢e,♯{σ}⇑⸢\$⸣♯{ς}»
are modified versions of the original abstract semantics, but with recursive
judgements replaced by «⟨♯{v},♯{σ}′⟩ ∈ \$(e,♯{ρ},♯{σ},♯{τ})» and «♯{ς} ∈
\$(e,♯{ρ},♯{σ},♯{τ})» respectively. Therefore «ℱ» is not recursive; the
recursion in the relations is lifted to the outer fixpoint of the analysis.
Because the state space «♯{config} ↦ ℘(♯{val}×♯{store})» is finite and «ℱ» is
monotonic, «\$⁺» can be computed algorithmically in finite time by a simple
Kleene fixed-point iteration. See Nielson et al~\cite{dvanhorn:Neilson:1999}
for more background and examples of static analyzers computed in this style,
and from which the current development was largely inspired.
\begin{theorem}[Algorithm Correctness]
  «\$⁺» is a valid analysis for «e₀», that is:
  \[ \$⁺ ⊨ e₀ \]
\end{theorem}
The proof is by induction on the assumed derivations
«♯{ρ}₀,♯{τ}₀⊢e₀,♯{σ}₀⇑⟨♯{e},♯{ρ},♯{σ},♯{τ}⟩» and «♯{ρ},♯{τ}⊢e,♯{σ}⇓♯{v},♯{σ}′»,
and utilizes the fact that «\$⁺» is a fixed point, that is:
\[ ℱ(\$⁺) = \$⁺ \]
Our final theorem relates the analysis cache «\$⁺» back to the concrete
semantics of the initial program as a sound approximation:
\begin{theorem}[Algorithm Soundness]
  \begin{alignat*}{1}
    & ⦑If:⦒ \mathrel{} ρ₀,τ₀⊢e₀,σ₀⇑⟨e,ρ,σ,τ⟩ \;∧\; ρ,τ⊢e,σ⇓v,σ′  
\\[\mathgobble] & ⦑then:⦒ \mathrel{} ∃♯{ρ},♯{τ},♯{σ},♯{v},♯{σ}′ \mathrel{⦑s.t.⦒}  ⟨♯{v},♯{σ}′⟩ ∈ \$⁺(⟨e,♯{ρ},♯{σ},♯{τ}⟩)  
\\[\mathgobble] & \hspace{1em} ∧\; η(ρ) ⊑ ♯{ρ} \;∧\; η(τ) ⊑ ♯{τ} \;∧\; η(σ) ⊑ ♯{σ} 
\\[\mathgobble] & \hspace{1em} ∧\; η(v) ⊑ ♯{v} \;∧\; η(σ′) ⊑ ♯{σ}′
  \end{alignat*}
\end{theorem}
The proof follows by composing Theorems~1-4.

\paragraph{Computing with Definitional Interpreters}

The algorithm described in Section~\ref{s:cache} is a more efficient strategy
for computing «\$⁺» using an extensible open-recursive definitional
interpreter. This technique is general, and bridges the gap between the
big-step abstract semantics formalized in this section and the definitional
interpreters we wish to execute to obtain analyses.

An extensible open-recursive definitional interpreter for \lamif (the small
language formalized in this section) has domain:
\begin{alignat*}{1}
  & ℰ ∈ Σ → Σ \quad ⦑\emph{where}⦒ \quad Σ ≔ ♯{config} → ℘(♯{val}×♯{store})
\end{alignat*}
and is defined such that its denotational-fixpoint («Y(ℰ)») recovers concrete
interpretation when instantiated with the concrete state-space. For example,
the recursive case for binary operator expressions is defined:
\begin{alignat*}{1}
  & ℰ(ℰ′)(⟨b(e₁,e₂),♯{ρ},♯{σ},♯{τ}) ≔  
\\[\mathgobble] & \hspace{1em} ❴\mathrel{} ♯{⟦b⟧}(♯{v}₁,♯{v₂}) 
\\[\mathgobble] & \hspace{1em} ∣\mathrel{} ⟨♯{v}₁,♯{σ}₁⟩ ∈ ℰ′(⟨e₁,♯{ρ},♯{σ},♯{τ}⟩) ∧ ⟨♯{v}₂,♯{σ}₂⟩ ∈ ℰ′(⟨e₂,♯{ρ},♯{σ}₁,♯{τ}⟩) ❵
\end{alignat*}
The iteration strategy to analyze the program «e₀» is then to run «e₀» using
«ℰ», but intercepting recursive calls to:
\begin{enumerate}
  \item cache results for all intermediate configurations «♯{ς}», and
  \item cache seen states to prevent infinite loops.
\end{enumerate}
(1) is required to fulfill the specification that «\$⁺» include results for all
reachable configurations from «e₀», and (2) is required to reach a fixed point
of the analysis. To track this extra information we add functional state to the
interpreter (which was done through a monad transformer in the tutorial
development) of type:
\[ ♯{cache} ≔ ♯{config} ↦ ℘(♯{val}×♯{store}) \]
such that the open-recursive evaluator has type:
\begin{alignat*}{1}
  & ℰ ∈ Σ → Σ \quad ⦑\emph{where}⦒ 
\\[\mathgobble] & \hspace{1em} Σ ≔ ♯{config}×♯{cache} → ℘(♯{val}×♯{store})×♯{cache}
\end{alignat*}
The iteration to compute «\$⁺» given «ℰ» is then defined:
\begin{alignat*}{1}
  & \hspace{0em} \$⁺ ≔ ⦑\emph{lfp}⦒(λ\$ᵒ. 
\\[\mathgobble] & \hspace{1em} \mathrel{⟬let⟭} ℰ⋆ ≔ Y(λℰ′. 
\\[\mathgobble] & \hspace{2em}    ℰ(λ⟨♯{ς},\$ⁱ⟩. 
\\[\mathgobble] & \hspace{3em}      \mathrel{⟬if⟭} ♯{ς} ∈ \$ⁱ \mathrel{⟬then⟭} ⟨\$ⁱ(♯{ς}),\$ⁱ⟩ \mathrel{⟬else⟭} 
\\[\mathgobble] & \hspace{3em}      \mathrel{⟬let⟭} ⟨♯{VS},\$⸢i\prime⸣⟩ ≔ ℰ′(♯{ς},\$ⁱ[♯{ς}↦\$ᵒ(♯{ς})]) 
\\[\mathgobble] & \hspace{3em}      \mathrel{⟬in⟭} ⟨♯{VS},\$⸢i\prime⸣[♯{ς}↦♯{VS}]⟩)) 
\\[\mathgobble] & \hspace{1em} \mathrel{⟬in⟭} π₂(ℰ⋆(⟨e₀,♯{ρ}₀,♯{σ}₀,♯{τ}₀⟩,❴❵)))
\end{alignat*}
The fixed interpreter «ℰ⋆» calls the unfixed interpreter «ℰ», but intercepts
recursive calls to perform (1) and (2) described above. When loops are
detected, the results from the previous complete result «\$ᵒ» is used, and the
outer fixpoint of the algorithm computes the least fixed point of this «\$ᵒ».

The end result is that, rather than compute analysis results and reachable
states naively with Kleene fixpoint iteration, we are able to reuse the
standard definitional interpreter—written in open-recursive form—to
simultaneously explore reachable states, cache intermediate configurations, and
iterate towards a least fixpoint solution for the analysis. This method is more
efficient, and reuses an extensible definitional interpreter which can recover
a wide range of analyses, including concrete interpretation.

\paragraph{Widening}

Two forms of widening can be employed to the semantics and iteration algorithm
to achieve acceptable performance for the abstract interpreter.

The first form of widening is to widen the store in the result set
«℘(♯{val}×♯{store})» to «℘(♯{val})×♯{store}» in the evaluator «ℰ»:
\begin{alignat*}{1}
  & ℰ ∈ Σ → Σ \quad ⦑\emph{where}⦒ 
\\[\mathgobble] & \hspace{1em} Σ ≔ ♯{config} × ♯{cache} → ℘(♯{val})×♯{store}×♯{cache}
\end{alignat*}
We perform this widening systematically and with no added effort through the
use of Galois Transformers in Section~\ref{s:widening}. The iteration strategy for
this widened state space is the same as before, which computes a fixed point of
the outer cache «\$ᵒ».

The next form of widening is to pull the store out of the configuration space
\emph{entirely}, that is:
\begin{alignat*}{2}
  ♯{ς} ∈ &\mathrel{} ♯{config} ≔ exp × ♯{env} × ♯{time} 
\\[\mathgobble] \$ ∈ &\mathrel{} ♯{cache} ≔ ♯{config} ↦ ℘(♯{val})
\end{alignat*}
and:
\begin{alignat*}{1}
  & ℰ ∈ Σ → Σ \quad ⦑\emph{where}⦒ 
\\[\mathgobble] & \hspace{1em} Σ ≔ ♯{config}×♯{store}×♯{cache} → ℘(♯{val})×♯{store}×♯{cache}
\end{alignat*}
The fixed point iteration then finds a mutual least fixed-point of both the
outer cache «\$ᵒ» \emph{and} the store «♯{σ}»:
\begin{alignat*}{1}
  & \hspace{0em} ⟨\$⁺,♯{σ}⁺⟩ ≔ ⦑\emph{lfp}⦒(λ⟨\$ᵒ,♯{σ}⟩. 
\\[\mathgobble] & \hspace{1em} \mathrel{⟬let⟭} ℰ⋆ ≔ Y(λℰ′. 
\\[\mathgobble] & \hspace{2em}    ℰ(λ⟨♯{ς},♯{σ}ⁱ,\$ⁱ⟩. 
\\[\mathgobble] & \hspace{3em}      \mathrel{⟬if⟭} ♯{ς} ∈ \$ⁱ \mathrel{⟬then⟭} ⟨\$ⁱ(♯{ς}),σⁱ,\$ⁱ⟩ \mathrel{⟬else⟭} 
\\[\mathgobble] & \hspace{3em}      \mathrel{⟬let⟭} ⟨♯{V},♯{σ}⸢i\prime⸣,\$⸢i\prime⸣⟩ ≔ ℰ′(♯{ς},♯{σ}ⁱ,\$ⁱ[♯{ς}↦\$ᵒ(♯{ς})]) 
\\[\mathgobble] & \hspace{3em}      \mathrel{⟬in⟭} ⟨♯{V},♯{σ}⸢i\prime⸣,\$⸢i\prime⸣[♯{ς}↦♯{V}]⟩)) 
\\[\mathgobble] & \hspace{1em} \mathrel{⟬in⟭} π⸤2×3⸥(ℰ⋆(⟨e₀,♯{ρ}₀,♯{τ}₀⟩,♯{σ},❴❵)))
\end{alignat*}
This second version of widening, which computes a fixpoint also over the store,
recovers a so-called \emph{flow-insensitive} analysis. In this model, all
program states are re-analyzed in the store resulting from execution. Also, the
cache («\$») does not index over store states «♯{σ}» in its domain, greatly
reducing its size, and leading to a much more efficient (although less precise)
static analyzer.

\paragraph{Recovering Classical 0CFA}

From the fully widened static analyzer, which computes a mutual fixpoint
between a cache and store, we can easily recover a classical 0CFA analysis. We
do this by instantiating «♯{time}» to the singleton abstraction «❴•❵», as was
shown in the original work on AAM~\cite{dvanhorn:VanHorn2010Abstracting}. In
this setting, the lexical environment «ρ» is uniquely determined by the program
expression «e», and can therefore be eliminated, resulting in the analysis
state space:
\begin{alignat*}{2}
  ♯{ς} ∈ &\mathrel{} ♯{config} ≔ exp 
\\[\mathgobble] \$ ∈ &\mathrel{} ♯{cache} ≔ exp ↦ ℘(♯{val}) 
\\[\mathgobble] ♯{σ} ∈ &\mathrel{} ♯{store} ≔ var ↦ ℘(♯{val})
\end{alignat*}
The specification for the analysis and the fully store-widened least
fixed-point iteration for computing it recovers the constraint-based
description of 0CFA given by Nielson \emph{et al}
in~\cite{dvanhorn:Neilson:1999}, where 0CFA is defined as the smallest cache
(«\$») and store («σ») which satisfy a co-inductively defined judgment:
\[ \$,σ ⊨ e \]

\paragraph{Recovering Pushdown Analysis}

We borrow from the recent result in pushdown analysis~\cite{local:p4f} which shows
that full pushdown precision can be achieved in a small-step store-widened
abstract semantics by allocating continuations using the address space of
program expressions paired with abstract environments («⟨e,♯{ρ}⟩»). In other
words, «⟨e,♯{ρ}⟩» is sufficient to achieve full pushdown precision because
the tuple uniquely identifies the evaluation context up to the final result of
evaluation.

Our fully widened semantics recovers this pushdown setting because the
cache maps tuples «⟨e,♯{ρ},♯{τ}⟩», which trivially contains «⟨e,♯{ρ}⟩».
Furthermore, we can immediately see that the abstract time component «♯{τ}» is
redundant, and can eliminate it from the cache, resulting in analysis state
space:
\begin{alignat*}{2}
  ♯{ς} ∈ &\mathrel{} ♯{config} ≔ exp × ♯{env} × ♯{time} 
\\[\mathgobble] \$ ∈ &\mathrel{} ♯{cache} ≔ exp × ♯{env} ↦ ℘(♯{val}) 
\\[\mathgobble] ♯{σ} ∈ &\mathrel{} ♯{store} ≔ var × ♯{addr} ↦ ℘(♯{val})
\end{alignat*}

An advantage of our setting is that we recover pushdown analysis also for
varying degrees of store-widening, which is not the case in Gilray \emph{et
al}~\cite{local:p4f}, although pushdown precision for non-widened semantics has been
shown before in Johnson and Van Horn~\cite{dvanhorn:Johnson2014Abstracting}.
Furthermore, the implementation of our analyzer achieves this precision by
precise call-return matching in the defining metalanguage of a definitional
interpreter, requiring no added instrumentation to the state-space of the
analyzer (although in the case of Gilray \emph{et al} the instrumentation is
minor).

Going back to Nielson et al~\cite{dvanhorn:Neilson:1999}, it would be interesting to redevelop
their constraint-based analysis descriptions of kCFA in a form that
recovers pushdown precision. Such an exercise would amount to translating our
big-step abstract semantics instantiated to kCFA to a constraint system. The
resulting system would differ from classical kCFA by the addition of
environments «♯{ρ}» (which Nielson et al call context environments) to the
domain of the cache. In this way our formal framework is able to bridge the gap
between results in pushdown analysis described via small-step machines \emph{a
la} Van Horn and Might~\cite{dvanhorn:VanHorn2010Abstracting}, and
constraint-based systems \emph{a la} Nielson et al for which pushdown analysis
has yet to be described effectively.
