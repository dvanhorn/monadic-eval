\section{Symbolic Execution and Path-sensitive Verification}\label{s:symbolic}

As a final exercise in abstract interpretation component engineering, we
develop a monadic view of symbolic execution.

We present an extension to the monad stack and metafunctions that gives rise to
a symbolic executor~\cite{dvanhorn:King1976Symbolic}, then show how
abstractions discussed in previous sections can be applied to enforce
termination, turning a traditional symbolic execution into a path-sensitive
verification engine.

\subsection{Symbolic Execution}
Figure~\ref{s:symbolic} shows the units needed to turn the existing interpreter
into a symbolic executor, in addition to adding symbolic numbers ⸨(sym X)⸩
to the language syntax.
Primitives such as ⸨'quotient⸩ now may also take as input
and return symbolic values.
As standard, symbolic execution employs a path-condition
accumulating assumptions made at each branch,
allowing the elimination of infeasible paths and construction of test cases.
We represent the path-condition ⸨φ⸩ as a set of symbolic values
known to have evaluated to ⸨0⸩.
This set is another state component provided by ⸨StateT⸩.
Monadic operations ⸨get-path-cond⸩
and ⸨refine⸩ reference and update the path-condition.
Metafunction ⸨zero?⸩ works similarly to the concrete counterpart,
but also uses the path-condition to prove that some symbolic numbers
are definitely ⸨0⸩ or non-⸨0⸩.
In case of uncertainty, ⸨zero?⸩ returns both answers
besides refining the path-condition with the assumption made.
Operator ⸨'¬⸩ represents negation in our language.

In the following example, the symbolic executor recognizes that
result ⸨3⸩ and division-by-0 error are not feasible:
ℑ⁅
¦ > (if0 'x (if0 'x 2 3) (quotient 5 'x))
ℑ,
¦ (set
¦    (cons '(quotient 5 x) (set '(¬ x)))
¦    (cons 2 (set 'x)))
ℑ⁆

A scaled up symbolic executor can have ⸨zero?⸩ calling out
to an SMT solver for interesting arithmetics,
and extend the language with symbolic functions
and blame semantics for sound higher-order symbolic
execution~\cite{dvanhorn:TobinHochstadt2012Higherorder,dvanhorn:Nguyen2015Relatively}.

\begin{figure}
\begin{alignat*}{4}
   e ∈ &&\mathrel{}   exp ⩴ … ∣ &\mathrel{} 𝔥⸨(sym⸩\ x𝔥⸨)⸩             &\hspace{3em} [⦑\emph{symbolic number}⦒]
\end{alignat*}
\rfloat{⸨symbolic-monad@⸩}
\begin{lstlisting}
¦ (define-monad
¦   (ReaderT (FailT (StateT (StateT (NondetT ID))))))
\end{lstlisting}
\figskip\rfloat{⸨ev-symbolic@⸩}
\begin{lstlisting}
¦ (define (((ev-symbolic ev₀) ev) e)
¦   (match e
¦     [(sym x) (return x)]
¦     [e       ((ev₀ ev) e)]))
\end{lstlisting}
\figskip\rfloat{⸨δ-symbolic@⸩}
\begin{lstlisting}
¦ (define (δ . ovs)
¦   (match ovs
¦     ... ; TODO can't put comment in here...
¦     [(list 'quotient v₀ v₁)
¦      (do z? ← (zero? v₁)
¦          (cond
¦           [z? fail]
¦           [(and (number? v₀) (number? v₁))
¦            (return (quotient v₀ v₁))]
¦           [else
¦            (return `(quotient ,v₀ ,v₁))]))]
¦     [(list '¬ 0) 1]
¦     ... ; TODO can't put comment in here...
¦     ))
¦ (define (zero? v)
¦   (do φ ← get-path-cond
¦       (match v
¦         [(? number? n) (return (= 0 n))]
¦         [v #:when (∈ v φ) (return #t)]
¦         [v #:when (∈ `(¬ ,v) φ) (return #f)]
¦         [`(¬ ,v′) (do a ← (zero? v′)
¦                       (return (not a)))]
¦         [v (mplus (do (refine v)
¦                       (return #t))
¦                   (do (refine `(¬ ,v))
¦                       (return #f)))])))
\end{lstlisting}
\caption{Symbolic Execution Variant}
\label{f:symbolic}
\end{figure}

\subsection{From Symbolic Execution to Verification}

Traditional symbolic executors mainly aim to find bugs
and provide no termination guarantee.
We can apply abstracting units presented in previous sections,
namely base value widening (Section~\ref{s:base}), finite allocation
(Section~\ref{s:closures}), caching and fixing (Sections~\ref{s:cache}
and~\ref{s:fixing-cache}) to turn a symbolic execution into a sound,
path-sensitive program verification.

Operations on symbolic values introduce a new source of infinite configurations
by building up new symbolic values.
We therefore straightforwardly widen a symbolic value to the abstract
number ⸨'N⸩ when it shares an address with a different number.
Figure~\ref{f:symbolic-widen} shows extension to ⸨δ⸩ and ⸨zero?⸩
in the presence of ⸨'N⸩.
The different treatments of ⸨'N⸩ and symbolic values
clarifies that abstract values are not symbolic values:
the former stands for a set of multiple values,
whereas the latter stands for an single unknown value.
Tests on abstract number ⸨'N⸩ do not strengthen the path-condition.
It is unsound to accumulate any assumption about ⸨'N⸩.

\begin{figure}
\rfloat{⸨δ-symbolic@⸩}
\begin{lstlisting}
¦ (define (δ . ovs)
¦   (match ovs
¦     ... ; TODO can't put comment in here...
¦     [(list 'quotient v₀ v₁)
¦      (do z? ← (zero? v₁)
¦          (cond
¦           [z? _fail]
¦           [else
¦            (match (list v₀ v₁)
¦             [(list (? number? n₀) (? number? n₁))
¦              (return (quotient n₀ n₁))]
¦             [(list _ ... 'N _ ...)
¦              (return 'N)]
¦             [(list v₀ v₁)
¦              (return `(quotient ,v₀ ,v₁))])]))]
¦     ... ; TODO can't put comment in here...
¦     ))
¦ (define (zero? v)
¦   (do φ ← get-path-cond
¦       (match v
¦         ['N (mplus (return #t) (return #f))]
¦         ...)))
\end{lstlisting}
\caption{Symbolic Execution with Abstract Numbers}
\label{f:symbolic-widen}
\end{figure}
