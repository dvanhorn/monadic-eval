\section{From Machines to Compositional Evaluators}

In recent years, there has been considerable effort in the systematic
construction of abstract interpreters for higher-order languages using abstract
machines---first-order transition systems---as a semantic basis.  The so-called
\emph{abstracting abstract machines} (AAM) approach to abstract
interpretation~\cite{dvanhorn:VanHorn2010Abstracting} is a recipe for
transforming a machine semantics into an easily abstractable form. There are a
few essential elements to the transformation:

\begin{itemize}
\item allocating continuations in the store
\item allocated variable bindings in the store
\item using a store that maps addresses to \emph{sets} of values
\item interpreting store updates as a join
\item interpreting store dereference as a non-deterministic choice
\end{itemize}

These transformations are semantics-preserving as the original and derived
machines operate in a lock-step correspondence.  But the real value of the
derived semantics stems from the fact that it's possible to turn the derived
machine into an abstract interpreter with two simple steps:

\begin{itemize}
\item bounding store allocation to a finite set of addresses
\item widening base values to some abstract domain
\end{itemize}

Moreover, the soundness of the resulting abstraction is self-evident and easily
proved.

The AAM approach has been applied to a wide variety of languages and
applications, and given the success of the AAM approach, it's natural to wonder
what is essential about the low-level machine basis of the semantics and
whether a similar approach is possible with a higher-level formulation of the
semantics such as a compositional evaluation function.

This paper shows that the essence of the AAM approach can be put on a
high-level semantic basis.  We show that compositional evaluators, written in
monadic style can express similar abstractions to that of AAM.  Moreover, we
show that the high-level semantics offers a number of benefits not available to
the machine model.  

First, as we will see, the definitional interpreter approach is not formulated
as a transformation on the semantics itself, but rather uses alternative
notions of a monad to express the ``abstracting'' transformations.  This means
the concrete and abstract interpreters for a language can share large parts of
their implementation; there is just one interpreter with a multiplicity of
interpretations.

Second, there is a rich body of work and many tools and techniques for
constructing \emph{extensible} interpreters, all of which applies to high-level
semantics, not machines.  By putting abstract interpretation for higher-order
languages on a high-level semantic basis, we can bring these results to bear on
the construction of extensible abstract interpreters.  In particular, we use
\emph{monad transformers} to build re-usable components for mixing and matching
the constiuent parts of an abstract interpreter.

Finally, using definitional interpreters for abstract interpretation satisfies
an intellectual itch that asks whether it can be done at all.  In solving this
technical challenge, we discover a pleasant surprise about the definitional
interpreter approach: it is inherently ``pushdown.''  Under the interpreter
approach, the property follows for free as a gift from the metalanguage.
