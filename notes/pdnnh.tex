\documentclass[12pt]{article}
\usepackage{fullpage}
\usepackage{mathpartir}
\usepackage{times}
\usepackage{mathpartir}
\usepackage{amsmath}
\usepackage{amsthm}
\usepackage{color}
\usepackage{calc}
\usepackage{xspace}
\usepackage{pifont}
\usepackage{upgreek}
\usepackage{graphicx}
\usepackage{hyperref}

\newcommand\alloc{\widehat{\mathit{alloc}}}
\newcommand\cache{\widehat{C}}
\newcommand\mvar{x}
\newcommand\mlab{\ell}
\newcommand\mexp{e}
\newcommand\mval{v}
\newcommand\menv{\rho}
\newcommand\maddr{a}
\newcommand\msto{\sigma}
\newcommand\mlvar{\mvar^\mlab}
\newcommand\mce{{\mathit{ce}}}
\newcommand\mcntr{\delta}
\newcommand\slam[2]{(\lambda{#1}.{#2})}
\newcommand\sllam[2]{\slam{#1}{#2}^{\mlab}}
\newcommand\sapp[2]{({#1}\;{#2})}
\newcommand\slapp[2]{({#1}\;{#2})^\mlab}
\newcommand\sclos[2]{\langle{#1},{#2}\rangle}
\newcommand\NNH[1]{\NNHalt\mce\mcntr{#1}}
\newcommand\NNHalt[3]{\cache\models^{#1}_{#2} {#3}}

\newcommand\CES[1]{\NNHalt\menv\msto{#1}}
\newcommand\CESalt[3]{\NNHalt{#1}{#2}{#3}}
\newcommand\PD[3]{\CES{#1 \Downarrow #2,#3}}
\newcommand\PDalt[5]{\NNHalt{#1}{#2}{#3 \Downarrow #4,#5}}


\begin{document}
\begin{abstract}
This note explains how to present pushdown flow analysis in the style
of Nielsen, Nielsen and Hankin.
\end{abstract}
\section{Classic NNH}

Here is the NNH-style presentation of $k$-CFA:
\[
\begin{array}{lcl}
\NNH\mlvar
&\iff&
\cache(\mvar,\mce(\mvar)) \subseteq \cache(\mlab,\mcntr)
\\[2mm]
\NNH{\sllam\mvar\mexp}
&\iff&
\{\sclos{\slam\mvar\mexp}\mce\} \subseteq \cache(\mlab,\mcntr)
\\[2mm]
\NNH{\slapp{\mexp^{\mlab_0}}{\mexp^{\mlab_1}}}
&\iff&
\NNH{\mexp^{\mlab_0}}\\
&\wedge&
\NNH{\mexp^{\mlab_1}}\\
&\wedge&
\forall \sclos{\slam\mvar{\mexp^{\mlab'}}}{\mce_0} \in \cache(\mlab_0,\mcntr).\\
&&
\qquad \cache(\mlab_1,\mcntr) \subseteq \cache(\mvar,\mcntr)\  \wedge\\
&&
\qquad \NNHalt{\mce[\mvar\mapsto\mcntr']}{\mcntr'}{\mexp^{\mlab'}}\\
&&
\qquad \cache(\mlab',\mcntr') \subseteq \cache(\mlab,\mcntr)
\\
&&
\qquad \mbox{where }\mcntr' = \lfloor \mcntr\cdot\mlab \rfloor_k
\end{array}
\]


As I noted in my diss., if you squint, you'll see a ``big-step''
evaluator lurking within the formalism.  However it is a weird evaluator:
\begin{itemize}
\item it is stated as validity relation of a ``cache'' w.r.t.~a program,
\item it does not ``return'' values, instead it uses the cache to
  communicate returns, i.e. the abstract evaluation of
  $\NNH{\mexp^\mlab}$ is a value $\mval \in \cache(\mlab,\mcntr)$,
\item it does not ``bind'' variables, instead it uses the cache to
communicate bindings,
\end{itemize}

The validity relation is coinductive, implying that to check the
validity of a cache, you must coinductively memoize validation to
avoid looping.

To \emph{construct} the minimal cache that satisfies the validity
relation, an easy algorithm is as follows:

\begin{enumerate}
\item Start with the empty cache.
\item Try to validate the cache,
\begin{enumerate}
\item if it fails, add the minimal information needed to validate the rule that failed, try again.
\item otherwise return the cache.
\end{enumerate}
\end{enumerate}

Since the call and return mechanism is intertwined with a finite
cache, it does not properly model function calls.  This leads to a
imprecision that is fundamental to the model (i.e. there are sound
analyses (caches), which are not validated by this judgment).

\section{Pushdown NNH}

First of all, let's just assume syntax is identified up to source
location in a program (i.e. we implicitly assume a unique labelling
and ommit labels in the model).

Second, we'll use a bonafide heap and environment, but make the usual
restriction that the heap is finite and maps to a set of values.

The cache will now map ``configurations'', i.e. $\mathit{Exp,Env,Sto}$
triples to sets of values.

\emph{We will avoid the problem of interleaving abstract evaluation
  with the table by extending the relation to talk about the result of
  evaluation.  The table simply goes along for the ride in
  verification, it never drives the control of verification as it does
  in NNH}.

The new relation takes the form $\PD\mexp\mval{\msto'}$, which can be
read as ``the cache validates $\mexp$ evaluating
under $\menv$ and $\msto$ to $\mval,\msto'$.''

\begin{mathpar}
\inferrule
    {\mval \in \cache(\mvar,\menv,\msto)\\
      \mval \in \msto(\menv(\mvar))}
    {\PD\mvar\mval\msto}

\inferrule
    {\sclos{\slam\mvar\mexp}\menv \in \cache(\slam\mvar\mexp,\menv,\msto)}
    {\PD{\slam\mvar\mexp}{\sclos{\slam\mvar\mexp}\menv}{\msto}}

\inferrule
    {\PD{\mexp_0}{\sclos{\slam\mvar\mexp}{\menv'}}{\msto_1}\\
      \PDalt{\menv}{\msto_1}{\mexp_1}{\mval'}{\msto_2}\\
      \PDalt{\menv'[\mvar\mapsto\maddr]}{\msto_2\sqcup[\maddr\mapsto\mval']}{\mexp}{\mval}{\msto'}\\
      \mval \in \cache(\sapp{\mexp_0}{\mexp_1},\menv,\msto)\\
      \maddr \in \alloc(\dots)}
    {\PD{\sapp{\mexp_0}{\mexp_1}}{\mval}{\msto'}}
\end{mathpar}

This relation is also coinductive, and the least solution can also be
generated in the same way as before.  However the least cache that is
valid for a program under this relation is potentially smaller than
NNH because this allows the proper nesting of calls and returns.

This abstract evaluator looks a lot more like a big-step, environment
and store-based evaluator.  

\section{Pushdown NNH 0CFA}

Just to make this concrete by instantiating to a familiar
approximation, (Pushdown) 0CFA (with store-variance):

\begin{mathpar}
\inferrule
    {\mval \in \cache(\mvar,\msto)\\
      \mval \in \msto(\mvar)}
    {\cache \models_\msto \mvar \Downarrow \mval,\msto}

\inferrule
    {\slam\mvar\mexp \in \cache(\slam\mvar\mexp,\msto)}
    {\cache \models_\msto \slam\mvar\mexp \Downarrow \slam\mvar\mexp,\msto}

\inferrule
    {\cache \models_\msto \mexp_0 \Downarrow \slam\mvar\mexp, \msto_1\\
      \cache \models_{\msto_1} \mexp_1 \Downarrow \mval', \msto_2\\
      \cache \models_{\msto_2\sqcup[\mvar\mapsto\mval']} \mexp \Downarrow \mval, \msto'\\
      \mval \in \cache(\sapp{\mexp_0}{\mexp_1},\msto)}
    {\cache \models_\msto {\sapp{\mexp_0}{\mexp_1}} \Downarrow \mval,\msto'}
\end{mathpar}


\section{Non-termination}

What about terms that should be proven non-terminating by the
analysis?  E.g. $\Omega = \sapp{\slam\mvar{\mvar\mvar}}{\slam\mvar{\mvar\mvar}}$

What should $\cache$ hold for 0CFA?  Let $C$ be defined as:

\begin{align*}
C(\mvar,[\mvar \mapsto \{\slam\mvar{\mvar\mvar}\}]) &= \{\slam\mvar{\mvar\mvar}\}\\
C(\slam\mvar{\mvar\mvar},\emptyset) &= \{\slam\mvar{\mvar\mvar}\}\\
C(\mvar\mvar,[\mvar \mapsto \{\slam\mvar{\mvar\mvar}\}]) &= \emptyset\\
C(\sapp{\slam\mvar{\mvar\mvar}}{\slam\mvar{\mvar\mvar}},\emptyset) &= \emptyset
\end{align*}

This ought to be a good cache for $\Omega$, but it's not according to
the relation given so far (but it \emph{is} valid by NNH).
The reason is we have no $\mval$ to put as the result of $\Omega$.

We could just pick one, say $5$:
\[
C\models \Omega \Downarrow 5, \msto
\]

But this won't validate for $C$ as given, since it requires $5 \in
C(\Omega,\emptyset)$.  We could of course throw into $C$ anything
needed to make this result fall out.  But this implies there is no
least solution, since swapping $6$ for $5$ is just as small.

One option: add a notion of non-termination that has no value
associated with it.
\[
\cache \models^\menv_\msto \mexp \Uparrow
\]
such that
\[
C \models_\emptyset \Omega \Uparrow
\]

\begin{mathpar}
\inferrule
    {\cache\models^\menv_\msto \mexp_0 \Uparrow}
    {\cache\models^\menv_\msto {\sapp{\mexp_0}{\mexp_1}} \Uparrow}

\inferrule
    {\cache\models^\menv_\msto \mexp_0 \Downarrow \mval,\msto'\\
     \cache\models^\menv_{\msto'} \mexp_1 \Uparrow}
    {\cache\models^\menv_\msto {\sapp{\mexp_0}{\mexp_1}} \Uparrow}

\inferrule
    {\cache\models^\menv_\msto \mexp_0 \Downarrow \sclos{\slam\mvar\mexp}{\menv'},\msto'\\
     \cache\models^\menv_{\msto'} \mexp_1 \Downarrow \mval,\msto''\\
    \maddr = \alloc(...)}
    {\cache\models^{\menv'[\mvar\mapsto\maddr]}_{\msto''\sqcup[\maddr\mapsto\mval]} \mexp \Uparrow}
\end{mathpar}

This judgement lets you exclude results from the cache when an
expression provably loops.  Now the least valid cache for the example
(assuming 0CFA) is $C$.

[I worry this is unsound when there \emph{may} be a loop (but not
  must), because it allows you cut out parts of the cache that are
  needed.  It might need side conditions on $\cache$ being empty for
  the non-terminating expression.  Should look at an example with a
  false loop in it and work through the least analysis.]

\section{Soundness}

How do you prove soundness?  (You need to be able to talk about an
infinite table for the concrete semantics).

\section{Analysis}

\begin{align*}
\mathcal{A}(\mexp) = \mbox{ the least }C\mbox{ s.t.}&
\vdash_\emptyset \mexp \Downarrow \mval,\msto' \implies C\models_\emptyset \mexp \Downarrow \mval,\msto'\\
&
\vdash_\emptyset \mexp \Uparrow \implies C\models_\emptyset \mexp \Uparrow
\end{align*}

\end{document}
