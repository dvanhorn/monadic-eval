\documentclass[12pt]{article}
\usepackage{fullpage}
\usepackage{mathpartir}
\usepackage{times}
\usepackage{mathpartir}
\usepackage{amsmath}
\usepackage{amssymb}
\usepackage{amsthm}
\usepackage{color}
\usepackage{calc}
\usepackage{xspace}
\usepackage{pifont}
\usepackage{upgreek}
\usepackage{graphicx}
\usepackage{hyperref}

\newcommand\mproof{\nabla}
\newcommand\avals[3]{({#1},{#2},{#3})}
\newcommand\alloc{\widehat{\mathit{alloc}}}
\newcommand\cache{\widehat{C}}
\newcommand\mvar{x}
\newcommand\mlab{\ell}
\newcommand\mexp{e}
\newcommand\mval{v}
\newcommand\mans{a}
\newcommand\menv{\rho}
\newcommand\maddr{\ell}
\newcommand\msto{\sigma}
\newcommand\mlvar{\mvar^\mlab}
\newcommand\mce{{\mathit{ce}}}
\newcommand\mcntr{\delta}
\newcommand\slam[2]{(\lambda{#1}.{#2})}
\newcommand\sllam[2]{\slam{#1}{#2}^{\mlab}}
\newcommand\sapp[2]{({#1}\;{#2})}
\newcommand\slapp[2]{({#1}\;{#2})^\mlab}
\newcommand\sclos[2]{\langle{#1},{#2}\rangle}
\newcommand\NNH[1]{\NNHalt\mce\mcntr{#1}}
\newcommand\NNHalt[3]{\cache\models^{#1}_{#2} {#3}}

\newcommand\CES[1]{\NNHalt\menv\msto{#1}}
\newcommand\CESalt[3]{\NNHalt{#1}{#2}{#3}}
\newcommand\PD[3]{\CES{#1 \Downarrow #2,#3}}
\newcommand\PDalt[5]{\NNHalt{#1}{#2}{#3 \Downarrow #4,#5}}


\begin{document}
\begin{abstract}
This note explains how to present pushdown flow analysis in the style
of Nielsen, Nielsen and Hankin.
\end{abstract}
\section{Classic NNH}

Here is the NNH-style presentation of $k$-CFA:
\[
\begin{array}{lcl}
\NNH\mlvar
&\iff&
\cache(\mvar,\mce(\mvar)) \subseteq \cache(\mlab,\mcntr)
\\[2mm]
\NNH{\sllam\mvar\mexp}
&\iff&
\{\sclos{\slam\mvar\mexp}\mce\} \subseteq \cache(\mlab,\mcntr)
\\[2mm]
\NNH{\slapp{\mexp^{\mlab_0}}{\mexp^{\mlab_1}}}
&\iff&
\NNH{\mexp^{\mlab_0}}\\
&\wedge&
\NNH{\mexp^{\mlab_1}}\\
&\wedge&
\forall \sclos{\slam\mvar{\mexp^{\mlab'}}}{\mce_0} \in \cache(\mlab_0,\mcntr).\\
&&
\qquad \cache(\mlab_1,\mcntr) \subseteq \cache(\mvar,\mcntr)\  \wedge\\
&&
\qquad \NNHalt{\mce[\mvar\mapsto\mcntr']}{\mcntr'}{\mexp^{\mlab'}}\\
&&
\qquad \cache(\mlab',\mcntr') \subseteq \cache(\mlab,\mcntr)
\\
&&
\qquad \mbox{where }\mcntr' = \lfloor \mcntr\cdot\mlab \rfloor_k
\end{array}
\]


As I noted in my diss., if you squint, you'll see a ``big-step''
evaluator lurking within the formalism.  However it is a weird evaluator:
\begin{itemize}
\item it is stated as validity relation of a ``cache'' w.r.t.~a program,
\item it does not ``return'' values, instead it uses the cache to
  communicate returns, i.e. the abstract evaluation of
  $\NNH{\mexp^\mlab}$ is a value $\mval \in \cache(\mlab,\mcntr)$,
\item it does not ``bind'' variables, instead it uses the cache to
communicate bindings,
\end{itemize}

The validity relation is coinductive, implying that to check the
validity of a cache, you must coinductively memoize validation to
avoid looping.

To \emph{construct} the minimal cache that satisfies the validity
relation, an easy algorithm is as follows:

\begin{enumerate}
\item Start with the empty cache.
\item Try to validate the cache,
\begin{enumerate}
\item if it fails, add the minimal information needed to validate the rule that failed, try again.
\item otherwise return the cache.
\end{enumerate}
\end{enumerate}

Since the call and return mechanism is intertwined with a finite
cache, it does not properly model function calls.  This leads to a
imprecision that is fundamental to the model (i.e. there are sound
analyses (caches), which are not validated by this judgment).

\section{Pushdown NNH}

First of all, let's just assume syntax is identified up to source
location in a program (i.e. we implicitly assume a unique labelling
and ommit labels in the model).

Second, we'll use a bonafide heap and environment, but make the usual
restriction that the heap is finite and maps to a set of values.

Third, let's ditch the idea of a cache modeling the result of
analysis.  We'll take the set of (abstract) evaluations of the program
as the notion of analysis.


Abstract evaluation looks just like the natural semantics for the
language:
\[
\avals\mexp\menv\msto \Downarrow \mans
\]
The only exceptions are heaps are finite and store sets of values and
heap update is interpreted as a join.

\begin{mathpar}
\inferrule
    { \mval \in \msto(\menv(\mvar))}
    {\avals\mvar\menv\msto \Downarrow (\mval,\msto)}

\inferrule
    { }
    {\avals{\slam\mvar\mexp}\menv\msto \Downarrow (\sclos{\slam\mvar\mexp}\menv,\msto)}

\inferrule
    {\avals{\mexp_0}\menv\msto \Downarrow (\sclos{\slam\mvar\mexp}{\menv'},\msto_1)\\
      \avals{\mexp_1}\menv{\msto_1} \Downarrow (\mval',\msto_2)\\
      \avals\mexp{\menv'[\mvar\mapsto\maddr]}{\msto_2\sqcup[\maddr\mapsto\mval']} \Downarrow \mans\\
      \maddr \in \alloc(\dots)}
    {\avals{\sapp{\mexp_0}{\mexp_1}}\menv\msto \Downarrow \mans}
\end{mathpar}

We also specify a relation for (abstractly) divergent programs:

\begin{mathpar}
\inferrule
    {\avals{\mexp_0}\menv\msto \Uparrow}
    {\avals{\sapp{\mexp_0}{\mexp_1}}\menv\msto \Uparrow}

\inferrule
    {\avals{\mexp_0}\menv\msto \Downarrow \mval,\msto'\\
      \avals{\mexp_1}\menv{\msto'} \Uparrow}
    {\avals{\sapp{\mexp_0}{\mexp_1}}\menv\msto \Uparrow}    

\inferrule
    {\avals{\mexp_0}\menv\msto \Downarrow (\sclos{\slam\mvar\mexp}{\menv'},\msto_1)\\
      \avals{\mexp_1}\menv{\msto_1} \Downarrow (\mval',\msto_2)\\
      \avals\mexp{\menv'[\mvar\mapsto\maddr]}{\msto_2\sqcup[\maddr\mapsto\mval']} \Uparrow\\
      \maddr \in \alloc(\dots)}
    {\avals{\sapp{\mexp_0}{\mexp_1}}\menv\msto \Uparrow}
\end{mathpar}


These relations are both co-inductively defined (usually only
divergence is co-inductive, but the approximation of the evaluation
requires it to be co-inductive as well).

The relations are over a finite set of objects, so, e.g., the set
$
\{ \mval\ |\ \avals\mexp\emptyset\emptyset \Downarrow \mval \}
$
is clearly computable.

It's possible to have an infinite number of proofs of
$\avals\mexp\menv\msto \Downarrow \mval$ or $\avals\mexp\menv\msto
\Uparrow$, which correspond to unrollings of the co-inductive
hypothesis an arbitrary number of times, however we can consider a
single canonical proof, which never unrolls co-IH, ie. we identify the
following proof structures:
\begin{mathpar}
\infer[co-IH]{A}
      {\infer{\vdots}
        {A}}

\equiv

\infer[co-IH]{A}
{\infer{\vdots}
  {\infer{A}
    {\infer{\vdots}
      {A}}}}
\end{mathpar}

Let $\mproof$ range over proofs and write $\mproof :
\avals\mexp\menv\msto \Downarrow \mans$ when $\mproof$ is a proof of
$\avals\mexp\menv\msto \Downarrow \mans$ and $\mproof :
\avals\mexp\menv\msto \Uparrow$ when $\mproof$ is a proof of
$\avals\mexp\menv\msto \Uparrow$.


Analysis is defined as:
\[
\mathcal{A}(\mexp) = \{ \mproof : \avals\mexp\emptyset\emptyset \Downarrow \mans \}
\cup
\{ \mproof : \avals\mexp\emptyset\emptyset \Uparrow \}
\]
Under the equivalence given above, $\mathcal{A}(\mexp)$ is finite.

\section{Concrete semantics}

The concrete semantics are obtained by replacing $\alloc$ with a fresh
allocation strategy (and dropping the requirement of a bounded heap).

Convergence and divergence are exclusive (unlike in analysis).
%
It is not computable to decide which applies.
%
Some programs have no finite proof of convergence or divergence
(e.g. an infinitely allocating loop).

Proving soundness of a program assuming it either (provably) converges
or diverges should be straightforward, but we don't have a real
semantics for the programs that neither terminate nor revisit the same
state.

One possibility is to think of such a program as being given by a
series of proofs with a single hole in it.  The hole represents the
place where the program continues to evaluate.  (This is just
introducing the idea of a evaluation context into the world of natural
semantics, I think, and talking about one (big) step reduction.

\begin{mathpar}

\infer[Ar]
      {\avals{\mexp_0}\menv\msto \Downarrow \top}
      {\avals{\sapp{\mexp_0}{\mexp_1}}\menv\msto \Downarrow \top}

\infer[Fn]
      {\avals{\mexp_0}\menv\msto \Downarrow (\mval,\msto')\\
  \avals{\mexp_1}\menv{\msto'} \Downarrow \top}
      {\avals{\sapp{\mexp_0}{\mexp_1}}\menv\msto \Downarrow \top}

\infer[Hole]{ }
      {\avals{\sapp{\mexp_0}{\mexp_1}}\menv\msto \Downarrow \top}
\end{mathpar}


For any proof of $\avals\mexp\menv\msto \Downarrow \top$, there is
exactly one subproof of Hole.

Define $\mproof \sqsubseteq \mproof'$ as
\[
\mproof \sqsubseteq \mproof' \mbox{ if } \exists\mproof'' . \mproof \equiv [\mproof''/\mathit{Hole}]\mproof'
\]

For every proof $\mproof$, there is a unique sequence
\[
\mproof_0 \sqsubset \mproof_1 \sqsubset ... \sqsubset \mproof_n = \mproof
\]

We can define evaluation as
\[
\mathcal{E}(\mexp) = \{ \mproof : \avals\mexp\emptyset\emptyset \Downarrow \top \}
\cup \{ \mproof : \avals\mexp\emptyset\emptyset \Downarrow \mans \}
\cup \{ \mproof : \avals\mexp\emptyset\emptyset \Uparrow \}
\]

This is a potentially infinite set.  We can prove
\[
\mproof \in \mathcal{E}(\mexp) \implies
\exists \mproof' \in \mathcal{A}(\mexp) .
\mproof \in \gamma(\mproof')
\]
where $\gamma$ is a not unsurprising concretization map from abstract
evaluation proofs to sets of concrete evaluation proofs.

\end{document}
